\chapter{Sprint: 3}
Following sprint 2 where we focused on improving the stability of the underlying
libraries of GIRAF, this sprint will have the same focus since there where a lot
of issues with the build tools. This meant that some tasks could not be resolved
as these are dependent on Artifactory. As such, the goals of the sprint have
been defined as follows:\nl

\say{Continuous integration need to be fully functional, such that gradle can
handle JAVA libraries corectly.}\nl

\say{Improve stability in libraries, accessibility and usability for apps, to
fulfill user stories 1, 4 and 6.(Med udgangspunkt i de nye user-stories???)}\nl

\say{Server and client library, continue implementation of functionality for
user stories 2, 3 and 4.}\nl

During the first week of this sprint, our build tools (Artifactory and Jenkins)
were repaired. Based on this breakthrough, our purpose in this sprint has been
redefined to implementing new features based on user feedback, and testing the
changes comitted during the first and second sprint. The user feedback in
question comes from a meeting group SW612 held with a group of users at the
during the second sprint. The relevant user stories are shown in
\autoref{sprint3userstories}.

\begin{table}[H]
\centering 
\begin{tabular}{|l|p{12.5cm}|}
\hline
Nr. & User Story \\\hline
1 & As a guardian I would like the option to toggle between grayscale and fully
colored in the system for a citizen. \\\hline
2 & As a guardian I would like not to have a Quick-Response (QR) codes for
login, but a regular password instead.\\\hline
3 & As a guardian I would like the ability to set a password for specific users
of the system.\\\hline
4 & As a guardian I would like the system to automatically log me out of my
account after a set time limit (approx. 20 min).\\\hline
5 & As a citizen I would like the system to not to log me out
automatically.\\\hline
\end{tabular}
\caption{User stories created from the meeting during the third sprint}
\label{sprint3userstories}
\end{table}


\section{Tasks and Solutions}
Based on the user stories presented in \autoref{sprint3userstories}, and the
need to implementing the REST server communication framework developed by SW613 and SW615, we have
been assigned the tasks shown in \autoref{Tasks3}.

\begin{table}[H]
\begin{centering}
\begin{tabular}{|l|p{9cm}|l|}
\hline
Number 	& Description & Man-hours \\ \hline
T634   	& Wiki: Continually update Launcher info on the wiki & 8\\\hline
T728  	& Pictoreader refactoring. & 60 \\ \hline
T739	& Wiki: Document Weekplanner and Pictosearch & 64\\ \hline
T785	& Handle crash such that the user does not get an error message and is
		  brought back to the launcher & 6\\ \hline
T786 	& Check if an app can be set to grayscale globally & 10\\ \hline
T800 	& Discuss how models should be implemented group SW609, SW610, SW613 and
		  SW615. & 16\\ \hline
T822 	& Implement grayscale functionality & ??\\ \hline
T830 	& Hide error splash screen and seperate it from analytics & 2\\ \hline
T832 	& Upload libraries to Artifactory & 12\\ \hline
T833 	& Update build scripts to use new Artifactory libraries & 2\\ \hline
T834 	& Implement new login screen (Beta branch) & 2\\ \hline
T835 	& Implement model-view-controller structure & 4\\ \hline
T836 	& Remove deprecated libraries/implement new rest-lib & 60\\ \hline
T837 	& Rework buttons to not use listeners & 6\\ \hline
\end{tabular}
\caption{Tasks for the third sprint}
\label{Tasks3}
\end{centering}
\end{table}

\subsection{T785, T830 - Prevent stack-trace from showing for users}
Based on general feedback, we have been tasked with making sure that the
detailed error-messages are not shown to the users. The users have requested
this, as the citizen users were confused/uncomfortable with the large amount of
incomprehensible information being suddenly shown on the screen whenever an
error occoured.\nl

\fix{}{Fjernet kaldet til bugSplashActivity, og forsøgt at restarte
applikationen}


\subsection{T786, T822 - Research and implement grayscale functionality}
Based on user story 1, which is shown in \autoref{sprint3userstories}, we have
been tasked with researching how it would be possible to make a global implementation of
grayscale in the GIRAF project.\nl

Following our research, we came up with three different approaches to
implmenting grayscale:

\begin{enumerate}
  \item Create an alternative set of resources for all giraf components, which
  make use of a grayscale color scheme.
  \item Apply an opaque layer to all GIRAF applications, which would make all
  components less colorful and more gray.
  \item Reduce the saturation of all GIRAF component's colors, such that all
  applications would look gray.
\end{enumerate}

Based on the fact that we want to create a global option which is forward
compatible with future GIRAF applications, and the fact that we want
to minimize the amount of added resourses, we have chosen the third
option, which is to change the saturation of all GIRAF applications.\nl

We have chosen to implement the grayscale functionality in the
\textc{giraf-component-lib} library, as this library is included in all GIRAF
applications. As such, the core of the implementation is made in the class
\textc{GrayScaleHelper}. Changing a given screen to grayscale consists of two
steps; changing the activity, and changing the action bar at the top.\nl

\begin{minipage}[H]{\linewidth}
\begin{lstlisting}[caption = Code used to determine the views for the activity and the top action bar., label = grayEx1] 
public class GrayScaleHelper
{

    public static void setGrayScaleForActivity(Activity activity, boolean shouldBeGray){
        View screen = activity.getWindow().getDecorView();
        int barId = activity.getResources().getIdentifier("action_bar_container", "id", "android");
        View bar = screen.findViewById(barId);
        View view = activity.findViewById(android.R.id.content);
        setGray(view,shouldBeGray);
        setGray(bar,shouldBeGray);
    }
...
}
\end{lstlisting}
\end{minipage}

As can be seen in \autoref{grayEx1}, the first step is to identifying the
\textc{views} containing the \textc{activity} and the top \textc{action bar}
respectively. The method \textc{setGrayScaleForActivity} takes a reference to
the calling \textc{activity}, and a boolean determining if the \textc{activity}
should be grayscale. On \textbf{line 5}, the \textc{view} for the background
\textc{android} screen is retrieved. On \textbf{line 6}, the ID for the
\textc{action bar} is determined. On \textbf{line 7}, the \textc{view} for the
\textc{action bar} is determined based on the ID. On \textbf{line 8}, the
\textc{view} for the \textc{activity} is determined. Finally on \textbf{lines 9-10}, the \textc{setGray}
method is called with the \textc{view} for the \textc{activity} and the
\textc{action bar} respectively. The \textc{setGray} method can be seen in
\autoref{grayEx2}.\nl

\begin{minipage}[H]{\linewidth}
\begin{lstlisting}[caption = Method used to change to color saturation of views., label = grayEx2]
public class GrayScaleHelper {
...
    private static void setGray(View view, boolean state){
        if(state) {
            ColorMatrix cm = new ColorMatrix();
            cm.setSaturation(0); //Set the color saturation to 0
            Paint grayscalePaint = new Paint();
            grayscalePaint.setColorFilter(new ColorMatrixColorFilter(cm));
            // Create a hardware layer with the greyscale paint
            view.setLayerType(LAYER_TYPE_HARDWARE, grayscalePaint);
        }else{
            //Removes the grayscale
            view.setLayerType(LAYER_TYPE_NONE, null);
        }
    }
}
\end{lstlisting}
\end{minipage}

In the \textc{setGray} metod in \autoref{grayEx2}, we first use the boolean
\textc{state} to determine if we should set the \textc{view} to grayscale. This
boolean represents the value of a \textc{switch button} in the settings menu,
which indicates if the option is turned on or off. \textbf{Lines 5-8} are used
to create an android \textc{Paint}, which has the color saturation of value
zero. On \textbf{line 10}, a \textc{hardware layer} using the defined
\textc{paint} is added to the supplied \textc{activity}.\nl

In order for a given activity to implement grayscale, it needs to have a call to
the \textc{setGrayScaleForActivity} method. An example of this call is
implemented in the \textc{Launcher's HomeActivity} class, and can be seen in
\autoref{grayEx3}.\nl

\begin{minipage}[H]{\linewidth}
\begin{lstlisting}[caption = Setting grayscale by calling the setGrayScaleForActivity method, label = grayEx3] 
private void setGrayscale(){
        SharedPreferences prefs = SettingsUtility.getLauncherSettings(this,
            LauncherUtility.getSharedPreferenceUser(currentUser));

        GrayScaleHelper.setGrayScaleForActivity(this, prefs.getBoolean(
            getString(R.string.toggle_gray_scale_preference_key), true));
    }
\end{lstlisting}
\end{minipage}

In \autoref{grayEx3}, we start by retrieving the GIRAF launcher's shared
preferences. We then use the boolean
\textc{toggle\_gray\_scale\_preference\_key's} value in the \\
\textc{setGrayScaleForThisActivity} method, to set grayscale for this
application. While not implemented yet, the intention is to supply the boolean
value to other GIRAF applications through the \textc{android intent}, which is
used to launch those applications.\nl

As stated above, in order to change the grayscale setting through the GUI, we
have added a \textc{switch button}. This settings menu with the added button can
be seen in \autoref{grayButton}, where grayscale has been enabled. 

\figPH{grayButton}{The switch as shown in the Launcher's settings menu
\fix{}{Husk at fjerne placeholder}}


\subsection{T832 - Upload new libraries to Artifactory}

Before was Jenkins used to upload compiled libraries to Artifactory, the way
this worked was that the developers committed the code to Git and then went to
build it using Jenkins which took the code from Git and compiled it and if it
succed put it on Artifactory. \nl 

Now we build the libraries localy on our own machines and the upload them to
Artifactory using a shared build script which the server infrastructur group has
made.

Før lagde vi bare op på GIT, hvorefter Jenkins hentede dem fra GIT, compilede
dem, og lagde dem op på artifactory. Nu builder vi lokalt og ligger dem op på artifactory.\nl


Placeholder\ldots

\subsection{T833 - Implement new libraries from Artifactory}
To implement the new libraries from Artifactory, the gradle build script has
been changed to both use the new Artifactory maven repos and also the new names
of the libraries. It has also been implemeneted such that some libraries version
which changes often is downloaded when they are put up on Artifactory and a
gradle sync is performed.

Placeholder\ldots

\subsection{T834 - Implement new login screen (Beta branch)}
Based on T731, see \autoref{DesignLogin}, we have built the new gui for the
login screen. This design is based off an existing gui for the login screen
which was developed last semester. The design 

step by step:
- We used the design of T731
- We adapted an in indevelopment gui from a branch
- has no functionality as we are waiting on REST to be finished

new information as of 01-05-2017:
- We need the user to input: name, password, department
  since name is not unique
- We will need to either add a dropdown or an additional textview for the
  department
- can departments be saved as a cookie? (suggested text in the textview)


\subsection{T835 - Launcher: Implement model-view-controller}
Placeholder\ldots

\subsection{T836 - Remove deprecated libraries/implement new rest-lib}
Placeholder\ldots

\subsection{T837 - Rework buttons to not use listeners}
Placeholder\ldots

\section{Retrospective}
Placeholder\ldots
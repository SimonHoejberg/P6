\chapter{Sprint: 01}
As this sprint represents the groups' initial experiences with the GIRAF
project, the purpose of this initial sprint is to gain a better understanding of
the inner workings of the various GIRAF apps, and to make sure that the quality
of the previosly written code is up to par. As such, the tasks for this sprint
are mostly superficial, but they do also include a number of fixes to critical
bugs.\nl

As part of the initial organization of the project, each group was assigned a
general area of responsibility. For our group, SW609, this responsibility
encompasses the apps \textc{Launcher} and \textc{Pictogram Searcher}. While
these apps represent our core responsibilities, not all of our tasks are
required to revolve around these apps.

\section{Tasks}
Based on our area of responsibility and the backlog from last semesters'
students, we have chosen an initial list of tasks, which can be seen below in
\autoref{SprintOneTasks}. 

\begin{table}[H]
\centering
\begin{tabular}{|l|l|}
\hline
Number 			& Task Description 											\\\hline
T620  			& Search: Identify unhandled exceptions.                 	\\\hline
T621    	   	& Launcher: Identify unhandled exceptions. 					\\\hline 
T623	       	& Search: Handle unhandled exceptions.                 		\\\hline
T624    	   	& Launcher: Handle unhandled exceptions.  					\\\hline
T625   	   		& Launcher: Evaluate possible exceptions.        			\\\hline
T626   	   		& Search: Evaluate possible exceptions.                		\\\hline
T629  	   		& Search: Refactor code.			   						\\\hline
T630       		& Launcher: Refactor code.                 					\\\hline
T631       		& Planner and Launcher: CheckStyle                 			\\\hline
T636			& Identify the weekplanner's dependencies.					\\\hline
T639       		& Disable non-functional GUI elements.               		\\\hline
T646       		& Identify necessary resources.                 			\\\hline 
T647       		& Visualize client/server interface.                 		\\\hline 
T660       		& Solve the pictogram problem.                 				\\\hline
T670       		& Launcher: Fix crash on startup w/o internet or other apps.\\\hline
\end{tabular}
\caption{Tasks for the first sprint (Group SW609)} 
\label{SprintOneTasks}    
\end{table} 

In the following subsections, we will elaborate upon each of the task, and
document what actions we have taken in order to resolve the issues.

\subsection{T620, T623 - Search: Handle Exceptions}
Based on the backlog from the previous semester, we identified an unhandled
\\\textc{nullPointerException} in the \textc{PictoAdapter} class's
\textc{getView} method. This exception occurred when a used tried to enter a
comma into the pictogram search-bar. Due to faulty implementation, when the
search was executed on a comma, it returned an empty list of pictograms. Given
that the rest of the system never expects the pictogram list to be empty, this
led to a problem where the system tried to acces the first element of the list
which was null.\nl

In order to fix this problem, we have implemented multiple checks to make sure
that the system does not attempt to invoke any methods on the non-existing
object. Examples of these checks can be seen below in
\autoref{PictoListNullPoint}.\nl

\begin{minipage}[H]{\linewidth}
\begin{lstlisting}[caption = Newly implemented check for null values., label = PictoListNullPoint] 
...
final Object object = objectList.get(position);
Drawable catIndicator = context.getResources().getDrawable(R.drawable.icon_category);

if (convertView == null) {
	GirafPictogramItemView pictogramItemView;
    if (object != null && object instanceof Pictogram) {
    	Pictogram pictogramNew = (Pictogram) object;
        pictogramItemView = new GirafPictogramItemView(context, pictogramNew, pictogramNew.getName());
	} else if (object != null) {
	...
}
...
\end{lstlisting}
\end{minipage}

Checking the \textc{objectList} for null values solved the problem of trying to
access a non-existing object, but another problem is that the checked object is used to
modify and return a \textc{pictogramItemView}. As such, given the case where no
object exists, we need to identify if it is possible for the method to simply
return null or a dummy object. Through analysis of the
\textc{GirafPictogramItemView} class, we identified that the system is already
equipped to handle a \textc{pictogramItemView} with an uninitialized
\textc{imageEntity}. This can be seen below in \autoref{PictogramHandleNull}\nl

\begin{minipage}[H]{\linewidth}
\begin{lstlisting}[caption = \textc{setImageModel} method in the \textc{GirafPictogramItemView class.}, label = PictogramHandleNull] 
...
public synchronized void setImageModel(final ImageEntity imageEntity, final Drawable fallback) {
// If provided with null, do not update!
if (imageEntity == null) {
	return;
}
...
\end{lstlisting}
\end{minipage}

Based on this information, we have solved the problem by having the
\textc{getView} method return a newly made dummy object. This can be seen below
in \autoref{UnitImageEntity}.\nl

\begin{minipage}[H]{\linewidth}
\begin{lstlisting}[caption = Return dummy object in case of null-valued object., label = UnitImageEntity] 
...
if (convertView == null) {
	GirafPictogramItemView pictogramItemView;
    if (object != null && object instanceof Pictogram) {
    	...
    } else {
    	ImageEntity imageEntity = null; //The constructer doesn't accept a null as the second par 
    	pictogramItemView = new GirafPictogramItemView(context, imageEntity); 
  	}
...
\end{lstlisting}
\end{minipage}

\subsection{T621, T624, T670 - Launcher: Handle Exceptions}
During inspection of the Launcher's code, we identified two serious problems,
which both were capable of leading to crashes.

\subsubsection{LoadAnimation Crash}
The first problem was found in the launcher's \textc{MainActivity}, where a
global variable \textc{loadAnimation} was overshadowed by a local one defined in
the \textc{onCreate} method. As \textc{onCreate} is executed on creation of the
activity, the definition and assignment of \textc{loadAnimation} was contained
to this method, and the global variable was never initialized. This can be seen
below in \autoref{LoadAnimationCrash}. This problem led to a
\textc{nullPointerException} because the system tried to use an animation which
was not defined.\nl

\begin{minipage}[H]{\linewidth}
\begin{lstlisting}[caption = The variable \textc{loadAnimation} is overshadowed locally., label = LoadAnimationCrash] 
public class MainActivity extends
GirafActivity implements Animation.AnimationListener, GirafNotifyDialog.Notification { ...
    private Animation loadAnimation;
    ...
    
	public void onCreate(Bundle savedInstanceState) {
		...	
    	Animation loadAnimation;
		...
	}
...
}    
\end{lstlisting}
\end{minipage}

This problem was fixed by removing the local definition of \textc{loadAnimation}
in the \textc{onCreate} method.

\subsubsection{Package Manager Crash}
The second problem was caused by a timer, which was used to ping the Android
package manager, in order to determine if changes were made to the device's
installed applications. This can be seen below in \autoref{PackageTimer}. The
problem occours when running the launcher on an emulator, because pinging the package
manager would return a empty list of installed applications. Based on our
observations, an empty app-list would prompt the system to ping the package
manager again each time an empty list was returned (which was always). Over
time, this constant pinging would overload the package manager, and cause an
unhandled \textc{deadObjectException}.\nl

\begin{minipage}[H]{\linewidth}
\begin{lstlisting}[caption = The timer in the \textc{HomeActivity} class which pings the package manager., label = PackageTimer] 
private void startObservingApps() 
{ 
	appsUpdater = new Timer();
    AppsObserver timerTask = new AppsObserver();
    appsUpdater.scheduleAtFixedRate(timerTask, 5000, 5000);
 	...
}   	
\end{lstlisting}
\end{minipage}

This problem was fixed by implementing a new system to identify changes to the
list of installed applications. Instead of pinging the package manager on a
timer, we have implemented an Android \textc{broadcastListener}, which listens
for when the operating system broadcasts changes to the list of installed
applications. Otherwise, this listener is still set to execute the same code
that the previous timer would have. This new implementation can be seen below in
\autoref{broadcastListener}.\nl

\begin{minipage}[H]{\linewidth}
\begin{lstlisting}[caption = New implementation of a \textc{broadcastReceiver}., label = broadcastListener] 
private void startObservingApps() {
	//Makes sure this Receiver is only registered once
    if(!appObserverReceiverRegistered) {
    	BroadcastReceiver broadcastReceiver = new BroadcastReceiver() {
        @Override
        public void onReceive(Context context, Intent intent) {
        	appsChangedScan();
        }
    };
    //Sets up the filter to only trigger when these actions are
    received IntentFilter intentFilter = new IntentFilter();
    intentFilter.addAction(Intent.ACTION_PACKAGE_ADDED);
    intentFilter.addAction(Intent.ACTION_PACKAGE_REMOVED);
    registerReceiver(broadcastReceiver, intentFilter);
    appObserverReceiverRegistered = true;
    Log.d(Constants.ERROR_TAG, "Applications are being observed.");
	}
}
\end{lstlisting}
\end{minipage}

\subsection{T625 - Launcher: Evaluate Necessity of Present Exception Handling}
During analysis of the Launcher's code, we identified a number of bad practices
regarding implementation of exception handling. While these issues were not
dangerous to the program, they were examples of ineffecient implementations, and
therefore subject for refactoring.

\subsubsection{Explicit Exception Handling}
One of the most prevalent problems in the launcher's code is the explicitly
implemented exception handling. Instead of checking variable's for
null-values, a try-catch structure has been implemented to catch exceptions.
While this is technically not wrong, it is a bad practice, as the try-catch
approach is vastly more expensive in regards to required computations. There are
many examples of this approach in the code. An example from the
\textc{AppComparator} class can be seen below in
\autoref{ExplicitHandlingEx}.\nl

\begin{minipage}[H]{\linewidth}
\begin{lstlisting}[caption = An example of explicit exception handling., label = ExplicitHandlingEx] 
...
try {
	PackageManager packageManager = context.getPackageManager();
	String lhsName = ((ResolveInfo) lhs).activityInfo.loadLabel(packageManager).toString();
    String rhsName = ((ResolveInfo) rhs).activityInfo.loadLabel(packageManager).toString();
	res = lhsName.compareToIgnoreCase(rhsName);	
} catch (NullPointerException e) {

}
...
\end{lstlisting}
\end{minipage}

Our approach to solving this problem is to implement checks on each individual
variable, to make sure that its value is not null. The rewritten code from the
previous example in \autoref{ExplicitHandlingEx} can be seen below in
\autoref{HandlingSol}.\nl

\begin{minipage}[H]{\linewidth}
\begin{lstlisting}[caption = Our approach to variable checking., label = HandlingSol] 
...
PackageManager packageManager = context.getPackageManager();
if(packageManager != null) {
	String lhsName = ((ResolveInfo)lhs).activityInfo.loadLabel(packageManager).toString(); 
	String rhsName = ((ResolveInfo) rhs).activityInfo.loadLabel(packageManager).toString(); 
	if(lhsName != null && rhsName != null) {
    	res = lhsName.compareToIgnoreCase(rhsName);
    }
...
\end{lstlisting}
\end{minipage}

\subsubsection{Throwing Exceptions}
Another problem can be found in the \textc{LauncherUtility} class, where the
previous student have decided to create a try-catch structure in which they
themselves check the variables for null-values. Afterwards, if it value was
null, they throw their own \textc{ActivityNotFoundException}. This is done in
order to send the exception-log to Google Analytics. This can be seen below in
\autoref{FuckedExample}.\nl

\begin{minipage}[H]{\linewidth}
\begin{lstlisting}[caption = Throwing exceptions in order to send them to Google., label = FuckedExample] 
public static void secureStartActivity(Context context, Intent intent) { 
try {
	//If the activity exists, start it. Otherwise throw an exception.
    if (intent.resolveActivity(context.getPackageManager()) != null) {
    	context.startActivity(intent);
    } else {
    	throw new ActivityNotFoundException();
    }
} catch (ActivityNotFoundException ex) {
	// Sending the caught exception to Google Analytics
    LauncherUtility.sendExceptionGoogleAnalytics(context, ex);
 	...
}
...
\end{lstlisting}
\end{minipage}

We have deemed that this approach is unnecessary, and that a simpler approach
would be to simply check the variable for null-values through an if-statement.
Our revised code can be seen below in \autoref{FixedFuck}.\nl

\begin{minipage}[H]{\linewidth}
\begin{lstlisting}[caption = Our approach without throwing exceptions., label = FixedFuck] 
public static void secureStartActivity(Context context, Intent intent) {
	//If the activity exists, start it. Otherwise throw an exception.
	if (intent.resolveActivity(context.getPackageManager())!=null) {
		context.startActivity(intent);
	} else {
	//Display a toast, to inform the user of the problem.
	Toast toast = Toast.makeText(context, context.getString(R.string.activity_not_found_msg), Toast.LENGTH_SHORT);
	toast.show(); 
	Log.e(Constants.ERROR_TAG, "App could not be started");
...
\end{lstlisting}
\end{minipage}

\subsection{T631 - Planner and Launcher: CheckStyle}
In order to increase the quality and readability of the previously written code,
all the groups on this semester have chosen to make use of a generalized code
style, which will be enforced by the IntelliJ plugin CheckStyle.\\
As such, the purpose of this task is to used the CheckStyle plugin to identify
all instances where the code deviates from the agreed upon code style, and
manually refactor it such that the whole code base is uniformly written.\nl

When CheckStyle is executed on a project, it uses an XML file where the code
style is defined, to identify all cases where the code deviated from this
defined style. This information is presented in an ordered list. An example of
an error-list for the \textc{Launcher project} can be seen below in
\autoref{CodeStyleErrors}.

\figx[0.85]{CodeStyleErrors}{List of code style deviations in the Launcher
project.}

In order to solve the problems with the unsystematic code style, we have
methodically resolved the issues identified by CheckStyle.

\subsection{T646 - Identify Apps Required Resources}
At the beginning of the sprint we had a meeting with the other groups
reponsible for either applications or the database, namely SW610, SW613, and
SW615. This meeting was held in order to determine what database resources each
application would need. Through discussion with the other groups we compiled
\autoref{DataObjList}, which shows all relevant database resources, and what
application would need to make use of them. 

\begin{table}[H]
\centering
\begin{tabular}{|l|l|}
\hline
\textbf{Data}	& \textbf{Application}	\\\hline
Guardian		& Weekplanner 			\\\hline
Citizen    	   	& Weekplanner 			\\\hline 
Pictogram 		& Weekplanner 			\\\hline 
Empty    	   	& Weekplanner 			\\\hline
Choice   	   	& Weekplanner  			\\\hline
Day   	   		& Weekplanner 			\\\hline
Schedule  	   	& Weekplanner			\\\hline
Category       	& Pictosearch/Launcher	\\\hline
\end{tabular}
\caption{Data objects on the database, and the applications which use them.} 
\label{DataObjList}    
\end{table}

Based on these resources, we shortly discussed how these resources would most
effeciently be packaged together, and how each application should access
them.\nl

Any further work relating to optimization of the database's handling of
resources has been relegated to the database groups SW613 and SW615. As such, we
consider our part of this task to be completed.

\subsection{T647 - Document Apps' Database Requirements}
Through a meeting with the groups responsible for the databases, we made the
following UML diagram, modelling the client-server interface.

\figx{DatabaseUML}{UML diagram modelling the client-server interface.}

\subsection{T660 - Decrease Pictogram Search-Time}
Through analysis of the problem, we have concluded that the pictogram search has
the possibility of having slow load times beacuase the server returns all
elements related to the search in one large package. This can result in a large
load time, as the program does not respond while waiting for the database to
send the package. \nl

In addition, because of the way the search functionality is
implemented, if a user is slow to write, the database receives multiple requests
for packages to send. As an example, if a user wants to find all pictograms
relating to \textc{cars}, the program may queue multiple requests: one for
all pictograms starting with \textc{c}, then for all pictograms starting with
\textc{ca}, and so on. This queueing of requests can make the program wait for a
very long time before responding.\nl

Due to the fact that main problem lies with the server sending all pictograms in
one large package, we have chosen to forward this task to the group responsible
for the server/database, namely group SW613. In addition, during a meeting with
the other groups, we recommended that the server might send the pictograms in
packages of 20 each, such that the user may receive feedback in the form of a
continously updating list of pictograms.
 
\subsection{Removed Tasks}
Throughout the sprint, we have revised our list of tasks, and identified
that a number of them were outside the scope of our group's responsibility.
When such a task was identified and deemed non-trivial, these tasks were either
reported to the SCRUM Master group (SW611), or we made sure that another group
was taking care of it. These tasks are listed below.

\begin{itemize}
  	\item T626 - Search: Evaluate Necessity of Present Exception Handling
		\begin{enumerate}
	  		\item The purpose of this task was to evaluate the currently implemented
	  		exception handling, and modify/remove it, if it was redundant or low
	  		quality.
 			\item No unnecessary exception handling was identified in the pictosearch
 			library. The task has been marked as finished.
		\end{enumerate}
  	\item T639 - Disable non-functional GUI elements.
		\begin{enumerate}
 			\item The purpose of this task was to identify and remove any elementes of
 			the graphical user interface, which could either cause crashes, or give
 			access to non-functional elements of the program. This is done in order to
 			create a ``minimum viable product''.
 			\item None such elements were found in the system. This task has been marked
 			as finished.
		\end{enumerate}
	\item T636 - Weekplanner: Identify Functional Requirements
		\begin{enumerate}
 			\item \fix{}{We actually need to do this as a part of out collaboration.}
		\end{enumerate}
	\item T629, T630 - Search/Launcher: Refactor Code
		\begin{enumerate}
  			\item The purpose of this task were to make sure that the entire code base
  			adhered to some standard of quality. This was deemed neccessary in order to
  			prepare code for future modifications.
  			\item This task has been implicitly completed through the work on the other
  			tasks. Refactoring in regards to code style was done as a part of
  			\textc{T631}, and general changes to functionality was done as a part of
  			evaluating existing exception handling in \textc{T625} and \textc{T626}.
		\end{enumerate}

\end{itemize}


% How did we identify the relevant tasks?
% Our focus area.
% Our Tasks.
% -----
% Solution of tasks

\section{Retrospective}
Following the finalization of the first sprint, we met with the other
groups. The meeting was used in order to reflect upon the
development/organizational process used throughout the sprint, as well as
discuss the problems which occured throughout the sprint. It was determined
that everyone was satisfied with the current Scrum process of meeting once a
week to discuss the progress of each group as well as the various issues each
group faced.\nl

\subsection{Problems and Solutions}
During the sprint several issues were encountered. The first issue is that some
tasks did not have a descriptive title nor a well defined explanation. This lead
to some questions back and forth between the Scrum group and other groups in
order to figure out what the task meant. Another problem occured as a result of
several groups working together on and sharing the same task on phabricator.
When one group declared the task as resolved on phabricator, it would be
resolved for the other groups as well. As a result it was agreed that when
multiple groups work on the same tasks they should consult each other as to
when it can classified as resolved. A problem which the groups
working on the applications faced was that it is impossible to test the
solutions, \fix{}{add problem with Jenkins, artifactory and gradle.}\nl

During the sprint it agreed that each group should post a short a daily update
to keep the SCRUM-team up to date on what each group was doing. This was not a
success as it resulted in what several groups felt like unnecessary overhead, as
quite a lot of work was done on refactoring, implementing the code-style and
other tasks that did not quite result in a meaningful daily update. As
a result it was decided to drop the daily updates as it was very rarely used.\nl

It was decided that every task should have an estimate of how many man-hours it
takes to finish attached to it. This should be estimated when the task is
created and if proven unrealistic can be changed. This is done to better measure
productivity and workload throughout the sprint. It also helps reducing the
apparent workload that comes with having alot of tasks. This is one of the minor
problems we faced this sprint, which left us with suprisingly much time towards
the end of the sprint.

\subsection{Code Review}
blabla



\subsection{Evaluation}

The sprint ended with 55.3\% of tasks being declared resolved. Alot of the
unresolved tasks have solutions but have not been tested since Jenkins,
Artifactory and Gradle currently cannot be used to build the project and are
being worked on. The burndown chart for this sprint was set up the wrong way,
this will be fixed for the next sprint, see \autoref{S1BurndownChart}.

\figx[0.6]{S1BurndownChart}{The burndown chart for the first sprint}

As su


The sprint is evaluated as a success despite4


type checking happens on runtime which is hard to debug.
Add new task to next sprint
only 55.3 \% of the tasks that should have been done in the sprint has been
done - some were not finished due to lack of testing due to gradle

Add burndown chart



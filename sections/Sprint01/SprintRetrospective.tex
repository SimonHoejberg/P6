\section{Retrospective}
Following the finalization of the first sprint, we held a meeting with the other
groups in order to reflect upon the development/organizational process used
throughout the sprint. Based on this meeting, this chapter will be used to
discuss the contents of the first sprint from the perspective the multi-project and our
group.

\subsection{Multi-Project Retrospective}



more clear description of the tasks, some have multiple meanings

Groups should be better at giving updates on what they have been doing.
Daily updates will be dropped - since it seems like unnecessary overhead.

Tasks shared among groups should be agreed on being done before anyone changes the status.
When the man-hours for a task must change, the group should add a comment for why and how they have changed.
The development team seems satisfied with the Scrum methodology used so far.

What has been done?
grp 9:
Have worked with the launcher to resolve issues
Have debugged the pictosearch application to resolve exceptions.
Needs to test their solutions, but need Jenkins/Artifactory to work before they can
grp 10:
Have worked with the weekplanner application.
They have removed at least a thousand lines of redundant code.
They have solved some issues in the user interface.
grp 11:
Have worked on phabricator
Worked on creating a code review process
Worked on Scrum management
grp 12:
They set up a meeting with the customer which will be conducted in Sprint 2
Updated the wiki
grp 13:
Uniform interface to library for applications
Gained an understanding of Gradl, and can be used in testing
Artifactory is running, grp 09 can’t test because they have dependencies
started implementing the interface for the client library
grp 14:
received a master server from ITS has come - it has been updated.
Artifactory error has been identified, but is expected to take much time to handle.
Cannot build on Jenkins, it has been shut down.
Everything has been put in containers with scripts that can fascilitate building
Configuration files
grp 15:
investigated how the rest server worked in java
changed to asp.net core because some limitations in java. This seems to take longer than expected.
Jenkins has been shut down.

General:
Does not crash on normal use but is very unstable in edge-cases
The burndown chart is bad, as it was set up in a wrong way. The burndown chart for the next sprint will be more correct.
Other stuff:
type checking happens on runtime which is hard to debug.
Add new task to next sprint
only 55.3 \% of the tasks that should have been done in the sprint has been
done - some were not finished due to lack of testing due to gradle

Add burndown chart


\subsection{Group Retrospective}

On the group level we discussed the sprint, this 


%Alot of tasks this sprint was about getting familiar with the code and 
%Lots of small tasks that were solved quickly leaving us with more downtime than
%expected

Alot of the tasks this sprint was about getting familiar through exception
handling, code refacturing and verifing the code standard. Alot of these tasks
took alot less time than expected because the launcher and pictosearch
applications were in an acceptable state, with only a couple of crashes and a
few try-catches. These tasks was finished quickly since we could not build
and test our solutions. This resulted in\ldots 


Problems with building the project since Jenkins and Artifactory does not work

more clear description of the tasks, some have multiple meanings

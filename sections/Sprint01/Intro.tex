\chapter{Sprint: 01}
As this sprint represents the groups' initial experiences with the GIRAF
project, the purpose of this initial sprint is to gain a better understanding of
the inner workings of the various GIRAF apps, and to make sure that the quality
of the previosly written code is up to par. As such, the tasks for this sprint
are mostly superficial, but they do also include a number of fixes to critical
bugs.\nl

As part of the initial organization of the project, each group was assigned a
general area of responsibility. For our group, SW609, this responsibility
encompasses the apps \textc{Launcher} and \textc{Pictogram Searcher}. While
these apps represent our core responsibilities, not all of our tasks are
required to revolve around these apps.

\section{Tasks}
Based on our area of responsibility and the backlog from last semesters'
students, we have chosen an initial list of tasks, which can be seen below in
\autoref{SO1:Tasks}. 

\begin{table}[H]
\centering
\begin{tabular}{|l|l|}
\hline
Number 			& Task Description 											\\\hline
T620  			& Search: Identify unhandled exceptions.                 	\\\hline
T621    	   	& Launcher: Identify unhandled exceptions. 					\\\hline 
T623	       	& Search: Handle unhandled exceptions.                 		\\\hline
T624    	   	& Launcher: Handle unhandled exceptions.  					\\\hline
T625   	   		& Launcher: Evaluate possible exceptions.        			\\\hline
T626   	   		& Search: Evaluate possible exceptions.                		\\\hline
T629  	   		& Search: Refactor code.			   						\\\hline
T630       		& Launcher: Refactor code.                 					\\\hline
T631       		& Planner and Launcher: CheckStyle                 			\\\hline
T636       		& Identify requirements for Weekplanner to function.   		\\\hline 
T639       		& Disable visual elements.                 					\\\hline
T646       		& Identify necessary resources.                 			\\\hline
T647       		& Visualize client/server interface.                 		\\\hline 
T660       		& Solve the pictogram problem.                 				\\\hline
T670       		& Launcher: Fix crash on startup w/o internet or other apps.\\\hline
\end{tabular}
\label{S01:Tasks}  
\caption{Tasks for the first sprint (Group SW609)}
\end{table}

In the following subsections, we will elaborate upon each of the task, and
document what actions we have taken in order to resolve the issues.

\subsection{T620, T623, T626 - Search: Handle Exceptions}
Placeholder\ldots

\subsection{T621, T624, T625 - Launcher: Handle Exceptions}
Placeholder\ldots

\subsection{T629 - Search: Refactor Code}
Placeholder\ldots

\subsection{T630 - Launcher: Refactor Code}
Placeholder\ldots

\subsection{T631 - Planner and Launcher: CheckStyle}
In order to increase the quality and readability of the previously written code,
all the groups on this semester have chosen to make use of a generalized code
style, which will be enforced by the IntelliJ plugin CheckStyle.\\
As such, the purpose of this task is to used the CheckStyle plugin to identify
all instances where the code deviates from the agreed upon code style, and
manually refactor it such that the whole code base is uniformly written.\nl

When CheckStyle is executed on a project, it uses an XML file where the code
style is defined, to identify all cases where the code deviated from this
defined style. This information is presented in an ordered list. An example of
an error-list for the \textc{Launcher project} can be seen below in
\autoref{CodeStyleErrors}.

\figx[0.85]{CodeStyleErrors}{List of code style deviations in the Launcher
project.}

In order to solve the problems with the unsystematic code style, we have
methodically resolved the issues identified by CheckStyle.

\subsection{T636 - Weekplanner: Identify Functional Requirements}
Placeholder\ldots'


\subsection{T636 - Disable Problematic Visual Elements}
Placeholder\ldots

\subsection{T646 - Identify Apps Required Resources}
Placeholder\ldots

\subsection{T647 - Document Apps' Database Requirements}
Placeholder\ldots

\subsection{T660 - Decrease Pictogram Search-Time}
Placeholder\ldots

\subsection{T670 - Fix Package Manager Access Crash}
Placeholder\ldots


% How did we identify the relevant tasks?
% Our focus area.
% Our Tasks.
% -----
% Solution of tasks

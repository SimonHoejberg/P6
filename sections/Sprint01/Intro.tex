\chapter{Sprint: 01}
As this sprint represents the groups' initial experiences with the GIRAF
project, the purpose of this initial sprint is to gain a better understanding of
the inner workings of the various GIRAF apps, and to make sure that the quality
of the previosly written code is up to par. As such, the tasks for this sprint
are mostly superficial, but they do also include a number of fixes to critical
bugs.\nl

As part of the initial organization of the project, each group was assigned a
general area of responsibility. For our group, SW609, this responsibility
encompasses the apps \textc{Launcher} and \textc{Pictogram Searcher}. While
these apps represent our core responsibilities, not all of our tasks are
required to revolve around these apps.

\section{Tasks}
Based on our area of responsibility and the backlog from last semesters'
students, we have chosen an initial list of tasks, which can be seen below in
\autoref{SO1:Tasks}. 

\begin{table}[H]
\centering
\begin{tabular}{|l|l|}
\hline
Number 			& Task Description 											\\\hline
T620  			& Search: Identify unhandled exceptions.                 	\\\hline
T621    	   	& Launcher: Identify unhandled exceptions. 					\\\hline 
T623	       	& Search: Handle unhandled exceptions.                 		\\\hline
T624    	   	& Launcher: Handle unhandled exceptions.  					\\\hline
T625   	   		& Launcher: Evaluate possible exceptions.        			\\\hline
T626   	   		& Search: Evaluate possible exceptions.                		\\\hline
T629  	   		& Search: Refactor code.			   						\\\hline
T630       		& Launcher: Refactor code.                 					\\\hline
T631       		& Planner and Launcher: CheckStyle                 			\\\hline
T636       		& Identify requirements for Weekplanner to function.   		\\\hline 
T639       		& Disable visual elements.                 					\\\hline
T646       		& Identify necessary resources.                 			\\\hline
T647       		& Visualize client/server interface.                 		\\\hline 
T660       		& Solve the pictogram problem.                 				\\\hline
T670       		& Launcher: Fix crash on startup w/o internet or other apps.\\\hline
\end{tabular}
\label{S01:Tasks}  
\caption{Tasks for the first sprint (Group SW609)}
\end{table}

In the following subsections, we will elaborate upon each of the task, and
document what actions we have taken in order to resolve the issues.

\subsection{T620, T623, T626 - Search: Handle Exceptions}
Based on the backlog from the previous semester, we identified an unhandled
nullPointerException, which occurred when a used tried to enter a comma into the
pictogram search-bar. Due to faulty implementation, when the search was executed
on a comma, it returned an empty list of pictograms. Given that the rest of the
system never expects the pictogram list to be empty, this led to a problem where
the system tried to acces the first element of the list which was null.\nl

In order to fix this problem, we have implemented multiple checks to make sure
that the system does not attempt to invoke any methods on the non-existing
object. Examples of these checks can be seen below in
\autoref{PictoListNullPoint}.

\figx[0.80]{PictoListNullPoint}{Newly implemented check for non-existing
object.}
 
While this fixed the problem of trying to access a non-existing object, this
object was supposed to be used to return another object in the containing
method, \textc{getView}.
As such, given the case where no object exists, we need to identify if it is
possible for the method to simply return null, or if we are able to satisfy the
system by simply creating a dummy-object. As can be seen below in
\autoref{PictogramHandleNull}, the system is already equipped to handle an
object with uninitialized variables.

\figx[0.85]{PictogramHandleNull}{The setImageModel method can already handle
unitialized variables.}

Based on this information, we have solved the problem by having the
\textc{getView} method return a newly made dummy object. This can be seen below
in \autoref{UnitImageEntity}.

\figx[0.80]{UnitImageEntity}{Given a null object, create a new pictogramItemView
to return.}

\subsection{T626 - Search: Evaluate Necessity of Present Exception Handling}
Placeholder\ldots

\subsection{T621, T624 - Launcher: Handle Exceptions}
During inspection of the Launcher's code, we identified two serious problems,
which both were capable of leading to crashes.

\subsubsection{LoadAnimation Crash}
The first problem was found in the launcher's \textc{MainActivity}, where the
problem was that a global variable \textc{loadAnimation} was overshadowed by a
local one defined in the \textc{onCreate} method. As \textc{onCreate} is
executed on creation of the activity, the definition and assignment of
\textc{loadAnimation} was contained to this method, and the global variable was
never initialized. This problem led to a nullPointerException because the system
tried to access an animation which was not defined.\nl

\fix{}{Insert code example}

This problem was fixed by removing the local definition of \textc{loadAnimation}
in the \textc{onCreate} method.

\subsubsection{Package Manager Crash}
The second problem was caused by a code-segment implemented by the previous
semesters' students, which used a timer to ping the Android package manager, in
order to determine if changes were made to the device's installed applications.
The problem occours when running the launcher on an emulator, because pinging
the package manager would return a empty list of installed applications. Because
of the implementation seen below in \autoref{}, an empty list would prompt
the program to ping the package manager again each time an empty list was
returned (which was always). Over time, this constant pinging would overload
the package manager, and cause an unhandled \textc{deadObjectException}.\nl

\fix{}{Insert code example}\nl

This problem was fixed by implementing a new system to identify changes to the
list of installed applications. Instead of pinging the package manager on a
timer, we have implemented an Android \textc{broadcastListener}, which listens
for when the operating system broadcasts that there has been a change to the
list of installed applications. Otherwise, this listener is still set to
execute the same code that the previous timer would have. This new
implementation can be seen below in \autoref{}.\nl

\fix{}{Insert code example}

\subsection{T625 - Launcher: Evaluate Necessity of Present Exception Handling}
Placeholder\ldots

\subsection{T629 - Search: Refactor Code}
Placeholder\ldots

\subsection{T630 - Launcher: Refactor Code}
Placeholder\ldots

\subsection{T631 - Planner and Launcher: CheckStyle}
In order to increase the quality and readability of the previously written code,
all the groups on this semester have chosen to make use of a generalized code
style, which will be enforced by the IntelliJ plugin CheckStyle.\\
As such, the purpose of this task is to used the CheckStyle plugin to identify
all instances where the code deviates from the agreed upon code style, and
manually refactor it such that the whole code base is uniformly written.\nl

When CheckStyle is executed on a project, it uses an XML file where the code
style is defined, to identify all cases where the code deviated from this
defined style. This information is presented in an ordered list. An example of
an error-list for the \textc{Launcher project} can be seen below in
\autoref{CodeStyleErrors}.

\figx[0.85]{CodeStyleErrors}{List of code style deviations in the Launcher
project.}

In order to solve the problems with the unsystematic code style, we have
methodically resolved the issues identified by CheckStyle.

\subsection{T636 - Weekplanner: Identify Functional Requirements}
Placeholder\ldots'


\subsection{T636 - Disable Problematic Visual Elements}
Placeholder\ldots

\subsection{T646 - Identify Apps Required Resources}
Placeholder\ldots

\subsection{T647 - Document Apps' Database Requirements}
Placeholder\ldots

\subsection{T660 - Decrease Pictogram Search-Time}
Placeholder\ldots

\subsection{T670 - Fix Package Manager Access Crash}
Placeholder\ldots


% How did we identify the relevant tasks?
% Our focus area.
% Our Tasks.
% -----
% Solution of tasks

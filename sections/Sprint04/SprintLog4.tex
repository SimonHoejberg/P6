\chapter{Sprint: 4}

Following sprint 3 where we once again focused on adding stability to the
various GIRAF libraries, the focus for this sprint is to allow the next students
to have an easier and more efficient start than our own. Each year the various
libraries increase in size and complexity, and it becomes increasingly
important for new students to understand the code quickly. As such, the goal
for this sprint is to:

\say{The hand over to next years students allows understand the project
quickly.}\nl

For us this means that we will need for REST to be implemented and working
in the various libraries, as well as document the changes and additions.


\section{Tasks and Solutions}
Based on this goal we have been assigned the following tasks, see
\autoref{Tasks3}, which include the verification of our various solutions from
earlier sprints. These verification tasks have been marked with the tag
``Verify''.

\begin{table}[H]
\begin{centering}
\begin{tabular}{|l|p{9cm}|l|}
\hline
Number 	& Description & Man-hours \\ \hline
T772    & Verify: Security problem when changing system date/time & ??\\\hline
T785	& Verify: Handle crash such that the user does not get an error message and
is brought back to the launcher & 6\\ \hline
T822    & Verify: implementation of Grayscale & 6\\\hline
T830    & Never show the bug screen & 2 \\\hline
T836	& Remove old libraries from artifactory, and begin implementing REST
snapshot & ??\\\hline
T837	& Rework buttons to use methods called from XML & ??\\\hline
T859   	& Crosschecking wikis & ??\\\hline
T862  	& Guide: General development & ?? \\ \hline
T863	& Guide: Sources for starting out & ??\\ \hline
T864	& Guide: Android compatibility java8, sync and build & ??\\
\hline 
T865	& G609, G610: Update the Google store guide & 2\\\hline
T868    & REST: Fix Launcher app & 70 \\\hline
T870    & REST: Fix component-lib & 60\\\hline
T872 	& Documentation: Launcher & ??\\ \hline
T875 	& Launcher: Add functionality to Log In screen & 20\\\hline 
T878 	& Everyone: Fix bugs & ??\\ \hline
T889	& Launcher: Add an Institute to Guardian login & 12 \\\hline
\end{tabular}
\caption{Tasks for the third sprint}
\label{Tasks3}
\end{centering}
\end{table}

In the following subsections we will discuss the tasks and document our
solutions.

\subsection{T836: Remove old libraries from artifactory, and begin implementing REST
snapshot}
We have not removed the old libraries from artifactory, because some of the
older apps might still need them in the transition to the new rest api. But they
are no longer used as a deprendecy in the launcher or giraf component lib and we
have implementet not just the rest snapshot but the full rest release which can
be seen how we use it in the rest fix launcher.

\subsection{T868: REST: Fix Launcher app}
Implementing REST --> We can't make a request handler due to not being able
to use lambda expression since we don't use java 8.
This means we have to Responses within responses since we need some error
handling, although later in the sprint a sort of solution to the problem have
been made, but this was so late that all places where we uses rest we already
have implementen it using the other but still supported way of doing requests.
We have also remove all the usage of profiels and switched them out for users
from the new rest model libray.

code show an example of how we use reqeuest
\fix{}{Write me}


\subsection{T870 - REST: Fix component-lib}
With the introduction of the \textc{rest-models} library, and the removal of the
\textc{db-lib} library, the previously implemented database communication has
been rendered invalid. As such, it needs to be replaced by the equivalent
\textc{rest-models} approach. As this change spans a large part of the 58
classes in the \textc{component-lib} library, the presentation of this task will
focus on the class \textc{GirafPictogramItemView}, as it is one of the
classes which have recieved the most changes.\nl

The \textc{GirafPictogramItemView} class is responsible for presenting ItemViews
on the GUI (graphical use interface). In the presented implementation, both
Users and Pictograms extended the ItemView class, as both of those classes
contained pictures or icons which should be presented on the GUI. As such, The
\textc{GirafPictogramItemView} class was used to present both of these
elements.\nl

In the rest-models approach, many of the previous classes which were part of
the db-lib library have been deprecated, including the ItemView class. As such,
we have had to split the \textc{GirafPictogramItemView} class into two seperate
classes, which handles pictograms and users respectively. The class for
pictogram retained the old name, while the new class for users was named
\textc{GirafUserItemView}.\nl

Changes to the implementation of server requests can be seen in
\autoref{UsingRest} where the new REST approach is presented and discussed.
\fix{}{Write some more}


\subsection{T875 Launcher: Add functionality to Log In screen}

This task is centered around adding functionality to the login screen we made
in \autoref{LoginXML}, as REST is now ready to be implemented. The GUI is
implemented in the \textc{LoginActivity} class, where we create the reference
to XML layout and construct the needed listeners, this can be seen in
\autoref{onCreate} and \autoref{onLoginButtonClick}. The \textc{LoginController}
class handles the login method which consists of two nested requests to the
server.\nl

In \textc{LoginActivity} we use the \textc{onCreate} method, see
\autoref{onCreate}, to instantiate the \textc{LoginController} and set the
screen as \textc{R.layout.login\_activity}, which corresponds to the XML
file for the login screen. We then create references to the various UI elements,
such as the textboxes on \textbf{lines 7-8}. On \textbf{line 11-22} we create a
\textc{setOnKeyListener} for when the Enter key is pressed, which calls the
\textc{onLoginButtonClick} method.\nl

\begin{minipage}[H]{\linewidth}
\begin{lstlisting}[caption = Creating the correct references when logging in,
label = onCreate] public class LoginActivity extends GirafActivity {
	...
	public void onCreate(Bundle savedInstanceState) {
		controller = new LoginController(this);
    	setContentView(R.layout.login_activity);
		...
        usernameTextBox = (TextView) findViewById(R.id.username_textbox);
        passwordTextBox = (TextView) findViewById(R.id.password_textbox);
        
        //Tries to login when Enter is pressed
        passwordTextBox.setOnKeyListener(new View.OnKeyListener() {
            @Override
            public boolean onKey(View view, int keyCode, KeyEvent keyEvent) {
                if(keyEvent.getAction() == KeyEvent.ACTION_DOWN){
                    if(keyCode == KeyEvent.KEYCODE_ENTER){
                        onLoginButtonClick(view);
                        return true;
                    }
                }
                return false;
            }
        });
        ...
}
\end{lstlisting}
\end{minipage}

The \textc{onLoginButtonClick}, see \autoref{onLoginButtonClick}, method is used
to disable the elements the user can interact with, start the load animation and
pass the username and password to \textc{controller} via the \textc{login}
method.\nl

\begin{minipage}[H]{\linewidth}
\begin{lstlisting}[caption = Passes the user information and begins the waiting
animation, label = onLoginButtonClick] public class LoginActivity extends
GirafActivity { ...
	public void onLoginButtonClick(View view) {
        loginButton.setEnabled(false);
 		...
        findViewById(R.id.girafHeaderIcon).startAnimation(loadAnimation);
        controller.login(username, password);
    }
	...
\end{lstlisting}
\end{minipage}

In the \textc{LoginController} class we use the \textc{login} method to
communicate with the server through a couple different requests, which are
detailed more in \autoref{UsingRest}.\\

In this code example, \autoref{loginReq}, we Instantiate a \textc{RequestQueue}
on \textbf{line 4} along with a \textc{LoginRequest} on \textbf{lines 7-14}
which we add to the queue on \textbf{line 15}. When the server responds to our
login request can then either be granted or denied access. If we are denied
access the \textc{Response.ErrorListener} on \textbf{lines 11-13} uses an
\textc{onErrorResponse} method to displays a message depending on the type of
error it has received, this is further illustrated in
\autoref{CSLoginRequest3}.\nl

\begin{minipage}[H]{\linewidth}
\begin{lstlisting}[caption = Verifing the user information through a login
request, label = loginReq] public class LoginController {
	...
    public void login(final String username, String password) {
        RequestQueue queue = RequestQueueHandler.getInstance(gui.getApplicationContext()).getRequestQueue();

        //Creates a login request which is then added to the queue later
        LoginRequest loginRequest = new LoginRequest(username, password,
        	new Response.Listener<Integer>() {
        		...
        	},
        	new Response.ErrorListener() {
        		...
        	}
   		);
		queue.add(loginRequest);
	}
}
\end{lstlisting}
\end{minipage}

When a login request is validated we proceed to the \textc{Response.Listener} on
\textbf{lines 8-10} of \autoref{loginReq}, this code is described in
\autoref{userReq}.\\
In the OnResponse listener on \textbf{lines 3-20} we make a \textc{GetRequest}
on \textbf{6} for a user with the same username and password as before. This is
because the \textc{LoginRequest} does not return a user. Inside the
\textc{Response.Listener} on \textbf{line 10-15} we proceed to pass the user
to \textc{homeIntent} and start the next activity.\nl

\begin{minipage}[H]{\linewidth}
\begin{lstlisting}[caption = If the user information is authenticated we proceed
with a request for user, label = userReq] new Response.Listener<Integer>() {
	@Override
    public void onResponse(Integer statusCode) {
    	//Creates a GetRequest for the user, which is then added to the queue
    	//within the loginRequest 
    	GetRequest<User> userGetRequest = new GetRequest<User>(username, User.class, 
    	new Response.Listener<User>() {
        	//Passes the userinfo to homeIntent
            @Override
            public void onResponse(User response) {
            	Intent homeIntent = new Intent(gui, HomeActivity.class);
                homeIntent.putExtra(IntentConstants.CURRENT_USER,response);
                homeIntent.setFlags(Intent.FLAG_ACTIVITY_NEW_TASK);
                gui.startActivity(homeIntent);
            }
        }, new Response.ErrorListener() {
			...
        });
        queue.add(userGetRequest);
	}
}
\end{lstlisting}
\end{minipage}

%institute screen
% Ny implementation til institution --> login request --> user request hvor user
% nu er af typen institution --> user selection screen Vi genbruger en
% user_selection screen

%No time for institute

\subsection{Guides and Help and Help For Future Students}
In order to help the next batch of students, every group has been tasked with
creating a set of guides. These guides are meant to explain the various
parts of GIRAF that are most likely going to be confusing, such as starting out,
how GIRAF is developed and how the various tools work together. The
majority of these guides are written by other groups can be found on the GIRAF
wiki\citep{GWiki}. As such \textbf{T862: Guide for general development},
\textbf{T863: Guide for starting out} and \textbf{T864: Guide for android
compatibility, java8, sync and build} can be considered resolved.

\subsubsection{T859: Crosscheck wikis}
The purpose of this task is to ensure that wikis are of acceptable state, this
is done having
\fix{}{write me}

\subsubsection{T865: Update the Google Store guide}

The goal of this task is to update the guide on releasing builds to Google
Store. By reading the guide from 2015 and comparing the pictures in it with
Google Stores current design, we can see that it has changed quite a bit since
the guide was written. All the changes of importances were due to additional
tabs which moved various parts of the functionality. We then updated the
pictures and along with the relevant part of text to better reflect the new
design.
\fix{Consider if we need to add pics - it was very minor work after all}{}

\subsubsection{T872: Document the Launcher}
\fix{}{Write me a wiki page}



\subsection{Verification}

\subsubsection{T785 and T830: Handle crash such that the user does not get an
error message and is brought back to the launcher}
Thoughout testing the launcher after the REST API was included, we see that when
we crash the bug screen is no longer shown. We have tried implementing the
return to launcher but have not been able to resolve it due to time constraints.
As such this task can only be considered partially resolved.

\subsubsection{T822: Verify the implementation of Grayscale}
We have removed all \textc{SharedPreferences}, including the one we used in
\autoref{grayEx2}, as such we have no way to see the users setting. We have thus
changed it to better make use of REST by making a request to the server for the
users setting, see \autoref{NEWsetGrayScaleUser}. As seen on \textbf{lines
xx-xy} we make a simple \textc{GetRequest<User>} pass the settings and activity to the
\textc{setGrayScaleForActivity} method.\nl

\begin{minipage}[H]{\linewidth}
\begin{lstlisting}[caption = Finds the grayscale setting for each user, label =
NEWsetGrayScaleUser] 
public class GrayScaleHelper {
	...
	public static void setGrayScaleForActivityByUser(final Activity activity, 
	  final User user){
		...
		GetRequest<User> userGetRequest = new GetRequest<User>(user.getUsername(), User.class, new Response.Listener<User>() {
            @Override
            public void onResponse(User response) {
           		if(response.getSettings() != null) {
               		setGrayScaleForActivity(activity, response.getSettings().getUseGrayScale());
               	}
	 			...
}
\end{lstlisting}
\end{minipage}

In \textc{setGrayScaleForActivity} we experience a problem as we in
\autoref{grayEx1} tried to use \textc{setGray} for each view. This does not work
correctly, as such we use the \textc{root View} on \textbf{line
5}, as we are able to affect the other views through it, as can be seen on
\autoref{NEWsetGrayScale}.

\begin{minipage}[H]{\linewidth}
\begin{lstlisting}[caption = The root View is used instead of all the seperate
views, label = NEWsetGrayScale] public class GrayScaleHelper {
	...
    public static void setGrayScaleForActivity(Activity activity, boolean shouldBeGray) {
        //Get the root view
        View view = activity.getWindow().getDecorView().getRootView();
        setGray(view, shouldBeGray);
    }
    ...
}
\end{lstlisting}
\end{minipage}

There is a small problem as currently the \textc{dialogs} are not grayscaled
when the \textc{root View} is. We think this is because they are not attached to
it, see \autoref{GrayScaleColorDialog}, but due to time constraints we have
choosen to pass this particular part to next years students. 

\figx[0.2]{GrayScaleColorDialog}{An example of a colored dialog with a
gray scaled background.}

\subsubsection{T878: Fix bugs}
This is a task for every group. This sprint we have been busy implementing REST
in the Launcher and Component-lib, and as the deadline has come closer we are
under increasing time pressure to make it workable. As such we have chosen to
not directly work on this task and leave the minor bugs for next years students.
These bugs include:

\begin{itemize}
  \item Dialogs are in color while the rest of GIRAF is gray.
  \item It is possible to change users through settings.
\end{itemize}

There are likely more bugs which we have not discovered as \fix{}{naade vi at
blive 'faerdige' inden code freezer?}

\subsection{Unfinished and Invalid Tasks}\label{S4Invalid}
\begin{itemize}
  \item T772: Verify: Security problem when changing system date/time
  	\begin{itemize}
  		\item The orginal problem discoved during sprint 2 as seen in \autoref{T772}
 		was that the user could extend the login session by doing back in time on the
  		device.
		This is an issue because we do not check the time using some form of a time
		server, and therefor not able to detect this.  
		But now this task is invalid because the server now handles if a user is
		logged in, using tokens / cookies which the rest libary handes.
		\end{itemize}
  \item T837: Rework buttons to use methods called from XML
	\begin{itemize}
	  \item The purpose of this task is to increase the quelity of the code, and to
	  reduce unnecessary clutter. This can be done as specifying listeners in the
	  code itself is unnecessary, and can simply be done from the xml file itself.
	  \item We have managed to do this for a number of classes, but have yet to
	  do this for all classes in the Launcher. This approach can, and should, be
	  used in all xml files for GIRAF.
	  \end{itemize}
  \item T889: Add an Institute to Guardian login
   \begin{itemize}
     \item This task is an extension of \textbf{T875}, as it is intended for an
     institute-type user to be log in as a guardian. 
     This task is considered unfinished due to time constraints as there
     were several changes happening to REST while we were busy implementing it.
     It was not considered a vital task, and is as such left for the next years
     students.
   \end{itemize}
\end{itemize}
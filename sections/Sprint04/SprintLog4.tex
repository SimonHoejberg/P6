\chapter{Sprint: 4}

The focus for this sprint is to allow the next students
to have an easier and more efficient project start than our own. Each year the
various libraries increase in size and complexity, and it becomes increasingly
important for new students to understand the code quickly. As such, the goal
for this sprint is to:\nl

\say{Prepare the project to be handed over to next year's students, such that
they will be able to quickly understand the project and begin development.}\nl

For us, this means that we will need for \rlib\ to be implemented and
working in the \lapp\ and \clib, as well as document these changes and
additions, and shortcomings.

\section{Tasks and Solutions}
Based on this goal we have been assigned the following tasks, see
\autoref{Tasks4}, which include the assessment of our various solutions from
earlier sprints. These assessment tasks have been marked with the tag
``Assess''. Based on the workload associated with implementing the \rlib\ in the
\lapp\ and \clib, we have chosen to give group SW612 the responsibility for the \plib.

\begin{table}[H]
\begin{centering}
\begin{tabular}{|l|p{9cm}|l|}
\hline
Number 	& Description & Man-hours \\ \hline
T772    & Assess: Security problem when changing system date/time & 1\\\hline
T785	& Assess: Handle crash such that the user does not get an error message and
is brought back to the launcher & 6\\ \hline
T822    & Assess: implementation of Grayscale & 6\\\hline
T830    & Never show the bug screen & 2 \\\hline
T836	& Remove old libraries from artifactory, and begin implementing REST
snapshot & 3\\\hline
T837	& Rework buttons to use methods called from XML & 2\\\hline
T859   	& Crosschecking Wikis & 1\\\hline
T862  	& Guide: General development & ?? \\ \hline
T863	& Guide: Sources for starting out & ??\\ \hline
T864	& Guide: Android compatibility java8, sync and build & ??\\
\hline 
T865	& Update the Google Play store guide & 2\\\hline
T868    & REST: Fix Launcher app & 70 \\\hline
T870    & REST: Fix component-lib & 60\\\hline
T872 	& Documentation: Launcher & 2\\ \hline
T875 	& Launcher: Add functionality to Log In screen & 20\\\hline 
T878 	& Everyone: Fix bugs & ??\\ \hline
T889	& Launcher: Add an Institute to Guardian login & 12 \\\hline
\end{tabular}
\caption{Tasks for the fourth sprint}
\label{Tasks4}
\end{centering}
\end{table}

In the following subsections we will discuss the tasks and document our
solutions.

\subsection{T836, T868, T870: Implement REST in Launcher and
Component-lib}\label{T836_T868_T870} 
With the finalization of the beta snapshot of the \rlib, we have been assigned
the task of implementing it into the \lapp\ and the \clib\ library. This process
consists of removing the old \textc{DB-Lib} and replacing all of its
functionality with the equivalent in \rlib.\nl

While the database \ttt{DB-Lib} is using has been deprecated, we have
chosen not to remove the old libraries from Artifactory, as some of the older
applications, are still dependent on them, and we want next year's student to
be able to look at the library to determine what functionality they need to
replace in favor of \rlib. The exact implementation of \rlib\ is discussed in
\autoref{UsingRest}.\nl

Implementing \rlib\ consists of two phases. The first phase is to replace
database requests with the REST equivalent. The second phase is to replace
the helper and controller classes which have been implemented by \textc{DB-Lib}.
Additionally, we discuss the changes made to the \textc{GirafPictogramItemView}
class of the \rlib\ library, as switching to REST has required us to split the
class into two individual ones.

\subsubsection{Replacing Database Requests}
Originally the \lapp\ only requested data from the server whenever a user
logged in. When using the REST framework, we no longer use
\ttt{SharedPreferences} but store all data in the database. In addition, we make
a request to the database whenever we need user information. This is done to
make sure we always have the most up-to-date data. A problem with the current
implementation of \rlib, is that requests are always made asynchronously. This
requires the request to encapsulate all of the code which makes use of the
requested data, as the rest of the program will continue to execute while the
request is being made. If the code was not encapsulated in the request, it would
cause \ttt{nullPointerExceptions}, as the code attempts to access an object
which is not retrieved yet. The design decisions behind the implementation of
\rlib\ are discussed in \autoref{UsingRest}, where we also present a database
request.

\subsubsection{Replacing Helpers and Controllers}
As we want to replace all functionality which originates in \textc{DB-Lib}, we
start by removing all imports from this library. Since most of \textc{DB-Lib}'s
functionality is contained in a number of helper classes, we remove these and
fix the errors caused by their abscence. An example of a helper class could be
the \textc{ProfileApplicationController} in the \textc{GirafFragment} class of
the \lapp\ application.
This is shown in \autoref{pac}.\nl

\begin{minipage}[H]{\linewidth}
\begin{lstlisting}[caption = Declaration of the
\ttt{ProfileApplicationController} which is used to retrieve information about \ttt{ProfileApplications}, label = pac] 
@Override 
void setListeners() { super.listener = new View.OnClickListener() {
 		@Override 
 		public void onClick(View v) {
			...
    		final ProfileApplicationController pac = new ProfileApplicationController(getActivity()); 
    		...
    	}
	};
}
\end{lstlisting}
\end{minipage}

Following the declaration and instantiation of the
\textc{ProfileApplicationController} under the identifier \textc{pac}, the
helper is used in the \textc{userHasApplicationInView} method. In this case,
\textc{pac} is used to retrieve a \textc{ProfileApplication}, which is a class
which maps users to applications, such that it can identify if a user is allowed
to access that application. If the user can't access the given application, the
returned value is \ttt{null}. The effect of this can be seen in
\autoref{PACheck}, where the returned value is a boolean which is based on if \textc{thisPA} is
\ttt{null}.\nl

\begin{minipage}[H]{\linewidth}
\begin{lstlisting}[caption = Method which checks if a user is allowed to access an application, label = PACheck] 
private boolean userHasApplicationInView(ProfileApplicationController pac, Application app, Profile user) 
{ 
	ProfileApplication thisPA = pac.getProfileApplicationByProfileIdAndApplicationId(app, user);

    return thisPA != null;
}
\end{lstlisting}
\end{minipage}

Given that all of the presented functionality is based on the \textc{DB-Lib}
library, we will have to replace it with the REST equivalent. In \rmlib,
\textc{Profiles} have been renamed to \textc{Users}, and has a settings object,
which contains information about that users settings. This object is stored in
the database and can be retrieved through a request. The process of retrieving
objects from the database is presented in \autoref{UsingRest}.
In the case of checking if a user can access an application, we have replaced
the old method in \autoref{PACheck} with the method named
\textc{userCanAccessApp}, which can be seen in \autoref{ucaa}. In this method,
we can simply access the user's settings object, and check if the given
application appears on the user's list of applications.\nl

\begin{minipage}[H]{\linewidth}
\begin{lstlisting}[caption = New method which replaces the functionality of the
\ttt{ProfileApplicationController}, label = ucaa] 
public class GirafFragment extends AppContainerFragment {
...
	private boolean userCanAccesApp(Application app, User user) 
	{
		return user.getSettings().getAppsUserCanAccess().contains(app);
	}
...
}
\end{lstlisting}
\end{minipage}

\subsubsection{Replacing ItemViews}
The \textc{GirafPictogramItemView} class is responsible for presenting
\textc{ItemViews} on the GUI. In the previous implementation, both \ttt{Users}
and \ttt{Pictograms} extended the \ttt{ItemView} class, as both of those classes
contained pictures or icons which should be presented on the GUI. As such, the
\textc{GirafPictogramItemView} class was used to present both of these
elements.\nl

In the \rmlib\ approach, many of the previous classes which were
part of \textc{DB-Lib} have been deprecated, including the approach to
using the \textc{ItemView} class. As such, we have had to split the
\textc{GirafPictogramItemView} class into two separate classes, which handles
\ttt{Pictograms} and \ttt{Users} respectively. The class for \ttt{Pictogram}
retained the old name, while the new class for \ttt{Users} was named
\textc{GirafUserItemView}. As both \ttt{Pictograms} and \ttt{Users} still have
fields for their icon or picture, the two \ttt{ItemView} classes have simply
been specialized to use the respective objects' methods to access this field.

\subsection{T875 Launcher: Add functionality to Log In screen}\label{T875}
The purpose of this task is to add functionality to the login screen we made
in \autoref{LoginXML}, as \rlib\ is now ready to be implemented. The GUI is
implemented in the \textc{LoginActivity} class, where we create the reference
to the \ttt{XML} layout and construct the needed listeners, this can be seen in
\autoref{onCreate} and \autoref{onLoginButtonClick}. The \textc{LoginController}
class handles the login method which consists of \textc{LoginRequest} with a 
\textc{GetRequest} nested inside it, this can be seen on \autoref{loginReq} and
\autoref{userReq}.\nl 

In \textc{LoginActivity} we use the \textc{onCreate} method, see
\autoref{onCreate}, to instantiate the \textc{LoginController} and set the
content to the layout by using the ID \\\textc{R.layout.login\_activity}, which
corresponds to the \ttt{XML} file for the login screen on \textbf{lines 4-5}. We
then create references to the various UI elements, such as the \ttt{TextBoxes}
on \textbf{lines 7-8}. On \textbf{line 12-23} we create a
\textc{setOnKeyListener} for when the Enter key is pressed, which calls the
\textc{onLoginButtonClick} method.\nl

\begin{minipage}[H]{\linewidth}
\begin{lstlisting}[caption = \ttt{OnCreate} method for the \ttt{LoginActivity}
class, label = onCreate] 
public class LoginActivity extends GirafActivity {
	...
	public void onCreate(Bundle savedInstanceState) {
		controller = new LoginController(this);
    	setContentView(R.layout.login_activity);
		...
        usernameTextBox = (TextView) findViewById(R.id.username_textbox);
        passwordTextBox = (TextView) findViewById(R.id.password_textbox);
        
        //Tries to login when Enter is pressed
        passwordTextBox.setOnKeyListener(new View.OnKeyListener() {
            @Override
            public boolean onKey(View view, int keyCode, KeyEvent keyEvent) {
                if(keyEvent.getAction() == KeyEvent.ACTION_DOWN){
                    if(keyCode == KeyEvent.KEYCODE_ENTER){
                        onLoginButtonClick(view);
                        return true;
                    }
                }
                return false;
            }
        });
        ...
}
\end{lstlisting}
\end{minipage}

The \textc{onLoginButtonClick}, see \autoref{onLoginButtonClick}, method is used
to disable the elements the user can interact with, start the load animation and
pass the username and password to the \textc{controller} via the \textc{login}
method.\nl

\begin{minipage}[H]{\linewidth}
\begin{lstlisting}[caption = Passes the user information and begins the waiting animation, label = onLoginButtonClick] 
public class LoginActivity extends GirafActivity 
{ 
...
	public void onLoginButtonClick(View view) {
        loginButton.setEnabled(false);
 		...
        findViewById(R.id.girafHeaderIcon).startAnimation(loadAnimation);
        controller.login(username, password);
    }
	...
}
\end{lstlisting}
\end{minipage}

In the \textc{LoginController} class we use the \textc{login} method to
communicate with the server through a request chain, which is detailed more in
\autoref{UsingRest}.\nl

\begin{minipage}[H]{\linewidth}
\begin{lstlisting}[caption = Verifing the user information through a login
request, label = loginReq] 
public class LoginController {
	...
    public void login(final String username, String password) {
        RequestQueue queue = RequestQueueHandler.getInstance(gui.getApplicationContext()).getRequestQueue();

        //Creates a login request which is then added to the queue later
        LoginRequest loginRequest = new LoginRequest(username, password,
        	new Response.Listener<Integer>() {
        		...
        	},
        	new Response.ErrorListener() {
        		...
        	}
   		);
		queue.add(loginRequest);
	}
}
\end{lstlisting}
\end{minipage}

In the \ttt{LoginContrller} class, see \autoref{loginReq}, we instantiate a
\textc{RequestQueue} on \textbf{line 4} along with a \textc{LoginRequest} on
\textbf{lines 7-14} which we add to the queue on \textbf{line 15}. When the
server responds to our login request we can then either be granted or denied
access. If we are denied access the \textc{Response.ErrorListener} on
\textbf{lines 11-13} uses an \textc{onErrorResponse} method to displays a
message depending on the type of error it has received, this is further
illustrated in \autoref{CSLoginRequest3}. When a login request is validated we
proceed to the \textc{Response.Listener} on \textbf{lines 8-10} of
\autoref{loginReq}, this code is described in \autoref{userReq}.\nl

\begin{minipage}[H]{\linewidth}
\begin{lstlisting}[caption = If the user information is authenticated we proceed
with a request for user, label = userReq] 
new Response.Listener<Integer>() {
	@Override
    public void onResponse(Integer statusCode) {
    	//Creates a GetRequest for the user, which is then added to the queue
    	//within the loginRequest 
    	GetRequest<User> userGetRequest = new GetRequest<User>(User.class, 
    	new Response.Listener<User>() {
        	//Passes the userinfo to homeIntent
            @Override
            public void onResponse(User response) {
            	Intent homeIntent = new Intent(gui, HomeActivity.class);
                homeIntent.putExtra(IntentConstants.CURRENT_USER,response);
                homeIntent.setFlags(Intent.FLAG_ACTIVITY_NEW_TASK);
                gui.startActivity(homeIntent);
            }
        }, new Response.ErrorListener() {
			...
        });
        queue.add(userGetRequest);
	}
}
\end{lstlisting}
\end{minipage}

In the \ttt{onResponse} listener on \textbf{lines 3-20} we make a
\textc{GetRequest} on \textbf{line 6} for the user. This is because the
\textc{LoginRequest} does not return a user. Inside the
\textc{Response.Listener} on \textbf{line 10-15} we proceed to pass the user to
\textc{homeIntent}'s \ttt{Extra} and start the next \ttt{Activity}.\nl

%institute screen
% Ny implementation til institution --> login request --> user request hvor user
% nu er af typen institution --> user selection screen Vi genbruger en
% user_selection screen

%No time for institute

\subsection{Guides and Help for Future Students}
In order to help the next batch of students, every group has been assigned the
task of creating a set of guides. These guides are meant to explain the various
parts of GIRAF that are most likely going to be confusing, such as starting out,
how GIRAF is developed and how the various tools work together. The majority of
these guides are written by other groups and can be found on the GIRAF Wiki
\citep{GWiki}. As such \textbf{T862: Guide for general development},
\textbf{T863: Guide for starting out} and \textbf{T864: Guide for android
compatibility, java8, sync and build} can be considered resolved.

\subsubsection{T859: Crosscheck Wikis}
The purpose of this task is to make sure that each group reads the Wiki pages of
another group. This is done in order to ensure that the contained information is
easy to understand. This task is considered resolved on our end, as we have
read the guides for Gradle and Jenkins, and found them easy to understand.
The page for the \wapp\ lacks a list of dependencies and the repository but
describes the UI well.

\subsubsection{T865: Update the Google Play Store guide}
The goal of this task is to update the guide on releasing builds to Google Play
Store. By reading the guide from 2015 and comparing the pictures with Google
Play Store's current design, we can see that it has changed quite a bit since
the guide was written. All of identified changes were due to additional
tabs which moved various parts of the store's functionality. As such, we updated
the pictures along with the relevant part of the text to better reflect the
new design.

\subsubsection{T872: Document the Launcher}
We have made a Wiki page for the \lapp\ which has been included in \LaTeX\
format in \autoref{WikiLauncher}. On this page, we describe the noteworthy
features which a developer need to consider when implementing \rlib, and some of
missing features and bugs which we were unable to fix. For each of these we have
written how we would have fixed them if we had enough time. We have also
included a short Wiki page to describe the important parts of \clib, see
\autoref{WikiCompLib}. This page is mainly comprised of a description of the
classes we have made, and we advise that it should be more or less recreated
from scratch.

\subsection{Assessment}
Throughout the prior sprints we have not been able to assess our solutions, as
such now that the Launcher is in a working state we can assess the appropriate
solutions.

\subsubsection{T785 and T830: Handle crash such that the user does not get an
error message, and is brought back to the launcher} 
Throughout assessing the \lapp\ after the \rlib\ was included, we noticed that
when we crash, the \textc{GirafBugSplashActivity}, see \autoref{GBugSplash}, is
no longer shown, which was a part of the task. We have tried implementing the
return to the \lapp\ as seen in \autoref{T785-T830}, but this did not work and a
new implementation was made. We have not been able to resolve this task due to
time constraints. As such this task can only be considered partially resolved.


\subsubsection{T822: Assess the implementation of Grayscale}
In order to make use of the REST framework, we changed the \ttt{GrayScaleHelper}
class to include a server request, which retrieves the \ttt{User}'s
\ttt{Settings} object, see \autoref{NEWsetGrayScaleUser}. As seen on
\textbf{line 6-10}, we make a simple \textc{GetRequest} for the \ttt{User} and
access its \ttt{Settings} object. From the \ttt{Settings} object, we access its
boolean for grayscale, as pass it together with the current \ttt{Activity} to
the \textc{setGrayScaleForActivity} method.\nl

\begin{minipage}[H]{\linewidth}
\begin{lstlisting}[caption = Finds the grayscale setting for each user, label =
NEWsetGrayScaleUser] 
public class GrayScaleHelper {
	...
	public static void setGrayScaleForActivityByUser(final Activity activity, 
	  final User user){
		...
		GetRequest<User> userGetRequest = new GetRequest<User>(user.getUsername(), User.class, new Response.Listener<User>() {
            @Override
            public void onResponse(User response) {
           		if(response.getSettings() != null) {
               		setGrayScaleForActivity(activity, response.getSettings().getUseGrayScale());
               	}
	 			...
}
\end{lstlisting}
\end{minipage}

In the \textc{setGrayScaleForActivity} method we experience a problem, as we in
\autoref{grayEx1} tried to use \textc{setGray} for each view, which resulted in
a \ttt{nullPointerException}. Instead we use the \textc{root View} as seen on
\textbf{line 5} in \autoref{NEWsetGrayScale}. This works in same way as the
old code, but does not crash GIRAF applications.\nl

\begin{minipage}[H]{\linewidth}
\begin{lstlisting}[caption = The root View is used instead of all the seperate
views, label = NEWsetGrayScale] 
public class GrayScaleHelper {
	...
    public static void setGrayScaleForActivity(Activity activity, boolean shouldBeGray) {
        //Get the root view
        View view = activity.getWindow().getDecorView().getRootView();
        setGray(view, shouldBeGray);
    }
    ...
}
\end{lstlisting}
\end{minipage}

There is currently a small problem, as the \textc{dialogs} are not grayscaled
when the \textc{root View} is. We think this is because they are not attached to
it, see \autoref{GrayScaleColorDialog}, but due to time constraints, we have
chosen to pass this particular part on to next year's students.

\figx[0.2]{GrayScaleColorDialog}{An example of a colored dialog with a
gray scaled background.}

\subsubsection{T878: Fix bugs}\label{T878}
This is a task for every group. This sprint we have been busy implementing\\
\rlib\ in the \lapp\ and \clib, and as the deadline has come closer we are under
increasing time pressure to make it workable. As such we have chosen to not
directly work on this task and leave the minor bugs for next year's students.
These bugs and missing features include:

\begin{itemize}
  \item \ttt{Dialogs} are in color while the rest of GIRAF is grayscale.
  \item It is possible to change users through settings.
  \item When changing settings for a citizen, it is not possible
  to return to the guardian used to open the settings page. This is because 
  the citizen is now logged in.
  \item \ttt{SharedPreferences} should be used for the Login screen to determine
  if it needs to be in grayscale.
  \item \ttt{Settings} are not saved when a \ttt{ResourceRequest} is completed
  and as such grayscale does not work outside \textc{SettingsActivity}.
  \item The apps the user can access is never sent to the server, because of a
  \ttt{StackOverflow} in \rlib, when that is fixed, fix the out
  commented code in the respective classes for that issue.
  \item A better icon should be created for missing apps on the device. Namely
  an icon which better communicates that when clicked on, it redirects the user
  to Google Play Store.
  \item Identify a better approach to finding the icon of an app. In the current
  approach we iterate though every installed application when
  searching for an icon for a specific app.
\end{itemize}

The entirety of documented bugs and missing features can be found in
\autoref{fwork}, where we discuss what changes should be made to the
\lapp\ and \clib\ library by next year's students.


\subsection{Unfinished and Invalid Tasks}\label{S4Invalid}
\begin{itemize}
  \item T772: Assess: Security problem when changing system date/time
      \begin{itemize}
          \item The original problem discovered during sprint 2 as seen in \autoref{T772}
         was that the user could extend the login session by doing back in time on the
          device.
        \item This is an issue because we do not check the time using some form of a
        time server, and therefor not able to detect this.  
        But now this task is invalid because the server now handles if a user is
        logged in, using tokens/cookies which \rlib\ handles.
        \end{itemize}
  \item T837: Rework buttons to use methods called from XML
    \begin{itemize}
      \item The purpose of this task is to increase the quality of the code and to
      reduce unnecessary clutter. This can be reduced by creating methods which
      are called from the \ttt{onClick} property of button's in the \ttt{XML}
      file, instead of specifying listernes in the code. 
      \item We have managed to
      do this for a number of classes, but have yet to do this for all classes
      in the \lapp. This approach can, and should, be used in all \ttt{XML}
      files for GIRAF.
      \end{itemize}
  \item T889: Add an Institute to Guardian login \label{T889}
   \begin{itemize}
     \item The purpose of this task is to add a new approach to logging in,
     which should be used by institutions. When logging in to a \ttt{User} which
     has the role ``Department'', the GUI should present a \ttt{ProfileSelector}
     which allows the user to choose between all guardians in that institution.
     When selecting a guardian, the system should make a \ttt{LoginRequest} as
     that guardian.
     \item This task is considered unfinished due to time constraints as there
     were several changes happening to the REST framework while we were busy
     implementing it. As it was not considered a vital task, it has been
     left for the next year's students.
   \end{itemize} 
\end{itemize}

\section{Retrospective}
During this sprint, there were several small disruptions related to the REST
framework. Due to the applications needing additional functionality, the server
was redeployed several times without much warning. This also resulted in changes to
\rmlib\ and \rlib, which meant that work progressed slowly. There was a lot of
communication between the server and application groups, as there was an
increasing need to have a common understanding of how the applications used the
model. This is considered good, as it was easy to contact them and ask them for
help regarding the server problems we experienced.\nl

It proved a good idea to pass \textc{PictoSearch-Lib} to group SW612. While we
managed to leave the \textc{Launcher} and \textc{Giraf-Component-Lib} working,
with only a few missing features, having another sizeable library like
\textc{PictoSearch-Lib} would have left us unable to finish work in time.\nl

At the sprint meeting, none of the groups were satisfied with the final result
of the sprint, as most of the work done this semester was close to being stable
and working as intended but missing some features.\nl

\subsection{Evaluation}

This sprint has been the most successful of the sprints, the tools worked, the
REST framework was operational, and there was a lot of problem-solving between the
groups.
This is reflected in the fact that we managed to resolve 88.1\% of time units
allocated to the various tasks, see \autoref{Sprint4_piechart2}. The remaining
unsolved tasks are primarily relating to missing featues and have been left for
the next year's students.

\figx[1]{Sprint4_piechart2}{The percentage of total time for the tasks which
have been resolved}

Most of this sprint's tasks were finished towards the end of the sprint
when development of the REST framework stopped. This is reflected in the
burndown chart, see \autoref{Sprint4_burndownchart}, as many of the major tasks
with a lot of time allocated to them were only resolved on the last day.

\figx[0.4]{Sprint4_burndownchart}{The burndown chart for sprint 4}


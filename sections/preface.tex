\chapter*{Preface\markboth{Preface}{Preface}}
\addcontentsline{toc}{chapter}{Preface}
\vspace{\baselineskip}\hfill Aalborg University, \today
\vfill\noindent
\begin{minipage}[b]{0.45\textwidth}
 \centering
 \rule{\textwidth}{0.5pt}
   Jonas Ibrahim\\ 
 {\footnotesize <jibrah14@student.aau.dk>}
\end{minipage}
\hfill
\begin{minipage}[b]{0.45\textwidth}
 \centering
 \rule{\textwidth}{0.5pt}
   Jonathan Magnussen\\
 {\footnotesize <jnma14@student.aau.dk>}
\end{minipage}
\vspace{7\baselineskip}
\begin{center}
\noindent\hspace{0.26\textwidth}\begin{minipage}[b]{0.45\textwidth} 
 \centering
 \rule{\textwidth}{0.5pt}
 Christoffer Mouritzen\\
 {\footnotesize <cmouri14@student.aau.dk>}
\end{minipage}
\hfill
\vspace{3\baselineskip}
\end{center}

%From here on we have the written preface and the reading guide!!! 
\newpage 
\section{Reading Guide}
This section provides an overview of which order the different parts of the
report should be read and provides a list of the various terms used throughout
the report.

\subsection{Style List}
The following font styles are used throughout the report, and has the following meaning:

\begin{table}[H]
\centering

\begin{tabular}{|l|p{6cm}|}
\hline
\textbf{Style} & \textbf{Meaning} \\ \hline
\textc{Code} &  Code related elements e.g. libraries, methods, and classes.\\ \hline
\textbf{TXXX} & Tasks Reference \\\hline
\textbf{line x-y} & Line reference in code \\\hline
\end{tabular}
\caption{Notable font styles}
\end{table}


\subsection{Word List}
The following words are used throughout the report and has the following
meaning:

\begin{table}[H]
\centering

\begin{tabular}{|l|p{6cm}|}
\hline
\textbf{Word} & \textbf{Meaning} \\ \hline
User & A general user of the GIRAF system \\ \hline
Citizen & A person with Autism being taken care of \\ \hline
Guardian & A citizen's parent, legally assigned caretaker or someone working
at the Centre for Autism\\ \hline
REST &  REST stands for Representational State Transfer and for us is used to
reference the framework which is used to request information from the database\\\hline
REST Framework & This is used as a shorthand to refer to the REST server and
client.\\\hline
\end{tabular}
\caption{Notable words}
\end{table}

\subsection{General Reading Guide}
When reading the report both sprint 1 and 2 should be read lightly, as most of
the developed solutions have been theoretical and have been changed because of
the switch to \textc{REST} in sprint 3 and 4. During the reading of the second
sprint, before \autoref{DesignLogin} the reader could beneficial read the second
collaboration in \autoref{sec:Colab2} to better understand how the new login screen and its requiments
came to be. Then the rest of sprint 2 should be read. Before reading sprint 3, you should read
\autoref{S1CS} to get an understanding of what \textc{REST} is in
the context of application development. Then the reader should continue with
reading sprint 3 and 4 in depth, most of the development in sprint 3 is still
relevant.

\subsection{Reading Guide for Future Students}
The future students shoul read the collaboration in \autoref{S1CS} in order to
understand how the \textc{REST-Api-Client-Lib} works, and the ideas behind its
design. As such, it should be read before reading sprint 3 and 4. The other
collaboration can be read lightly but is not necessary. They should then read
from \autoref{LoginXML} in sprint 3 to the end of sprint 4. The other sprints
can be read lightly, most of what has been developed in those sprints has
changed in the last two sprints and is therefore not releavant to maintain the
\textc{Launcher}.
\chapter*{Preface\markboth{Preface}{Preface}}
\addcontentsline{toc}{chapter}{Preface}
\vspace{\baselineskip}\hfill Aalborg University, \today
\vfill\noindent
\begin{minipage}[b]{0.45\textwidth}
 \centering
 \rule{\textwidth}{0.5pt}
  Simon Højberg\\
 {\footnotesize <shajbe14@student.aau.dk>}
\end{minipage}
\hfill
\begin{minipage}[b]{0.45\textwidth}
 \centering
 \rule{\textwidth}{0.5pt}
  Jonas Ibrahim\\ 
 {\footnotesize <jibrah14@student.aau.dk>}
\end{minipage}
\vspace{7\baselineskip}
\begin{center}
\begin{minipage}[b]{0.45\textwidth} 
 \centering
 \rule{\textwidth}{0.5pt}
  Jonathan Magnussen\\
 {\footnotesize <jnma14@student.aau.dk>}
\end{minipage}
\hfill
\vspace{3\baselineskip}
\begin{minipage}[b]{0.45\textwidth}
 \centering
 \rule{\textwidth}{0.5pt}
  Christoffer Mouritzen\\
 {\footnotesize <cmouri14@student.aau.dk>}
\end{minipage}
\end{center}

%From here on we have the written preface and the reading guide!!! 
\newpage 
\section{Reading Guide}
This section provides an overview of the different parts of the report and
provides a list of the various terms used throughout the report.

\subsection{Content}
The report is divided into the following parts: Introduction, Sprints,
Collaboration Tasks, \fix{\ldots}{ add the addition parts}.

\subsubsection{Introduction}

\subsubsection{Sprints}

\subsubsection{Collaboration Tasks}

\subsection{Text Conventions}
\fix{
Throughout this report the Vancouver style of referencing has been used.}{See
next fix} References in the text are marked by arab numbers encased in square
brackets.
Additionally, references use a comma within the square brackets to separate the
source number and the definition of what part of the source text is being referenced.
An example of this could be ``This is a good example {[}6, ch. 3 p. 22-28{]}'',
where the source is the 6th entry in the bibliography, and the specific
referenced part is found in chapter 3 on pages 22 through 28. The sources listed
in the bibliography are presented in the order in which they are used. When
citing books, the author and the name of the book will be presented, and when
online sources are cited, the URL, title and date of access will be
presented.\nl

Throughout this report, code examples are used to present the implemented
functionality of the developed compiler. It should be noted, that the line
numbers used in the examples do not represent the actual line numbers as written in the
code files themselves. An example of this can be found in
\autoref{PrefaceExample}.\nl

\begin{minipage}[H]{\linewidth}
\begin{lstlisting}[caption = {Example code.}, label ={PrefaceExample} ]
if(true){
	...
	print("This is an example!")	//This is a comment
	...
}
\end{lstlisting} 
\end{minipage}

\fix{
In this report, the three dots used in the code examples represent additional
code, which has been deemed irrelevant to the example itself. Furthermore,
when code examples are referenced in the text, they will assume the name of
\say{Listings}. Additionally, in the code examples, sentences beginning with
double forward slashes should be considered comments, and not part of the
presented code.}{Gotta ask Ying if it is ok to plagiarise your past self? The
entire subsection has been stolen from our P4 report}

\subsection{Word List}
The following words are used throughout the report and has the following
meaning.

\begin{table}[H]
\centering
\caption{My caption}
\label{my-label}
\begin{tabular}{|l|l|}
\hline
\textbf{Word} & \textbf{Meaning} \\ \hline
User & A general user of the Giraf system \\ \hline
Citizen & A person with Autism being taken care of \\ \hline
Guardian & A citizen's parent, legally assigned caretaker or someone working
at the Centre for Autism\\ \hline
\end{tabular}
\end{table}


\section{Reading Guide for Future Students}
\fix{\ldots}{Figure out what future students can use this report for}




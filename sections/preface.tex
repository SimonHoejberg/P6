\chapter*{Preface\markboth{Preface}{Preface}}
\addcontentsline{toc}{chapter}{Preface}
\vspace{\baselineskip}\hfill Aalborg University, \today
\vfill\noindent
\begin{minipage}[b]{0.45\textwidth}
 \centering
 \rule{\textwidth}{0.5pt}
   Jonas Ibrahim\\ 
 {\footnotesize <jibrah14@student.aau.dk>}
\end{minipage}
\hfill
\begin{minipage}[b]{0.45\textwidth}
 \centering
 \rule{\textwidth}{0.5pt}
   Jonathan Magnussen\\
 {\footnotesize <jnma14@student.aau.dk>}
\end{minipage}
\vspace{7\baselineskip}
\begin{center}
\noindent\hspace{0.26\textwidth}\begin{minipage}[b]{0.45\textwidth} 
 \centering
 \rule{\textwidth}{0.5pt}
 Christoffer Mouritzen\\
 {\footnotesize <cmouri14@student.aau.dk>}
\end{minipage}
\hfill
\vspace{3\baselineskip}
\end{center}

%From here on we have the written preface and the reading guide!!! 
\newpage 
\section{Reading Guide}
This section provides an overview of the different parts of the report and
provides a list of the various terms used throughout the report.

\subsection{Content}
The report is divided into the following parts: Introduction, Sprints,
Collaboration Tasks, \fix{\ldots}{ add the addition parts}.

\subsubsection{Introduction}

\subsubsection{Sprints}
This part covers sprint 1-4


\subsubsection{Collaboration Tasks}


\subsection{Word List}
The following words are used throughout the report and has the following
meaning:

\begin{table}[H]
\centering

\begin{tabular}{|l|p{6cm}|}
\hline
\textbf{Word} & \textbf{Meaning} \\ \hline
User & A general user of the GIRAF system \\ \hline
Citizen & A person with Autism being taken care of \\ \hline
Guardian & A citizen's parent, legally assigned caretaker or someone working
at the Centre for Autism\\ \hline
\end{tabular}
\caption{The notable words of GIRAF}
\end{table}


\section{Reading Guide for Future Students}
\fix{\ldots}{Figure out what future students can use this report for}




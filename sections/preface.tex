\chapter*{Preface\markboth{Preface}{Preface}}
\addcontentsline{toc}{chapter}{Preface}
\vspace{\baselineskip}\hfill Aalborg University, \today
\vfill\noindent
\begin{minipage}[b]{0.45\textwidth}
 \centering
 \rule{\textwidth}{0.5pt}
   Jonas Ibrahim\\ 
 {\footnotesize <jibrah14@student.aau.dk>}
\end{minipage}
\hfill
\begin{minipage}[b]{0.45\textwidth}
 \centering
 \rule{\textwidth}{0.5pt}
   Jonathan Magnussen\\
 {\footnotesize <jnma14@student.aau.dk>}
\end{minipage}
\vspace{7\baselineskip}
\begin{center}
\noindent\hspace{0.26\textwidth}\begin{minipage}[b]{0.45\textwidth} 
 \centering
 \rule{\textwidth}{0.5pt}
 Christoffer Mouritzen\\
 {\footnotesize <cmouri14@student.aau.dk>}
\end{minipage}
\hfill
\vspace{3\baselineskip}
\end{center}

%From here on we have the written preface and the reading guide!!! 
\newpage 
\section{Reading Guide}
This section provides an overview of which order the different parts of the
report should be read and provides a list of the various terms used throughout
the report.

\subsection{Word List}
The following words are used throughout the report and has the following
meaning:

\begin{table}[H]
\centering

\begin{tabular}{|l|p{6cm}|}
\hline
\textbf{Word} & \textbf{Meaning} \\ \hline
User & A general user of the GIRAF system \\ \hline
Citizen & A person with Autism being taken care of \\ \hline
Guardian & A citizen's parent, legally assigned caretaker or someone working
at the Centre for Autism\\ \hline
REST & Is the framework which is used to request information from the database\\
\hline
\end{tabular}
\caption{The notable words of GIRAF}
\end{table}

\subsection{General Reading Guide}
When reading the report both sprint 1 and 2 should be read lightly most of the
developed solutions have been theroecatical or is invalid because of the
switch to \textc{REST} in the later sprints. During the reading of the second
sprint, before \autoref{DesignLogin} the reader could beneficial read the second
collaboration to better understand how the new login screen and its requiments
came to be. Then the rest of sprint 2 should be read. Then before reading sprint
3, the first collaboration \autoref{S1CS} should be read to get an understanding
of what REST is in the context of application development and how the REST client library is structured. 
Then the reader should continue with reading sprint 3, 4 and the rest of the
report more in depth, most of the development in sprint 3 is still relevant.

\subsection{Reading Guide for Future Students}
The future students should read from \autoref{LoginXML} in sprint 3 to the end
of sprint 4. The other sprint can be read lightly but most of what has been
developed in thoose sprints that has changed in the last two sprint and is
therefor not releavant to maintain the launcher. The collaboration
\autoref{S1CS} should be read to understand how our REST library should work and
the thoughts behind it and should be read before reading sprint 3 and 4. The
other collaboration can be read lightly but is not really nessaray.



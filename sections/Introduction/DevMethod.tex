\section{Development Method}
In this section we describe the development method in the GIRAF multi-project.
First we describe 


% this section/chapter should be used to describe the project in a multi-project
% setting
%This includes an analysis of organizational context


\section{Multi-Project Organization Method/Scrum}
\fix{}{Might need a bit of editing depending on how the scrum of scrums
progresses} This semester we have organized the multi-project as a scrum of
scrums. One group serves as the scrum master while the remaining 6 groups each send 1
member. These 10 people meet once every week to discuss the multi-project
and after each sprint decide which tasks from the product backlog are suitable
for the sprint backlog. They then further divide the sprint backlog among the
groups such that each group have a suitable number of tasks for each sprint.\nl



% The backlog is thus divided into 3 different types:
% \begin{itemize}
%   \item \textbf{The product backlog} consist of all the tasks found throughout
%         the prior semesters, as well as the newly found tasks which have been
%         identified since we started working on the multi-project.
%   \item \textbf{The sprint backlog} is the collection of all the tasks which are
%         meant to be completed this sprint.
%   \item \textbf{The group backlog} is each groups subset of the sprint backlog.
% \end{itemize}





scrum group looks at alot of the discussed tasks

(monthly giant scrum find tasks and provide estimates by discussion - usikker)

present tasks and who should take them

termer

\section{Role Distribution}

PO Group



\section{Tools}

Phabricator

Discord

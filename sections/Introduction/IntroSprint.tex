\chapter{GIRAF: Multi-Project}

In this project we focus on improving the stability of the GIRAF project. This
chapter is used to explain what GIRAF is, what the goal of this semester is
and how the whole project is organized. the initial phase of the project and what the
different group will have as their primary focus. Furthermore, we will touch
upon how the different groups will work together and which tools are used.


% this section/chapter should be used to describe the project in a multi-project
% setting
%This includes an analysis of organizational context
\section{What is GIRAF}
\fix{}{Verify application names}
GIRAF is tablet environment for android. It is aimed at providing citizens and
their guardians with an easy way to communicate, by providing daily schedules,
games and education. It is a project which has been worked on over multiple
years as part the 6'th semesters bachelor project for Software. At the end of
the semester, the project is left for next batch of students to continue working
on. GIRAF consists of many different applications, as can be seen in
\autoref{GApps}.

\begin{table}[H]
\centering
\begin{tabular}{|p{3cm}|p{9cm}|}
\hline
Application			& Description \\\hline
Launcher  			& The main application of GIRAF, controls access to other
GIRAF applications. \\\hline 
Sequence	       	& A tool to create a sequence of pictograms which can
communicate a process.\\\hline 
PictoCreator  	   	& Allows the user to manually create pictograms which can be
saved on the database.\\\hline
PictoReader	   		& A tool which can read the text of an associated pictrogram
aloud.\\\hline
LifeStories	   		& A tool which allows a citizen to describe their day using
pictograms.\\\hline 
Timer     	   		& Allows guardians to create create a timer which limits how
long a citizen can use an application.\\\hline 
Weekplanner 		& Allows for the creation of schedules for citizens
using pictograms. \\\hline 
Category Tool		& A tool which allows pictograms to be grouped together as a
specific category. \\\hline 
Category Game   	& A game which teaches citizens to associate different object
by a common feature. \\\hline 
Voice Game  		& A game which allows citizens to control an object using their
voice volume. \\\hline
\end{tabular}
\label{GApps} 
\caption{The applications available to GIRAF}
\end{table}



The previous semester ended with the Voice Game application working as a
standalone in an acceptable state. The other applications are in working state,
but have several issues which prohibit their use by the customer.




\section{Development Method}
This semester the multi-project is a collaboration between 7
different groups. Through a meeting with one of the customers it is
decided that the goal of the project is to make a minimal viable product. The
focus this semester is two part: 
\begin{enumerate}
  \item We need to get the weekplanner into a acceptable state, where it can be
  used reliably by the citizens with no unexpected errors occuring. It also
  needs to have the necessary functionality to satisfy the customer.
  \item We need the server to be reliable and store the data safely, this is
  considered necessary since it would otherwise be to use GIRAF in any sizable
  institution.
\end{enumerate}

The groups are divided among the goals according to the desires of each
group. This is meant to increase the quality of the work since each group is
able to specialize on a specific part. It also allows the groups to work easier
together since most of them share some common feature. These two goals lead to
the following distribution of labour:

\begin{table}[H]
\centering
\begin{tabular}{|p{1cm}|p{3cm}|p{8cm}|}
\hline
Group & Focus & Description \\ \hline
sw609 & Weekplanner depedencies & Works on the providing stability to
weekplanner and its dependencies.\\\hline 
sw610 & Weekplanner & Works on getting weekplanner into a working
state.\\
\hline sw611 & Scrum masters & Organizes the scrum meetings. \\\hline 
sw612 & Product owners & Creates and describes tasks, also acts as contact to
customers. \\\hline 
sw613 & REST client & \\ \hline

sw614 & Infrastructure & \\ \hline

sw615 & REST server & \\ \hline

\end{tabular}
\caption{The focus of each of the groups}
\label{GroupDivision}
\end{table}

%Mere specifict om vores gruppes identitet dette semester
Our focus this semester is as seen above in \autoref{GroupDivision} on
providing stability to the weekplanner app as well as its dependencies\ldots



\subsection{Organizational Method}

%scrum section
The groups are organized by scrum, one group serves as the scrum masters while
the remaining 6 groups each send 1 member. These 10 people meet once every week
to discuss status of multi-project and how each of the groups are progressing.


%sprint section
The multi-project is divided into 4 different sprints,
They also decide which tasks should be added to each of the groups backlog at
the start of each sprint.


First sprint 20/2 - 15/3 	(3 uger 2 dage)
Second sprint 15/3 - 7/4 	(3 uger 2 dage)
Third sprint 7/4 - 1/5		(3 uger 2 dage)
Fourth sprint 1/5 - 22/5 	(3 uger)
Delivery 29/5


the
multi-project and after each sprint decide which tasks from the product backlog are suitable
for the sprint backlog. They then further divide the sprint backlog among the
groups such that each group have a suitable number of tasks for each sprint.\nl


\section{Tools}

Phabricator

Git

GIRAF website

Discord

Android studio?

Code review?


\chapter{Introduction(initial sprint)}

In this project we focus on improving the stability of the GIRAF project. This
chapter will be used to explain the initial phase of the project and what the
different group will have as their primary focus. Furthermore, we will touch
upon how the different groups will work together and which tools are used.


\section{Development Method}
In this section we describe the development method in the GIRAF multi-project.
First we describe 



setting: minimal viable product
% this section/chapter should be used to describe the project in a multi-project
% setting
%This includes an analysis of organizational context



\subsection{Multi-Project Organization Method/Scrum}
\fix{}{Might need a bit of editing depending on how the scrum of scrums
progresses} This semester we have organized the multi-project as a scrum of
scrums. One group serves as the scrum master while the remaining 6 groups each send 1
member. These 10 people meet once every week to discuss the multi-project
and after each sprint decide which tasks from the product backlog are suitable
for the sprint backlog. They then further divide the sprint backlog among the
groups such that each group have a suitable number of tasks for each sprint.\nl



\subsection{Role Distribution}

Through a meeting with one of the customers it was decided that the goal of the
project was to make a minimal viable product. They also provided some requests
and wishes in regards to what they considered viable. This has lead to the
following distribution of labour:

\begin{table}[H]
\centering
\caption{Jonathans caption}
\label{Jonathans-label}
\begin{tabular}{|l|l|}
\hline
Group & Main topic \\ \hline
sw609 & Stability \\ \hline
sw610 & Weekplaner \\ \hline
sw611 & Scrum masters \\ \hline
sw612 & Product owners \\ \hline
sw613 & Rest client \\ \hline
sw614 & Infrastructure \\ \hline
sw615 & Rest server \\ \hline
\end{tabular}
\end{table}





\subsection{Tools}

Phabricator

Discord


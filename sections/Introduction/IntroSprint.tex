\chapter{Introduction(initial sprint)}

In this project we focus on improving the stability of the GIRAF project. This
chapter will be used to explain the initial phase of the project and what the
different group will have as their primary focus. Furthermore, we will touch
upon how the different groups will work together and which tools are used.


\section{Development Method}
In this section we describe the development method in the GIRAF multi-project.
First we describe 



setting: minimal viable product
% this section/chapter should be used to describe the project in a multi-project
% setting
%This includes an analysis of organizational context



\subsection{Multi-Project Organization Method/Scrum}
\fix{}{Might need a bit of editing depending on how the scrum of scrums
progresses} This semester we have organized the multi-project as a scrum of
scrums. One group serves as the scrum master while the remaining 6 groups each send 1
member. These 10 people meet once every week to discuss the multi-project
and after each sprint decide which tasks from the product backlog are suitable
for the sprint backlog. They then further divide the sprint backlog among the
groups such that each group have a suitable number of tasks for each sprint.\nl

\subsection{Role Distribution}

Through a meeting with one of the customers it was decided that the goal of the
project was to make a minimal viable product. They also provided some requests
and wishes in regards to what they considered viable. This has lead to the
following distribution of labour:

\begin{table}[H]
\centering
\begin{tabular}{|l|l|l|}
\hline
Group & Focus & Description \\ \hline
sw609 & Stability & Works on the providing stability to weekplanner and its
dependencies.\\\hline 
sw610 & Weekplanner & Works on getting weekplanner into a working
state.\\
\hline sw611 & Scrum masters & Organizes the scrum meetings. \\\hline 
sw612 & Product owners & Creates and describes tasks, also acts as contact to
customers. \\\hline 
sw613 & REST client & \\ \hline

sw614 & Infrastructure & \\ \hline

sw615 & REST server & \\ \hline

\end{tabular}
\caption{The focus of each of the groups}
\label{GroupDivision}
\end{table}

The focus this semester is as such on two distinct parts, the weekplanner and
the REST API. The weekplanner and its dependencies are considered to be the
minimal viable product. The REST API and server focus is considered neccessary
since it would otherwise be difficult to use GIRAF in any sizable institution.

add something about whats good about the way the groups have been split, such as
idea of ownership vs able to get help from other groups vs specilization.

Our focus this semester is as seen above in \autoref{GroupDivision} on providing
stability to the weekplanner app as well as its dependencies\ldots



 
\subsection{Tools}

Phabricator

Git

GIRAF website

Discord

Android studio?

Code review?


\chapter{Introduction}
The GIRAF project is an ongoing development process, which aims to develop a
number of applications for Android tablets, which are to be used by children
with varying degrees of autism. The project is being developed by bachelor
students at Aalborg University, and has been under development since early 2011.
Each semester, the new bachelor students must work together to analyze the state
of the project, and determine which improvements they intend to implement.

\section{State of GIRAF}

The current implementation of the GIRAF project consists of 10 different
interconnected applications, which are presented below in \autoref{GApps} along
with their development status this semester.

\begin{table}[H]
\centering
\begin{tabular}{|p{2.8cm}|p{2.2cm}|p{7cm}|}
\hline
Application			& Status & Description \\\hline
Launcher  			& Worked on & The main application of GIRAF, controls access
to other GIRAF applications. \\\hline 
Sequence	       	& Ignored & A tool to create a sequence of pictograms which can
communicate a process.\\\hline 
PictoCreator  	   	& Ignored & Allows the user to manually create pictograms
which can be saved on the database.\\\hline
PictoReader	   		& Ignored & A tool which can read the text of an associated
pictrogram aloud.\\\hline
LifeStories	   		& Ignored & A tool which allows a citizen to describe their day
using pictograms.\\\hline 
Timer     	   		& Ignored & Allows guardians to create a timer which
limits how long a citizen can use an application.\\\hline 
Weekplanner 		& Worked on & Allows for the creation of schedules for citizens
using pictograms. \\\hline 
Category Tool		& Ignored & A tool which allows pictograms to be grouped together
in a specific category. \\\hline 
Category Game   	& Ignored & A game which teaches citizens to associate
different object by a common feature. \\\hline 
Voice Game  		& Ignored & A game which allows citizens to control an object
using the volume of their voice. \\\hline
\end{tabular} 
\caption{The development status of the applications available to GIRAF}
\label{GApps}
\end{table}

The previous semester ended with the Voice Game application working as a
standalone in an acceptable state. The other applications are in a working
state, but have several issues, such as bugs and missing features which
prohibit their use by the customer.\\
Based on what the customer want, the goal of the multi-project is to make GIRAF
a minimal viable product.
  
\section{Project Goals}
\fix{This section requires more work - add launcher and PictoSearch
description}{} During the first meeting with representatives from one of the
institutions which GIRAF cooperate with, we were given a brief introduction to
how they curretly work and interact with the citizens. The majority of their
feedback concern a desire for a working version of GIRAF, especially something
that means an easier time of managing the schedules of citizens. This feedback
is the reason why GIRAF is to be developed as a ``minimum viable product'',
with a focus on the weekplanner, server and database.\nl

The weekplanner is chosen as it is an application that is intended to manage
the citizens' schedules. The core problems present in this application is
the occasional crashes, long load times when browsing for new pictograms and a
sometimes confusing user interface. These issues makes the application currently
unusable for the guardians and citizens. They need the product to be stable and
quick to use, as it needs to be used often and by people with varying degrees of
experience using computers. The server and database is chosen as any sizable
institutions will need an easy, quick and reliable way to transfer data between
devices using GIRAF.

% \begin{itemize}
%   \item Implement an acceptable version of the app ``Weekplanner''
%   	\begin{enumerate}
%   		\item Fix possible crashes
%   		\item Identify problems in Weekplanners dependencies
%   		\item Fix bugs in the launcher App
% 	\end{enumerate}
%   \item Simplify the database communication system
%    	\begin{enumerate}
%   		\item Allow for syncronization of data
%   		\item Increase safety of stored data
%   		\item Create a useful base for future student to improve
% 	\end{enumerate}
% \end{itemize}

\subsection{Launcher}
\fix{Add description of launcher and add pictures}{}

The launcher as of February 2017 contains several different menus. 


The user is able to login using a QR code. The screen shows a pictogram of
someone scanning a QR code while showing the user what is seen through the
camera. There was work done in the prior semester towards a login system using
passwords instead of using QR codes.\fix{Do we need a citation to a project from
last semester? if yes point to Voice Game ch.15}{}

%show pic of login screen

There are two types of users, the guardian type and the citizen type, each with
a variation of the same menu. The citizen has the simplest of version, with
access to three buttons

%show citizen menu and guardian menu side-by side

The guardian has an additional button that allows the guardian to change the
current profile with that of one of the citizens under its care. There is
currently no similar button on the citizens menu, which means it cannot change
profile to that of a guardian.





- 2 types of user each with an associated menu type
- a login - uses QR codes
- a way a for a guardian type user to manage citizen type users, this
controls which applications are shown to each citizen
- a way to switch from a guardian to a citizen no way from citizen to guardian
- a logout


\subsection{PictoSearch}
\fix{Add description of PictoSearch and add pictures}{}

The pictosearch menu allows the user to search for 

%show pictosearch

\section{Organizational Method}

The groups are divided among the goals according to the desires of each group.
This is meant to increase the quality of the work since each group is able to
specialize on a specific part which they want to work on. It also allows the
groups to work easier together since most of them share some common feature.
This leads to the following distribution of labour:

\begin{table}[H]
\centering
\begin{tabular}{|p{2cm}|p{3cm}|p{8cm}|}
\hline
Group & Focus & Description \\ \hline
SW609 & weekplanner depedencies & Works on the providing stability to
weekplanner and its dependencies.\\\hline 
SW610 & weekplanner & Works on getting weekplanner into a working
state.\\\hline 
SW611 & Scrum masters & Organizes the scrum meetings. \\\hline 
SW612 & Product owners & Creates and describes tasks, also acts as contact to
customers. \\\hline 
SW613 & REST client & \\ \hline

SW614 & Infrastructure & \\ \hline

SW615 & REST server & \\ \hline

\end{tabular}
\caption{The focus of each of the groups}
\label{GroupDivision}
\end{table}

As can be seen in \autoref{GroupDivision}, our group will focus
on improving the stability of the weekplanner through fixing the
bugs and refactoring the code written in its dependencies. As the project
is divided into a number of sprints.

\subsection{Development Method}

Based on the experiences from the previous semester's students, we choose to
make use of an agile developement process, where each group is organized through
the use of Scrum. This means that the development process is divided into a
number of ``sprints'', in which no more tasks can be added to the backlog, and
the groups work to resolve all currently defined tasks. We have organized the
multi-project this semester as a scrum of scrums. One group serves as the scrum
master while the remaining 6 groups each send 1 member, see
\autoref{ProjectStruct}.

\figx[0.80]{ProjectStruct}{The project structure used this semester.}

These 10 people meet once every week to discuss the multi-project and after each
sprint decide which tasks from the product backlog are suitable for the sprint
backlog. They then further divide the sprint backlog among the groups such that
each group have a suitable number of tasks for each sprint. The multi-project
itself is divided into 4 different sprints, as can be seen in
\autoref{TableActivity}. 

\begin{table}[H]
\centering
\begin{tabular}{|l|l|}
\hline
Activity & Date \\ \hline
1. Sprint & 20/2 - 15/3 \\\hline 
2. Sprint & 15/3 - 7/4\\\hline 
3. Sprint & 7/4 - 1/5\\\hline 
4. Sprint & 1/5 - 22/5\\\hline 
Project delivery date & 29/5\\\hline
\end{tabular}
\caption{The GIRAF activities and their dates.}
\label{TableActivity}
\end{table}

Based on this development framework, this report will be used to document and
reason about the different needed improvements, which have been assigned to our
group, SW609.


\section{Tools}

Phabricator - important

Git - important

Android studio?

Code review?

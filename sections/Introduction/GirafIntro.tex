\chapter{Introduction}
The GIRAF project consists of developing a set of applications for Android
tablets, which aim to assist citizens with autism, and to help their guardians in 
structuring the citizens daily lives. The project is a continous development
effort by bachelor students at Aalborg University, which has been
ongoing since early 2011. Each year, the new bachelor students must work
together to analyze the state of the project, determine which improvements
they intend to implement, and work together to organize and structure the
development process.\nl

The purpose of the GIRAF project is to give the students experience in analysis,
design, implementation and evaluation of complex software systems, as a part of
a larger development environment. As such, the focus of this project is to
document the collaborative development effort between the individual project
groups.


\section{State of GIRAF}
During the previous years development effort, the project groups were focused
on implementing a viable front-end for the GIRAF applications, such that they
would be able to conduct extensive usability tests. This resulted in them
attempting to make the applications look like the functioned from the users
perspective, while neglecting to develop parts of the necessary back-end.
This has resulted in a situation where the applications' front-ends are
practically finished, while large parts of the back-end still needs to be
developed.\nl

As of the start of this semester, the current implementation of the GIRAF
project consists of 10 different interconnected applications, which are
presented below in \autoref{GApps}. The applications interconnectivity mostly
relies on the GIRAF launcher, which is used to manage system users, their
permissions, and launching the GIRAF applications themselves. 

\begin{table}[H]
\centering
\begin{tabular}{|p{2.7cm}|p{6cm}|p{5cm}|}
\hline
Application			& Description & Status \\\hline
Launcher  			& The main application of GIRAF, controls access to other GIRAF
applications. & Working. Needs bugfixing and possible redesign. \\\hline
Sequence & A tool to create a sequence of pictograms which can communicate a
process. & Broken. Mostly implemented. Needs bugfixing.\\\hline 
PictoCreator  	   	& Allows the user to
manually create pictograms which can be saved on the database. & Broken. Mostly
implemented. Needs database implementation.\\\hline 
PictoReader	   		& A tool
which can read the text of an associated pictrogram aloud. & Working.
Needs bugfixing.\\\hline 
LifeStories	   		& A tool which allows a citizen to
describe their day using pictograms. & Broken. Unfinished features. \\\hline 
Timer     	   		& Allows guardians to create a timer which
limits how long a citizen can use an application. & Broken. Unfinished
features.\\\hline 
Weekplanner 		& Allows for the creation of schedules for citizens
using pictograms. & Working. Needs bugfixing.\\\hline 
Category Tool		& A tool which allows pictograms to be grouped together
in a specific category. & Working. Needs bugfixing. Possible redesign.\\\hline 
Category Game   	& A game which teaches citizens to associate
different object by a common feature. & Working. Needs bugfixing.\\\hline 
Voice Game  		& A game which allows citizens to control an object
using the volume of their voice. & Working. Needs bugfixing.\\\hline
\end{tabular} 
\caption{The development status of the applications available to GIRAF}
\label{GApps}
\end{table}

The previous year's development ended with several applications being in a
working state, but having several issues, such as bugs and missing features
which prohibit their use by the customer.

\section{Development Method}
Based on the experiences from the previous semester's students, we have chosen
to make use of an agile developement process, where each group is organized through
the use of Scrum. This means that the development process is divided into a
number of ``sprints'', in which no more development tasks can be added to the
backlog, and the groups work to resolve all currently defined tasks. We have
organized the multi-project this semester as a scrum of scrums. One group serves
as the scrum master while the remaining 6 groups each have one member who
attends the scrum meetings. This organizational structure can be seen in
\autoref{ProjectStruct}.

\figx[0.80]{ProjectStruct}{The project structure used this semester.}

These 10 people meet once every week to discuss the multi-project and after each
sprint decide which tasks from the product backlog are suitable for the sprint
backlog. They then further divide the sprint backlog among the groups such that
each group have a suitable number of tasks for each sprint. The multi-project
itself is divided into 4 different sprints, which are defined in
\autoref{TableActivity}. 

\begin{table}[H]
\centering
\begin{tabular}{|l|l|}
\hline
Activity & Date \\ \hline
1. Sprint & 20/2 - 15/3 \\\hline 
2. Sprint & 15/3 - 5/4\\\hline 
3. Sprint & 5/4 - 1/5\\\hline 
4. Sprint & 1/5 - 22/5\\\hline 
Project delivery date & 29/5\\\hline
\end{tabular}
\caption{The GIRAF activities and their dates.}
\label{TableActivity}
\end{table}

\subsection{Organizational Method}
Since all parts of the GIRAF system are in need of further development, the
goals for this semester have been based on what each group has requested to work
on. As such, the project goals are divided among the groups in such
a way that the each group works on a part of the project they want. This is
meant to increase the quality of the work since each group is able to specialize
on a specific part which they want to work on. It also allows the groups to work
easier together since most of them share some common feature. This leads to the
following distribution of responsibility:

\begin{table}[H]
\centering
\begin{tabular}{|p{2cm}|p{3cm}|p{8cm}|}
\hline
Group & Focus & Description \\ \hline
SW609 & Weekplanner depedencies. & Works on providing stability to
weekplanner and its dependencies.\\\hline 
SW610 & Weekplanner & Works on getting weekplanner into a working
state.\\\hline 
SW611 & Scrum master & Organizes the scrum meetings. \\\hline 
SW612 & Product owner & Creates and describes tasks, also acts as a contact to
the users. \\\hline 
SW613 & REST client & Works on REST client libraries.\\ \hline

SW614 & Infrastructure & Works on getting the server up and running in a way
that can allow it to be easily maintained. \\\hline

SW615 & REST server & Works on the REST server libraries.\\ \hline

\end{tabular}
\caption{The focus of each of the groups}
\label{GroupDivision}
\end{table}

\section{Project Goals}
As a part of initial introduction to the project, we have held a meeting with
representatives from one of the institutions which coorporate with the GIRAF
project. At this meeting, we were given a brief introduction to how they
curretly work and how they interact with the citizens during a normal day. The
majority of their feedback concern a desire for a working version of GIRAF,
especially something that means an easier time of managing the schedules of
citizens. This feedback is the reason why GIRAF is to be developed as a
``minimum viable product'', with a focus on the weekplanner, server and
database.\nl

The weekplanner is chosen as it is an application that is intended to manage
the citizens' schedules. The core problems present in this application is
the occasional crashes, long load times when browsing for new pictograms and a
sometimes confusing user interface. These issues makes the application currently
unusable for the guardians and citizens. They need the product to be stable and
quick to use, as it needs to be used often and by people with varying degrees of
experience using computers. The server and database is chosen as any sizable
institutions will need an easy, quick and reliable way to transfer data between
devices using GIRAF.\nl

As can be seen in \autoref{GroupDivision}, our group will focus on improving the
stability of the weekplanner through refactoring and improving the code of its
dependencies. This will mainly be done through work on the Launcher and a
library called \textc{pictosearch-lib}. Based on the presented development
framework, and the chosen project goals, this report will be used to document
the development and organizational processes of the tasks which have been
assigned to our group, SW609.

\section{Main Applications}
Since our responsibility is to improve the Launcher and the
\textc{pictosearch-lib} library, this section will be used to present the
two applications, and explain their user interface.

\subsection{Launcher}\label{LauncherReview}
The launcher as of February 2017 contains a login screen, different menues for
each of the user types and an options menu. The user is able to login using a
QR code, as can be seen in \autoref{Login}.In the lower left corner there is
currently a shortcut used for testing purposes which skips the QR code. There
was work done in the prior semester towards a login system using passwords
instead of using QR codes. This includes design sketches and a simple UI
implementation.

\figx[0.25]{Login}{The login screen shows a pictogram of someone scanning a QR
code while the user can see what it scans.}

After logging in, the user is taken to the guardian menu, which has access to 4
buttons and a list of applications which the user has access to,
see \autoref{MenuGuardian}. Guardian and citizens have similar menus with
exception that citizens only have access to button \textbf{2} and \textbf{3}.

\figx[0.32]{MenuGuardian}{The guardians menu screen.}
% This picture is smaller than the others

Button \textbf{1} allows the guardian to change the current profile with that of
one of the citizens under its care, see \autoref{ChooseCitizen}. There is currently no
similar button on the citizens menu, this means that a citizen cannot be changed
back to a guardian without logging out.

%skal billedet fjernes?
\figx[0.25]{ChooseCitizen}{The menu changes the guardian to one its citizens.}

Button \textbf{2} allows the user to log out. Button \textbf{3} provides the
user with a quick description of each of the other buttons. Button \textbf{4}
provides access to the settings, see \autoref{OptionsGuardian}, which is a
collection of menus regarding how many and which applications can the be shown
on \textbf{5} in \autoref{MenuGuardian}. 

\figx[0.32]{OptionsGuardian}{The settings menu}

The settings menu has a collection of universal buttons which can be accessed
from every menu except button \textbf{5}. Button \textbf{1} allows the guardian
to exit the settings menu, button \textbf{2} allows the guardian to change the settings for
the citizens in its care. This leads to a settings menu similar to the guardians.
Button \textbf{3} allows the guardian to control how many applications are
shown for the current user, which can be seen in \textbf{7}. Button \textbf{4}
allows the guardian to control which applications are shown to user. This is
divided into GIRAF, Android and Store applications, see
\autoref{OptionsGuardian2}. Button \textbf{6} as in the other menus provide the
user with a quick description of each of the other buttons.

\figx[0.25]{OptionsGuardian2}{The menu which allows the guardian to show }

\subsection{PictoSearch}\label{PictoSearchReview}
As of March 2017, the \textc{pictosearch-lib} is a library containing all
classes relating to the search of pictograms. As it is a library, it does not
represent a single application, but is implemented in projects such as
\textc{weekday-planner} and \textc{pictoreader}.\nl

The pictosearch library's visual components consists of a number of
elements. At the top right is a search-bar, which is used to search for
pictograms by name. To the left is a vertical bar used to showcase what
pictograms the user have selected. The large scrollable area in the middle of
the screen is used to showcase all the pictograms which correspond to the given
search criteria. All of these elements can be seen below in
\autoref{psearchScrn}.
 
\figx[0.3]{psearchScrn}{Graphical user interface for the pictosearch library.}

After choosing a number of pictograms, the user can click the green checkmark in
the top right corner, and the respective \textc{Pictogram} objects are passed to
the parent application.
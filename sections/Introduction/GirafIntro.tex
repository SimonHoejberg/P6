\chapter{Project Context}
The GIRAF project is an ongoing development process, which aims to develop a
number of applications for tablets, which are to be used by children with
varying degrees of autism. The project i being developed by bachelor student at
Aalborg University, and has been under development since early 2011. Each
semester, the new bachelor students must work together to analyze the state of
the project, and determine which improvements they intend to implement.\nl

Based on the experiences passed down from the previous students, we have chosen
to make use of an agile developement process, where each group is organized
through the use of Scrum. In other words, this semester we have organized the
multi-project as a scrum of scrums. One group serves as the scrum master while
the remaining 6 groups each send 1 member. These 10 people meet once every week
to discuss the multi-project and after each sprint decide which tasks from the
product backlog are suitable for the sprint backlog. They then further divide
the sprint backlog among the groups such that each group have a suitable number
of tasks for each sprint.\nl

Based on this development framework, this report will be used to document and
reason about the different needed improvements, which have been assigned to our
group, SW609.

\section{Initial Problem}
During the initial phases of the project we had a meeting with representatives
from one of the institutions which use GIRAF on a daily basis. Based on their
feedback on the curret state of the system, we have made the following list of
improvements, which are considered essential for the success of this years
developement of the GIRAF project.

\begin{itemize}
  \item Implement an acceptable version of the app ``Weekplanner''
  	\begin{enumerate}
  		\item Fix possible crashes
  		\item Identify problems in Weekplanners dependencies
  		\item Fix bugs in the launcher App
	\end{enumerate}
  \item Simplify the database communication system
   	\begin{enumerate}
  		\item Allow for syncronization of data
  		\item Increase safety of stored data
  		\item Create a useful base for future student to improve
	\end{enumerate}
\end{itemize}


In this project, our group will focus on improving the stability of the GIRAF
project, through the fixing of bugs and refactoring of code written by
the previous bachelor students. As the project is divided into a number of
sprints, \autoref{} is used to document our tasks in the initial sprint of the
project, and what the various groups will have as their primary focus.
Furthermore, as this semester focuses on teamwork and organizing groups of
developers, we will touch upon how the different groups will work together and
which tools are used to achieve this unified development process.

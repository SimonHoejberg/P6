\chapter{Introduction}

Bla bla bla working with 4 different institutions
android\ldots


\section{Project Context}
\fix{how much should be moved into introduction?}{}
The GIRAF project is an ongoing development process, which aims to develop a
number of applications for Android tablets, which are to be used by children
with varying degrees of autism. The project is being developed by bachelor students
at Aalborg University, and has been under development since early 2011. Each
semester, the new bachelor students must work together to analyze the state of
the project, and determine which improvements they intend to implement.\nl

The current implementation of the GIRAF project consists of 10 different
interconnected applications, which are presented below in \autoref{GApps} along
with their development status this semester.

\begin{table}[H]
\centering
\begin{tabular}{|p{2.8cm}|p{2.2cm}|p{7cm}|}
\hline
Application			& Status & Description \\\hline
Launcher  			& Worked on & The main application of GIRAF, controls access
to other GIRAF applications. \\\hline 
Sequence	       	& Ignored & A tool to create a sequence of pictograms which can
communicate a process.\\\hline 
PictoCreator  	   	& Ignored & Allows the user to manually create pictograms
which can be saved on the database.\\\hline
PictoReader	   		& Ignored & A tool which can read the text of an associated
pictrogram aloud.\\\hline
LifeStories	   		& Ignored & A tool which allows a citizen to describe their day
using pictograms.\\\hline 
Timer     	   		& Ignored & Allows guardians to create create a timer which
limits how long a citizen can use an application.\\\hline 
Weekplanner 		& Worked on & Allows for the creation of schedules for citizens
using pictograms. \\\hline 
Category Tool		& Ignored & A tool which allows pictograms to be grouped together
as a specific category. \\\hline 
Category Game   	& Ignored & A game which teaches citizens to associate
different object by a common feature. \\\hline 
Voice Game  		& Working & A game which allows citizens to control an object
using their voice volume. \\\hline
\end{tabular} 
\caption{The development status of the applications available to GIRAF}
\label{GApps}
\end{table}

The previous semester ended with the Voice Game application working as a
standalone in an acceptable state. The other applications are in a working
state, but have several issues, such as bugs and missing features which
prohibit their use by the customer.
  
\subsection{Initial Problem}
During the initial phases of the project we had a meeting with representatives
from one of the institutions which use GIRAF on a daily basis. Based on their
feedback on the curret state of the system, we have made the following list of
improvements, which are considered essential for the success of this years
developement of the GIRAF project.

\begin{enumerate}
  \item We need to get the weekplanner into a acceptable state, where it can be
  used reliably by the citizens with no unexpected errors occuring. It also
  needs to have the necessary functionality to satisfy the customer.
  \item We need the server to be reliable and store the data safely, this is
  considered necessary since it would otherwise be difficult to use GIRAF in any
  sizable institution.
\end{enumerate}

These requirements form a framework for developing what we consider to be a
``minimum viable product''. In order to further elaborate upon the requirements
and to better be able to divide the project into areas of responsibility, we
made the following list:

\begin{itemize}
  \item Implement an acceptable version of the app ``Weekplanner''
  	\begin{enumerate}
  		\item Fix possible crashes
  		\item Identify problems in Weekplanners dependencies
  		\item Fix bugs in the launcher App
	\end{enumerate}
  \item Simplify the database communication system
   	\begin{enumerate}
  		\item Allow for syncronization of data
  		\item Increase safety of stored data
  		\item Create a useful base for future student to improve
	\end{enumerate}
\end{itemize}

The groups are divided among the goals according to the desires of each group.
This is meant to increase the quality of the work since each group is able to
specialize on a specific part. It also allows the groups to work easier together
since most of them share some common feature. These two goals lead to the
following distribution of labour:

\begin{table}[H]
\centering
\begin{tabular}{|p{2cm}|p{3cm}|p{8cm}|}
\hline
Group & Focus & Description \\ \hline
SW609 & Weekplanner depedencies & Works on the providing stability to
weekplanner and its dependencies.\\\hline 
SW610 & Weekplanner & Works on getting weekplanner into a working
state.\\
\hline SW611 & Scrum masters & Organizes the scrum meetings. \\\hline 
SW612 & Product owners & Creates and describes tasks, also acts as contact to
customers. \\\hline 
SW613 & REST client & \\ \hline

SW614 & Infrastructure & \\ \hline

SW615 & REST server & \\ \hline

\end{tabular}
\caption{The focus of each of the groups}
\label{GroupDivision}
\end{table}

As can be seen in \autoref{GroupDivision}, in this project, our group will focus
on improving the stability of the GIRAF project, through the fixing of bugs and
refactoring of code written by the previous bachelor students. As the project is
divided into a number of sprints, \autoref{} is used to document our tasks in
the initial sprint of the project. Furthermore, as this semester focuses on
teamwork and organizing groups of developers, we will touch upon how the
different groups will work together and which tools are used to achieve this
unified development process.

\section{Organizational Method}

Based on the experiences from the previous semester's students, we choose to
make use of an agile developement process, where each group is organized
through the use of Scrum. We have organized the multi-project this semester as
a scrum of scrums. One group serves as the scrum master while the remaining 6
groups each send 1 member, see \autoref{ProjectStruct}.

\figx[0.80]{ProjectStruct}{The project structure used this semester.}

These 10 people meet once every week to discuss the multi-project and after each
sprint decide which tasks from the product backlog are suitable for the sprint
backlog. They then further divide the sprint backlog among the groups such that
each group have a suitable number of tasks for each sprint. The multi-project
itself is divided into 4 different sprints, as can be seen in
\autoref{TableActivity}. 

\begin{table}[H]
\centering
\begin{tabular}{|p{2cm}|p{3cm}|p{8cm}|}
\hline
Activity & Date \\ \hline
1. Sprint & 20/2 - 15/3 \\\hline 
2. Sprint & 15/3 - 7/4\\\hline 
3. Sprint & 7/4 - 1/5\\\hline 
4. Sprint & 1/5 - 22/5\\\hline 
GIRAF Showcase & ??\\\hline
Project delivery date & 29/5\\\hline
\end{tabular}
\caption{The activities and their date.}
\label{TableActivity}
\end{table}

Based on this development framework, this report will be used to document and
reason about the different needed improvements, which have been assigned to our
group, SW609.


\section{Tools}

Phabricator - important

Git - important

Android studio?

Code review?

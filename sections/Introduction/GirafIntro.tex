\chapter{Introduction}
The GIRAF project consists of developing a set of applications for Android
tablets, which aim to assist citizens with autism, and to help their guardians in 
structuring the citizens daily lives. The project is a continous development
effort by bachelor students at Aalborg University, which has been
ongoing since early 2011. Each year, the new bachelor students must work
together to analyze the state of the project, determine which improvements
they intend to implement, and work together to organize and structure the
development process.\nl

The purpose of the GIRAF project is to give the students experience in analysis,
design, implementation and evaluation of complex software systems, as a part of
a larger development environment. As such, the focus of this project is to
document the collaborative development effort between the individual project
groups.


\section{State of GIRAF}
During the previous years development effort, the project groups were focused
on implementing a viable front-end for the GIRAF applications, such that they
would be able to conduct extensive usability tests. This resulted in them
attempting to make the applications look like they functioned from the users
perspective, while neglecting to develop parts of the necessary back-end.
This has resulted in a situation where the applications' front-ends are
practically finished, while parts of the back-end still needs to be
developed.\nl

As of the start of this semester, the current implementation of the GIRAF
project consists of 10 different interconnected applications, see
\autoref{GApps}. The applications interconnectivity mostly relies on the
launcher, which is used to manage system users, their permissions, and
launching the GIRAF applications themselves.

\begin{table}[H]
\centering
\begin{tabular}{|p{2.7cm}|p{6cm}|p{5cm}|}
\hline
Application			& Description & Status \\\hline
Launcher  			& The main application of GIRAF, controls access to other GIRAF
applications. & Working. Needs bugfixing and possible redesign. \\\hline
Sequence & A tool to create a sequence of pictograms which can communicate a
process. & Broken. Mostly implemented. Needs bugfixing.\\\hline 
PictoCreator  	   	& Allows the user to
manually create pictograms which can be saved on the database. & Broken. Mostly
implemented. Needs database implementation.\\\hline 
PictoReader	   		& A tool
which can read the text of an associated pictrogram aloud. & Working.
Needs bugfixing.\\\hline 
LifeStories	   		& A tool which allows a citizen to
describe their day using pictograms. & Broken. Unfinished features. \\\hline 
Timer     	   		& Allows guardians to create a timer which
limits how long a citizen can use an application. & Broken. Unfinished
features.\\\hline 
Weekplanner 		& Allows for the creation of schedules for citizens
using pictograms. & Working. Needs bugfixing.\\\hline 
Category Tool		& A tool which allows pictograms to be grouped together
in a specific category. & Working. Needs bugfixing. Possible redesign.\\\hline 
Category Game   	& A game which teaches citizens to associate
different object by a common feature. & Working. Needs bugfixing.\\\hline 
Voice Game  		& A game which allows citizens to control an object
using the volume of their voice. & Working. Needs bugfixing.\\\hline
\end{tabular} 
\caption{The development status of the applications available to GIRAF}
\label{GApps}
\end{table}

The previous year's development ended with several applications being in a
working state, but having several issues, such as bugs and missing features
which prohibit their use by the customer.

\section{Development Method}
Based on the experiences from the previous semester's students, we have chosen
to make use of an agile developement process, where each group is organized through
the use of Scrum. This means that the development process is divided into a
number of ``sprints'', in which no more development tasks can be added to the
backlog, and the groups work to resolve all currently defined tasks. We have
organized the multi-project this semester as a scrum of scrums. One group serves
as the scrum master, while the remaining 6 groups each select one member to
attend the scrum meetings. This organizational structure can be seen in
\autoref{ProjectStruct}, where the scrum group organizes the project on a daily
basis, and the different group's representatives partake in weekly meetings with
the scrum group.

\figx[0.80]{ProjectStruct}{The project structure used this semester.}

These 10 people partake in the scrum meetings in order to discuss the
multi-project, what each group is currently working on, and whatever problems
have cropped up between each meeting.\nl

At the beginning of each sprint, all project members meet as a part of a larger
scrum meeting, in order to decide what new tasks can be defined, and which tasks
from the product backlog are suitable for the next sprint's backlog. The scrum
group then further divide the sprint backlog among the groups such that each
group have a suitable number of tasks for each sprint. The multi-project itself
is divided into 4 different sprints, which are defined in
\autoref{TableActivity}.

\begin{table}[H]
\centering
\begin{tabular}{|l|l|}
\hline
Activity & Date \\ \hline
1. Sprint & 20/2 - 15/3 \\\hline 
2. Sprint & 15/3 - 5/4\\\hline 
3. Sprint & 5/4 - 1/5\\\hline 
4. Sprint & 1/5 - 22/5\\\hline 
Project delivery date & 29/5\\\hline
\end{tabular}
\caption{The GIRAF activities and their dates.}
\label{TableActivity}
\end{table}

\section{Project Goals}\label{projectGoals}
As a part of initial introduction to the project, we have held a meeting with
representatives from one of the institutions which coorporate with the GIRAF
project. At this meeting, we were given a brief introduction of how they
curretly work and how they interact with the citizens during a normal day. The
majority of their feedback indicated a desire for a working version of GIRAF,
which they can use to easily manage the schedules of citizens. In the
current implementation of GIRAF, this is done through the use of series of
descriptive pictograms, which depict actions, events, and items in a determined
order. Based on this feedback, the overall goal for this project is to develop
GIRAF into a ``minimum viable product'', with a focus on the
\textc{weekplanner}, \textc{server} and \textc{database}.\nl

The weekplanner has been chosen, as it is the main application which is used
to manage the citizens' schedules. The core problems present in this application is
the occasional crashes, long load times when browsing for new pictograms, and a
occasionally confusing user interface. These issues makes it so, that the
application is currently not usable to an acceptable degree. The users need the
product to be stable and easy to use, as it needs to be used frequently, and by
users who have varying degrees of experience using computers. The server
and database is chosen as any sizable institutions will need an easy, quick and
reliable way to transfer data between devices using GIRAF.\nl

Based on the chosen area of focus for this project, we have chosen to define the
areas of reponsibility for each group based on what they have requested to work
on. The chosen distribution of responsibility can be seen in
\autoref{GroupDivision}.

\begin{table}[H]
\centering
\begin{tabular}{|p{2cm}|p{3cm}|p{8cm}|}
\hline
Group & Focus & Description \\ \hline
SW609 & Weekplanner depedencies. & Works on providing
stability to the Weekplanner by improving libraries and the Launcher.\\\hline 
SW610 & Weekplanner & Works on getting weekplanner into a working
state.\\\hline 
SW611 & Scrum master & Organizes the scrum meetings and helps with the creation of tasks. \\\hline 
SW612 & Product owner &  Acts as a contact to the users.\\\hline 
SW613 & REST client & Works on REST client libraries.\\ \hline

SW614 & Infrastructure & Works on getting the server up and running in a way
that can allow it to be easily maintained. \\\hline

SW615 & REST server & Works on the REST server libraries.\\ \hline

\end{tabular}
\caption{The focus of each of the groups}
\label{GroupDivision}
\end{table}

As can be seen in \autoref{GroupDivision}, our group will focus on improving the
stability of the weekplanner by means of refactoring and improving the code of
its dependencies. This will mainly be done through work on the Launcher and a
library called \textc{pictosearch-lib}. Based on the presented development
framework, and the chosen project goals, this report will be used to document
the development and organizational processes of the tasks which have been
assigned to our group, SW609.

\section{Main Applications}
Since our responsibility is to improve the Launcher and the
\textc{pictosearch-lib} library, this section will be used to present the
two applications, and explain their user interface.

\subsection{Launcher}\label{LauncherReview}
As of February 2017, the Launcher contains a login screen, different menus for
each of the user types and an options menu. The user is able to login using a QR
code, as seen in \autoref{Login}. In the lower left corner there is
currently a shortcut used for testing purposes which bypasses the QR code
requirement. There was work done in the prior semester towards a login system
using passwords instead of using QR codes, including design sketches and a
simple UI implementation.

\figx[0.25]{Login}{The login screen shows a pictogram of someone scanning a QR
code while the user can see what it scans.}

After logging in, the user is taken to the guardian menu, which has access to 4
buttons and a list of applications which the user has access to,
see \autoref{MenuGuardian}. Guardian and citizens have similar menus with
exception that citizens only have access to \textbf{2} which is a logout button,
\textbf{3} which provides a quick description of what the user can do and
\textbf{5} which is a list of the available applications.

\figx[0.37]{MenuGuardian}{The guardians menu screen.}

Through \textbf{1} the guardian can allow a citizen in its care limited access
to the tablet. The settings are accessed via \textbf{4}, which allows the
guardian to control everything from what applications a given citizen can see to
how many applications it can see.


% Button \textbf{1} allows the guardian to change the current profile with that of
% one of the citizens under its care, see \autoref{ChooseCitizen}. There is currently no
% similar button on the citizens menu, this means that a citizen cannot be changed
% back to a guardian without logging out.
% 
% %skal billedet fjernes?
% \figx[0.25]{ChooseCitizen}{The menu changes the guardian to one its citizens.}
% 
% Button \textbf{2} allows the user to log out. Button \textbf{3} provides the
% user with a quick description of each of the other buttons. Button \textbf{4}
% provides access to the settings, see \autoref{OptionsGuardian}, which is a
% collection of menus regarding how many and which applications can the be shown
% on \textbf{5} in \autoref{MenuGuardian}. 
% 
% \figx[0.32]{OptionsGuardian}{The settings menu}
% 
% The settings menu has a collection of universal buttons which can be accessed
% from every menu except button \textbf{5}. Button \textbf{1} allows the guardian
% to exit the settings menu, button \textbf{2} allows the guardian to change the settings for
% the citizens in its care. This leads to a settings menu similar to the guardians.
% Button \textbf{3} allows the guardian to control how many applications are
% shown for the current user, which can be seen in \textbf{7}. Button \textbf{4}
% allows the guardian to control which applications are shown to user. This is
% divided into GIRAF, Android and Store applications, see
% \autoref{OptionsGuardian2}. Button \textbf{6} as in the other menus provide the
% user with a quick description of each of the other buttons.
% 
% \figx[0.25]{OptionsGuardian2}{The menu which allows the guardian to show }

\subsection{PictoSearch}\label{PictoSearchReview}
As of March 2017, the \textc{pictosearch-lib} is a library containing all
classes relating to the search of pictograms. As it is a library, it does not
represent a single application, but is implemented in projects such as
\textc{weekday-planner} and \textc{pictoreader}.\nl

The pictosearch library's visual components consists of a number of
elements. At the top right is a search-bar, which is used to search for
pictograms by name. To the left is a vertical bar used to showcase what
pictograms the user have selected. The large scrollable area in the middle of
the screen is used to showcase all the pictograms which correspond to the given
search criteria. All of these elements can be seen in
\autoref{psearchScrn}.
 
\figx[0.3]{psearchScrn}{Graphical user interface for the pictosearch library.}

After choosing a number of pictograms, the user can click the green checkmark in
the top right corner, and the respective \textc{Pictogram} objects are passed to
the parent application.

\section{Giraf Problems}\label{GirafProblems}
Based on initial testing of the GIRAF Launcher, and the component-lib and
pictosearch-lib libraries, we have been able to determine a number of problems
related to these system components. As the project goal is to create a finished
product which can be distributed to users of the GIRAF system, fixing these problems
will be instrumental to succeeding in doing this.

\subsection{Launcher}
As described in \autoref{LauncherReview}, the launcher is the element
responsible for launching other GIRAF applications, and managing the global
settings of individual users. The following problems have been found in the
launcher:
\begin{enumerate}
  \item The launcher is very unresponsive when starting up, and takes up to a
  minute before we can access the main screen.
  \item The launcher experience frequent crashes, especially when switching
  between activities.
  \item The launcher only loads user-data on startup. This can lead to outdated
  data being used.
  \item The launcher uses the db-lib library for server communication, and needs
  to implement the new REST framework, which should be developed during the project.
\end{enumerate}

\subsection{Pictosearch-lib}
As described in \autoref{PictoSearchReview}, the pictosearch-lib library
contains the GUI and functionality required to request and retrieve pictograms
from the database. The following problems have been found in this library:
\begin{enumerate}
  \item Searching for pictograms can take a very long time, and sometimes it
  will get stuck and never finish.
  \item Some search-inputs crashes the application, namely special characters
  and commas.
  \item The library uses the db-lib library for server communication, and needs
  to implement the new REST framework, which should be developed during the project.
\end{enumerate}

\subsection{Component-lib}
This library contains various components used throughout all GIRAF applications.
These components range from GUI elements to backend calculations and
controller/helper classes. The following problems have been found in this
libary:
\begin{enumerate}
  \item The library contain large amounts of deprecated code and classes, which
  makes the library incomprehensible.
  \item The library uses the db-lib library for server communication, and needs
  to implement the new REST framework, which should be developed during the project.
\end{enumerate}
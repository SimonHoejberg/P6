\section{Retrospective}\label{S2Retro}
As with the first sprint, at the end of the second sprint, we held a meeting with
the other groups in order to reflect upon the development/organizational
process.\nl

The main problem encountered during this sprint is the fact that our build
tools (Artifactory/Jenkins) are not operational. This has led to a
situation where we are very limited in what programming tasks we can actually
complete. For the Launcher application this is not a major problem as we
acquired the necessary dependencies before the build tools broke down, but for
other applications and libraries, this is a major problem. For the most part, as
long as we are not adding new features, this allows for us to implement new
code, but not to test it. Based on this problem, we have been unable to finish
most of our tasks, as testing is an important part of developing the
software.\nl

Another problem we encountered was, that some of our tasks have had major
overestimates in the time required to complete them. This is a problem, as
workload is distributed based on the estimated time required, and as such, we
were running out of work in the last week of the sprint. It should be noted that
this problem partly exists as we are unable to allocate time to test our code
due to the build tools being broken. Additionally, due to a problem with the
sprint's schedule, the end of the sprint was moved back by three days, which
luckily prevented us from running completely out of work.

\subsection{Evaluation}
The sprint ended with 51.1\% of tasks being declared resolved. A lot of the
unresolved tasks have solutions but have not been tested since Jenkins,
Artifactory and Gradle currently cannot be used to build the project.
The burndown chart is not as informative as is could
be, as a lot of tasks are unresolved due to a lack of
possible testing. The burndown chart can be seen in \autoref{Burndown2}.

\figx[0.5]{Burndown2}{The burndown chart for the second sprint}






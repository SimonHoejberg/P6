\chapter{Sprint: 02}
Following the first sprint, each group has gained a deeper understanding of how
the individual applications are designed and implemented. Based on this
knowledge, this sprint will be used to further develop the features which were
worked on by the previous semester's students.\nl

While our initial area of responsibility were focused on the \textc{Launcher}
and \textc{Pictogram Searcher}, for this sprint we have chosen to expand this to
also include the \textc{Pictogram Reader} and developing a new \textc{Login
System}. This choice is based on the fact, that currently only two groups are
working on developing the front-end for applications (SW609 and SW610), and our
group is currently best suited to handle these new tasks. Additionally, through
a discussion with the other groups, we have concluded that our projects should
include documenting the applications/classes we are working on. This
documentation should be added to the GIRAF wiki.

\section{Tasks}
Based on our expanded area of responsibility and the need to develop a new login
system, we have been assigned the tasks presented in \autoref{SprintTwoTasks}.

\begin{table}[H]
\centering
\begin{tabular}{|l|l|}
\hline
Number	& Description												\\\hline
T634  	& Wiki: Continually update Launcher info on the wiki\\\hline 
T671    & Implement system to retrieve stack-trace from system crashes\\\hline 
T705	& Launcher: Evaluate/Change need to log out when changing  to guardian\\\hline 
T708    & Meeting: Discuss overall design goals for the GIRAF GUI \\\hline
T722   	& Launcher: Save login session. Prevent automatic logout before 8 hours \\\hline 
T724   	& Launcher: Prevent crash caused by other applications crashing\\\hline
T725  	& Search: Improve overall performance (If possible)			   							\\\hline
T726    & Meeting: Discuss REST interface and database communication \\\hline
T727    & Reader: Fix crash on failed server communication  				\\\hline 
T728	& Reader: Refactor and CheckStyle \\\hline
T731    & LogIn: Design new login system GUI               			\\\hline
T732    & Launcher: Evaluate what elements should be visible to child/guardian\\\hline 
T733    & Search: Implement vertical screen functionality    			\\\hline 
T739    & Wiki: Document pictosearch info on the wiki	\\\hline
\end{tabular}
\caption{Tasks for the second sprint (Group SW609)} 
\label{SprintTwoTasks}    
\end{table} 

In the following subsections, we will elaborate upon each of the task, and
document what actions we have taken in order to resolve the issues.

\subsection{T722 - Prevent login session from expiring}
While testing the launcher during the first sprint, we noticed that the login
session would sometimes expire before the defined interval of 8 hours.\nl

Placeholder\ldots

\subsection{T724 - Launcher: Prevent crash when child application crashes}
Placeholder\ldots

\subsection{T733 - Search: Implement vertical screen functionality}
During the first sprint, group SW610 was testing the Weekplanner application,
and noticed that some activities did not support vertical orientation of the
screen. Due to the inconsistancy of only some activities supporting this, we
have agreed to implement vertical functionality in the pictosearch activity.\nl

If an android activity is to support changing the orientation of its GUI, it
must contain a unique specification for each orientation. This is done in the
form of folder, containing an .XML file specifying the GUI. This means, that its
resource folder \textc{res} must contain subfolders named \textc{layout},
\textc{layout-port}, and \textc{layout-land}. The folders \textc{layout-port}
and \textc{layout-land} contain a defined layout for vertical and horizontal
orientation respectively, while \textc{layout} contains a default layout, which
is used if the other folders do not exist.\nl

In the current version of \textc{pictosearch-lib}, only the default folder
\textc{layout} exists. This horizontal layout can be seen in
\autoref{SearchHorizon}.

\figx{SearchHorizon}{Horizontal design for the
\textc{activity\_picto\_admin\_main.xml} file, as presented in the Android
Studio designer.}
    
In order to allow for vertical orientation, we have chosen to redesign and
implement a new version of the activity. Simply put, we have copied the files
for the horizontal layout, added them to a new folder called
\textc{layout-port}, and redesigned the layout to be vertical. Our newly
designed layout can be seen in \autoref{SearchVert}.

\figx{SearchVert}{New vertical design for the
\textc{activity\_picto\_admin\_main.xml} file, as presented in the Android
Studio designer.}

Finally, after designing the new layout, we needed to allow for the screen
orientation to change, as this is not allowed in the current version of
pictosearch. This was done by changing the \textc{android:screenOrientation}
variable in the \textc{AndroidManifest.xml} file from \textc{landscape} to
\textc{fullSensor}. Setting the variable to \textc{fullSensor} allows the tablet
to use its sensors to set the orientation of the screen. This can be seen in
\autoref{OrientAllow}.\nl

\begin{minipage}[H]{\linewidth}
\begin{lstlisting}[caption = Allowing changes to the screen orientation., label = OrientAllow] 
<activity
	android:name="dk.aau.cs.giraf.pictosearch.PictoAdminMain"
    ...
    android:screenOrientation="fullSensor" >
    ...
</activity>
\end{lstlisting}
\end{minipage}































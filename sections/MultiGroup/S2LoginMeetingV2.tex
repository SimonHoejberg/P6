\chapter{GUI Requirements Analysis}\label{sec:Colab2}
Durint the initial assignment of project roles, group SW612 was chosen as the
``Product Owner'', which is the group responsible for managing system
requirements, and evaluating user feedback and requests.\nl

Based on SW612's responsibility of requirement management, this chapter will be
used to document our collaboration with the groups SW610 and SW612 to analyze
and manage system requirements, of which the purpose is to support us as we
continually work on developing the looks and features of the GIRAF applications,
in order to best satisfy the requirements of the GIRAF user base.\nl

At the sprint meeting before sprint 2, SW612 announced that they had scheduled
the first of many meetings with GIRAF users. Therefore, we, together with SW610,
were tasked with defining a number of questions, which would aim to determine
what elements of the current GIRAF implementation the users liked, and what
elements should subject for change. Describing these initial questions were 
defined as the task \textbf{T709}:\nl

\textit{\textbf{T709:} Define UI questions before the customer meeting on the
22'nd of March}.\nl 

The goal of this task is to collaborate with group SW612 and determine what
elements of the launcher and weekplanner applications they should discuss with
the users. Together, we determined that the purpose of this initial meeting
would be to determine if the users are satisfied with the overall design of the
login screen and the weekplanner, as we (the project groups) personally believe
that the weekplanner is unnecessarily difficult to use, and the login screen
makes use of QR-codes, which at first glance seems woefully impractical. The
QR-code system is illustrated in
\autoref{LauncherReview}.\nl

\section{User Feedback}
Following our meeting, SW612 met with the customers to present GIRAF in its
current iteration, and allow the users to interact with the launcher and
weekplanner.
The goal was to see how the users interacted with GIRAF, hear their feedback and
finally observe what frustrations they encountered. This resulted in SW612
evaluating their responses, and defining a series of user stories which have
been divided into 4 categories, General, Launcher, Login System and Weekplanner.
The entirety of the user stories can be seen in
\autoref{EgebakkenUserStories2203} in the appendix.\nl

The stories which we consider most relevant for us can be seen in
\autoref{UserStoriesGeneral}, \autoref{UserStoriesLauncher}, and
\autoref{UserStoriesLogin} in the following subsections. In addition, following
each set of user stories, we discussed with SW612 in order to determine what
requirements could be defined from the users feedback.

\subsection{User Feedback: Overall System}
For the GIRAF system as a whole we received the following user stories:
\begin{table}[H]
\begin{tabular}{|c|p{12.5cm}|}
\hline 
\multicolumn{2}{|c|}{General}\\\hline
1. & As a user of the system, I would like the system to move back to the
nearest stable state when the app crashes.\\\hline
2. & As a guardian I would like the option to toggle between grayscale and fully
colored in the system for a citizen. \\ \hline
3. & As a user of the system I would like the option to toggle between grayscale
and fully colored in the system. \\\hline
4. & As a guardian I would like a video I can use for more advanced tasks such
as creating schedules etc.\\\hline
\end{tabular}
\caption{Stories from the General category}
\label{UserStoriesGeneral}
\end{table}

Based on the stories from the general category, we are able to determine that
the users would like to have a stable application environment, such that a
citizen does not get confused if the application crashes. In addition the users
also want an option to change the GIRAF applications to use a grayscale colour
scheme, in order to reduce the visual stimula that is associated with colors.
The 4'th story is not considered a requirement yet, as it is hoped that this may
solved by simplifing some of the advanced aspects. As such we determine that
GIRAF has the following additional requirements:
\begin{enumerate}
  \item If the system crashes it needs to return to a stable state or return the
  user to the Launcher.
  \item There needs to be an option for grayscale, both for the entire system
  and on a per user basis.
\end{enumerate}

\subsection{User Feedback: Launcher}
For the Launcher we received the following user stories:

\begin{table}[H]
\begin{tabular}{|c|p{12.5cm}|}
\hline 
\multicolumn{2}{|c|}{Launcher}\\ \hline 
1. & As a guardian I would like the login button to let a citizen tap to log in
as a citizen, and hold to login as a guardian, only providing login information
at that point.\\\hline
2. & As a guardian I would like to be able to push a template setup onto a group
of citizens for the launcher.\\\hline
3. & As a guardian I would like the option of adding an app to citizens both as
groups, individually or marking a list of citizens.\\\hline
\end{tabular}
\caption{User stories for the Launcher}
\label{UserStoriesLauncher}
\end{table}

Based on these stories we can see that one of the main concerns for the
guardians is to make it easier for them to manage multiple citizens, whether its
changing graphical option or changing which applications are visible. From this
we determine that the Launcher has the following additional requirements:

\begin{enumerate}
  \item A guardian should be able to change the settings for multiple
  citizens at the same time.
  \item A guardian should be able have a template of settings which can be used
  on citizens.
\end{enumerate}
The 1'st story is not considered a requirement for the Launcher as the purpose
of this story is to allow for citizens and guardians to login easily and is
going to depend on how the log in system is set up.\nl

\subsection{User Feedback : Login System}
For the Login System, which can be counted as a part of the Launcher, we
received the following stories:

\begin{table}[H]
\begin{tabular}{|c|p{12.5cm}|}
\hline 
\multicolumn{2}{|c|}{Login System}\\ \hline 
1. & As a guardian I would like not have a Quick-Response (QR) code for login,
but a regular password instead.\\ \hline 
2. & As a guardian I would like the ability to set a password for specific users
of the system.\\ \hline
3. & As a guardian I would like the system to automatically log me out of my
account after a set time limit (approx. 20 min).\\ \hline
4. & As a citizen I would like the system not to log me out automatically.\\ \hline
5. & As guardian I would like the option to make a specific device always login
into a citizen's account without password.\\ \hline
6. & As an institute, we would like to be able to log in as the institute as a
whole.\\\hline
\end{tabular}
\caption{User stories for the Login System}
\label{UserStoriesLogin}
\end{table}
 
Based on these stories we are able to determine that the users dislike the
current QR system and have made several considered to how they would like login
system to work. As such we can determine that the login system has the following
requirements:

 \begin{enumerate}
   \item The QR-system needs to be replaced by password based system.
   \item There needs to be limits on how long a guardian can remain inactive in
   the system before being logged out. 
   \item Citizen should not be logged out automatically. 
   \item A guardian should be able to change its password.
   \item There should be an institute-type user.
 \end{enumerate}% Hvad er resultatet

Following the initial meeting with the users, group SW612 has held an additional
2 meetings, in which they have continued to recieve user feedback, and attempted
to translate in into system requirements. All user stories can be seen in the
appendix in \autoref{ImportantUS}, where they are ordered by importance to the
work conducted this semester.














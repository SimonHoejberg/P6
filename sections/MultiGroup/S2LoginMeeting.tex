\chapter{Login System}\label{sec:LoginColab}
% This chapter will be used to describe the discussion that went into 
% the different collaborative elements
% that went into designing the login system. 
This chapter is part of the sprint 2 and is a collaboration between three
groups SW609, SW610 and SW612. The goal is to update the requirements for the
launcher and weekplanner applications and see if new requirements are needed.
As such this chapter is used to describe the process of analysing the
requirements each the applications and present the newly found requirements for
the launcer. We will finally discuss the resulting mock-up prototype.

\section{Requirements Analysis}


The first meeting we hold, consisting of SW609, SW610 and SW612,  to produce a
series of questions concerning how the user views the current state of the
launcher and weekplanner.


 The first meeting is with SW609, SW610 and SW612 and
is used to produce a series of questions concerning how the user views the current state of the
launcher and weekplanner. The questions ranged from specific questions about the
UI to general questions about added functionality.


A second meeting is held by SW612 with the customers from Egebakken on the 22.
of March. The goal of the meeting is see how the customers interact with GIRAF
and observe what frustrations they encounter. 


not only to receive answers to the
questions but also to 

 where the current product was demonstrated to the customers



Group SW612 


\section{Login}
This section will be used to describe the different phases of designing the new
login screen.

This collaboration is between us group (SW609) and groups SW610 and SW612. The
collaboration started with a meeting were we could ask some questions about the
user interface and what the user expects from the GIRAF Launcher and
Weekplanner. \\
Group SW612 is in the collaboration since they are the Product Owner. This means
that they are responsible for the communication between us and the users. Thus
after having had the meeting with the two other groups, group SW612 had a
meeting with some of the users. On this meeting they tried getting some answers
on some of the questions we had. This ended up with them creating some user
stories see \autoref{UserStoriesLogin}. We use the user stories to figure out
the functionality and looks of the new login screen. 

\begin{table}[H]
\begin{tabular}{|c|p{12.5cm}|}
\hline 
\multicolumn{2}{|c|}{Launcher}\\
\hline
1. & As a guardian I would like the login button to let a citizen tap to log in
as a citizen, and hold to login as a guardian, only providing login information
at that point. \\ \hline
\multicolumn{2}{|c|}{Login}\\ \hline
1. & As a guardian I would like not have a Quick-Response (QR) code for login,
but a regular password instead.\\ \hline
2. & As a guardian I would like the ability to set a password for specific users
of the system.\\ \hline
3. & As a guardian I would like the system to automatically log me out of my
account after a set time limit (approx. 20 min).\\ \hline
4. & As a citizen I would like the system not to log me out automatically.\\ \hline
5. & As guardian I would like the option to make a specific device always login
into a citizen's account without password.\\ \hline
\end{tabular}
\caption{User stories for the new login screen}
\label{UserStoriesLogin}
\end{table}

\subsection{Design}

On previous semester some design guide lines has been created these can be
found in, \fix{see,}{Reference to the design manual}. We then use the design
guidelines to design the login screen, see \autoref{LoginScreen}. This mock up
will act as a reference for when we need to program the actual login system,
this will also work as prototype such that other groups can validate this
design. 
\figx{LoginScreen}{Mock up for the login screen}

We add the following buttons on login since we need to allow for the
following methods for reaching the login.

\figx{LoginDiagram}{Diagram for the Login procedure}

Explain reasoning.

\subsection{Programming}





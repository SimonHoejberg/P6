\chapter{Login System}\label{sec:LoginColab}
% This chapter will be used to describe the discussion that went into 
% the different collaborative elements
% that went into designing the login system. 
This chapter is part of the sprint 2 and documents the collaboration 
between three groups SW609, SW610 and SW612. The goal is to update the
requirements for the launcher and weekplanner applications and see if new
requirements are needed. As such this chapter describes the
process of analysing the requirements for each the applications and presents
the newly found requirements for the launcher. We will finally discuss the
resulting mock-up prototype.

\section{?Requirements Analysis?}

\fix{}{Question if PO has received new results}

% Beskriv tasken
% Hvorfor er tasken relevant at samarbejde paa
The collaboration is centered around \textit{T709: Define UI questions
before the customer meeting on the 22'nd of March}. The goal of this task is to
define a series of questions concerning the UI and its functionality for the
launcher and weekplanner applications. The task itself is necessary as the
weekplanner is difficult to use, while the launcher's login screen uses a
QR-code. The task is collaborative as it requires SW609 and SW610 to produce
questions concerning the various aspects of the two applications they have
encountered, while SW612 acts as the intermediary between us and the various
institutions. The meeting which \textit{T709} refers to is a meeting with
Egebakken which provides education opportunities for citizens.\nl

% Hvordna blev den loest
The task was resolved during a meeting consisting of SW609, SW610 and SW612.
The questions were described during a walkthrough of the launcher and
weekplanner. The questions regarding the launcher were centered around its
login screen and how the citizen and guardian were supposed to interact. The
questions for the weekplanner were centered around how difficult it currently
is to use.\nl

% Hvad er resultatet
Following this meeting SW612 met with the customers and displayed GIRAF and
allowed the users to interact before they asked the questions. The goal was
to see how the users interacted with GIRAF, hear their feedback and finally
observe what frustrations they encountered. This resulted in a series of stories
from the users, as can be seen in \autoref{EgebakkenUserStories2203}. 




A second meeting is held by SW612 with the customers from Egebakken on the 22.
of March. The goal of the meeting is see how the customers interact with GIRAF
and observe what frustrations they encounter. 


% Hvad kan vi f� ud af resultatet









Through the meeting with customers SW612 produced a series of user
stories, see \fix{}{link til appendix}. 

The stories cover both general issues
and 

The stories are 

\begin{table}[H]
\begin{tabular}{|c|p{12.5cm}|}
\hline 
\multicolumn{2}{|c|}{Launcher}\\
\hline
1. & As a guardian I would like the login button to let a citizen tap to log in
as a citizen, and hold to login as a guardian, only providing login information
at that point. \\ \hline
\multicolumn{2}{|c|}{Login}\\ \hline
1. & As a guardian I would like not have a Quick-Response (QR) code for login,
but a regular password instead.\\ \hline
2. & As a guardian I would like the ability to set a password for specific users
of the system.\\ \hline
3. & As a guardian I would like the system to automatically log me out of my
account after a set time limit (approx. 20 min).\\ \hline
4. & As a citizen I would like the system not to log me out automatically.\\ \hline
5. & As guardian I would like the option to make a specific device always login
into a citizen's account without password.\\ \hline
\end{tabular}
\caption{User stories for the new login screen}
\label{UserStoriesLogin}
\end{table}

<<<<<<< .mine
Using these stories we are able to formulate a series of informal requirements
for the launcher and login.

\begin{table}[H]
\begin{tabular}{|c|p{12.5cm}|}
\hline
1 & Replace QRcode-system with a regular password-system. \\ \hline
2 & Do not log citizens out. \\ \hline
3 & Log guardians out after 20 minutes of inactivity. \\ \hline
4 & Allow the user to change password. \\ \hline
\end{tabular}
\caption{The additional new informal requirements for the launcher}
\label{LoginRequirements}
\end{table}

The remaining stories are\ldots



% \subsection{Design}
% 
% On previous semester some design guide lines has been created these can be
% found in, \fix{see,}{Reference to the design manual}. We then use the design
% guidelines to design the login screen, see \autoref{LoginScreen}. This mock up
% will act as a reference for when we need to program the actual login system,
% this will also work as prototype such that other groups can validate this
% design. 
% \figx{LoginScreen}{Mock up for the login screen}
% 
% We add the following buttons on login since we need to allow for the
% following methods for reaching the login.
% 
% \figx{LoginDiagram}{Diagram for the Login procedure}
% 
% Explain reasoning.
\inputS{MultiGroup/tempDesign}





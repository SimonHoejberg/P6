\chapter{GUI Requirements}\label{sec:Colab2}
% This chapter will be used to describe the discussion that went into 
% the different collaborative elements
% that went into designing the login system.
This chapter is part of sprint 2 and documents the collaboration between the
three groups SW609, SW610 and SW612. The goal is to update the GUI requirements
for the launcher and weekplanner applications, and to determine if new
requirements are needed. As such, this chapter describes the collaborative
process and presents and analyses the new user stories for additional system
requirements.

\section{Collaboration Process}
During the sprint meeting before sprint 2, group SW612 announced that they had
scheduled a meeting with a group of users. Therefore, the application groups
SW609 and SW610 were tasked with defining a number of questions to present to
the users. As such, the collaboration is centered around task \textbf{T709}:\nl

\textit{\textbf{T709:} Define UI questions before the customer meeting on the
22'nd of March}.\nl 

The goal of this task is to define a series of questions concerning
the UI and its functionality for the launcher and weekplanner applications. The task itself is necessary as the
weekplanner is difficult to use, while the launcher's login screen uses 
impractical QR-codes. The task is collaborative as it requires SW609 and SW610
to produce questions concerning the various aspects of the two applications they have
encountered, while SW612 acts as the intermediary between us and the various
institutions. The meeting which \textbf{T709} refers to is a meeting with an
institution named Egebakken, which provides educational opportunities for
citizens with autism.\nl

% Hvordna blev den loest
The task was resolved during a meeting consisting of SW609, SW610 and SW612.
The questions were described during a walkthrough of the launcher and
weekplanner. The questions regarding the launcher were centered around its
login screen and how the citizen and guardian were supposed to interact. The
questions for the weekplanner were centered around how difficult it currently
is to use.\nl

% Hvad er resultatet
Following this meeting SW612 met with the customers and displayed GIRAF and
allowed the users to interact with the launcher and weekplanner. The goal was to
see how the users interacted with GIRAF, hear their feedback and finally
observe what frustrations they encountered. This resulted in a series of user
stories which have been divided into 4 categories, General, Launcher, Login
System and Weekplanner, as can be seen in \autoref{EgebakkenUserStories2203}.

\section{Requirement Analysis}
The stories which we consider most relevant to us can be seen in
\autoref{UserStoriesGeneral}, \autoref{UserStoriesLauncher} and
\autoref{UserStoriesLogin}. Using these stories we are able to determine some
requirements which likely will become tasks in later sprints.\nl

For the GIRAF system as a whole we received the following user stories:
\begin{table}[H]
\begin{tabular}{|c|p{12.5cm}|}
\hline 
\multicolumn{2}{|c|}{General}\\\hline
1. & As a user of the system, I would like the system to move back to the
nearest stable state when the app crashes.\\\hline
2. & As a guardian I would like the option to toggle between grayscale and fully
colored in the system for a citizen. \\ \hline
3. & As a user of the system I would like the option to toggle between grayscale
and fully colored in the system. \\\hline
4. & As a guardian I would like a video I can use for more advanced tasks such
as creating schedules etc.\\\hline
\end{tabular}
\caption{Stories from the General category}
\label{UserStoriesGeneral}
\end{table}

Based on these stories from the general category we are able to determine that
user would like to have a stable application environment, such that A
citizen does not get confused if the application crashes. In addition the user
also wants an option for grayscale to reduce the visual stimula that is
associated with colors. The 4'th story is not considered a requirement yet, as
it is hoped that this may solved by simplifing some of the advanced aspects. As
such we determine that GIRAF has the following additional requirements:
\begin{enumerate}
  \item If the system crashes it needs to return to a stable state or return the
  user to the Launcher.
  \item There needs to be an option for grayscale, both for the entire system
  and on a per user basis.
\end{enumerate}

For the Launcher we received the following user stories:

\begin{table}[H]
\begin{tabular}{|c|p{12.5cm}|}
\hline 
\multicolumn{2}{|c|}{Launcher}\\ \hline 
1. & As a guardian I would like the login button to let a citizen tap to log in
as a citizen, and hold to login as a guardian, only providing login information
at that point.\\\hline
2. & As a guardian I would like to be able to push a template setup onto a group
of citizens for the launcher.\\\hline
3. & As a guardian I would like the option of adding an app to citizens both as
groups, individually or marking a list of citizens.\\\hline
\end{tabular}
\caption{User stories for the Launcher}
\label{UserStoriesLauncher}
\end{table}

Based on these stories we can see that one of the main concerns for the
guardians is to make it easier for them to manage multiple citizens, whether its
changing graphical option or changing which applications are visible. From this
we determine that the Launcher has the following additional requirements:

\begin{enumerate}
  \item A guardian should be able to change the settings for multiple
  citizens at the same time.
  \item A guardian should be able have a template of settings which can be used
  on citizens.
\end{enumerate}
The 1'st story is not considered a requirement for the Launcher as the purpose
of this story is to allow for citizens and guardians to login easily and is
going to depend on how the log in system is set up.\nl

For the Login System, which can be counted as a part of the Launcher, we
received the following stories:

\begin{table}[H]
\begin{tabular}{|c|p{12.5cm}|}
\hline 
\multicolumn{2}{|c|}{Login System}\\ \hline 
1. & As a guardian I would like not have a Quick-Response (QR) code for login,
but a regular password instead.\\ \hline 
2. & As a guardian I would like the ability to set a password for specific users
of the system.\\ \hline
3. & As a guardian I would like the system to automatically log me out of my
account after a set time limit (approx. 20 min).\\ \hline
4. & As a citizen I would like the system not to log me out automatically.\\ \hline
5. & As guardian I would like the option to make a specific device always login
into a citizen's account without password.\\ \hline
6. & As an institute, we would like to be able to log in as the institute as a
whole.\\\hline
\end{tabular}
\caption{User stories for the Login System}
\label{UserStoriesLogin}
\end{table}
 
Based on these stories we are able to determine that the users dislike the
current QR system and have made several considered to how they would like login
system to work. As such we can determine that the login system has the following
requirements:

 \begin{enumerate}
   \item The QR-system needs to be replaced by password based system.
   \item There needs to be limits on how long a guardian can remain inactive in
   the system before being logged out. 
   \item Citizen should not be logged out automatically. 
   \item A guardian should be able to change its password.
   \item There should be an institute-type user.
 \end{enumerate}







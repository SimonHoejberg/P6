\chapter{Resource Requirements}\label{S1CS}
During the initial meetings between the project groups, it was found that the
existing client/database communications system was incomplete and not of a high
enough quality for us to dedicate development time in order to attempt to
improve it. As such, it was decided, that the groups responsible for the
database (SW613 and SW615) would be tasked to develop a new system, which should
make use of the principles described in the REST model (Representational State
Transfer).\nl

As a part of developing this new system, the REST groups (SW613 and SW615)
requested a meeting the groups responsible for the GIRAF applications (SW609 and
SW610), in order to determine the database resource requirements for the
individual applications. Considering the incomplete state of the previous
system, this information was needed in order to model the server and client
libraries such that they would be able to transfer all necessary data to the
applications.\nl

\section{Requirement Analysis}
As our group (SW609) is responsible for the \textc{Launcher} application and the
\textc{pictosearch-lib} library, we were tasked with analyzing the relevant
code before the scheduled meeting, in order to determine what database
resources would be necessary for these applications/libraries. These tasks are
presented in \autoref{TaS1} as T636, T646, and T647.\nl

In our case, we mostly determined the requirements testing the \textc{Launcher}
and \textc{pictosearch-lib} through general use. Based on this testing, and 
analysis of the code, we determined the following requirements:\nl

In order to use the \textc{Launcher}, a user needs to login to an
\textc{account}, which is either a \textc{Guardian} or a \textc{Citizen}. Each account has a
\textc{name}, a \textc{picture}, an a \textc{unique identifier}. In order to use
\textc{pictograms} in applications, the system needs a to acces pictogram
objects by ID.
This creates the following resource requirements:

\begin{itemize}
  \item User (Guardian or Citizen)
  	\begin{itemize}
  		\item Name
  		\item UserID
  		\item Picture (Icon) 
	\end{itemize}
  \item Pictogram
  	\begin{itemize}
  		\item ID
	\end{itemize}
\end{itemize}

Based on a discussion with group SW610 (who is working on the
\textc{Weekplanner} application), we created the class-diagram which can be seen
in \autoref{class-diagram}. In this diagram, \textc{person} (user) objects are
either \textc{Guardians} or \textc{Citizens}. Citizens have a weekly schedule,
which is used in the Weekplanner application. The schedule covers a number of
days which each have a number of pictograms, which are contained in Frame
data objects.

\figx{class-diagram}{Classes and the information necessary for their use.}

\section{Further Requirement Refinement}
Based on the initial requirement analysis and the respective meeting, the
database groups (SW613 and SW615) began designing the client/database interface.
Towards the end of the first sprint, group SW612 held a meeting with the
system's users, who defined a number of new requirements for the GIRAF
applications. Based on the user's feedback, the database requirements that are
relevant for our groups were further defined.

\begin{itemize}
  \item We want to be able to change users pictures by uploading new ones.
  \item Pictograms should be retriewed in batches to decrease wait time.
  \item Pictograms should be searchable by category.
  \item In the Launcher, we only need direct access to guardians, as citizens
  are aggregated from them.
  \item We need a server timestamp in order to limit login sessions.
  \item When accessing users, we need a sub-method to only retrieve ID, name and
  citizens, as we don't need all information.
\end{itemize}

Based on these requirements, the database groups where able to further define
the design of the client/database interface.

\section{Final Client/Database Interface Design}
Based on the meetings and discussion between our groups, the database groups
(SW613 and SW615) formulated the following design for the interface, which
should be used by the GIRAF applications on order to retrieve data from the
database. This design can be seen in \autoref{interface-uml}.s

\figx{interface-uml}{UML diagram detailing the client/database interface.}

In order to describe the the UML diagram, group SW613 supplied us with the
following listing, which contains explanations for each of the diagram's
elements.

\begin{itemize}
    \item \textbf{createSchedule}\\
    The \texttt{createSchedule} function is called when a new schedule has been created on a device and the schedule should be synchronised across other devices connected to the same department.
    \item \textbf{createImage}\\
    This functions is called when a user takes a photograph and wishes to share it across devices.
    \item \textbf{readGuardianByDepartment}\\
    The launcher application needs to read all guardian-objects from the current department. This function returns all guardians associated with the relevant department.
    \item \textbf{readScheduleByCitizen}\\
    This function returns all schedules associated with a given citizen.
    \item \textbf{readPictogramByName}\\
    When searching for pictograms, this function is used for getting pictograms by their name.
    \item \textbf{readPictogramByQuery}\\
    This function allows query passing to the database, allowing the application to search for pictograms with certain tags.
    \item \textbf{updateGuardianImage}\\
    This function and the following overloads modify the attributes of the given guardian object.
    \begin{itemize}
        \item \textbf{updateGuardianName}
        \item \textbf{updateGuardianPhone}
        \item \textbf{updateGuardianEmail}
    \end{itemize}
    \item \textbf{updateCitizenName}\\
    This function and the following overloads modify the attributes of the given citizen object.
    \begin{itemize}
        \item \textbf{updateCitizenImage}
        \item \textbf{updateCitizenPhone}
        \item \textbf{updateCitizenEmail}
    \end{itemize}
    \item \textbf{updateScheduleAttribute}\\
    This function and the following functions modify the given schedule object and sub-objects.
    \begin{itemize}
        \item \textbf{updateScheduleDay}
        \item \textbf{updateScheduleFrame}
        \item \textbf{updateScheduleChoice}
    \end{itemize}
    \item \textbf{deleteSchedule}\\
    This function deletes a schedule object.
    \item \textbf{deleteSchedules} (batch)\\
    This function deletes a number of schedule object.
\end{itemize}

For now, we are using this design and the defined methods in order to implement
database communication in the Launcher application, and the pictosearch-lib
library.

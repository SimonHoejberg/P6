\chapter{REST Architectural Design}\label{S1CS}
This chapter covers a cooperative effort to make GIRAF use \textc{REST}, this
chapter covers several meetings held throughout the semester.\nl

The first meetings were held at the start of sprint and by SW609, SW610, SW613,
SW615. During these initial meetings it was found that the existing
client/database communications system was incomplete and not of high enough
quality for us to dedicate development time to improve. The first of two main
issues were, that the existing system did not implement the required safety
measures which are expected of applications which make use of private
information. The second issue was, that the system did not allow for uploading
data to the server, but only to retrieve it, which rendered us unable to update
information on the server.\nl

Based on these issues, it was decided, that the groups responsible for the
database (SW613 and SW615) would be tasked to develop a new system, which should
make use of the principles described in the REST model (Representational State
Transfer)\citep{RESTInfo}. This model outlines a set of requirements which aim
to establish a high quality framework for server-to-client communication. These
requirements contain, but are not limited to, the following:

\begin{itemize}
  \item Client-Server: Seperate the user interface from data storage and
  manipulation.
  \item Stateless: Each request to the server returns a response which contain
  all information required for the client to service the request.
  \item Cacheable: Responses from the server must implicitly or explicitly
  declare themself as either cacheable or not, in order to prevent the client
  from reusing expired data.
  \item Layered System: The client is incapable to determine if it is connected
  to the main server or an affiliated intermediate entry point.
  \item Uniform Interface: The retrieved data is conceptually different from the
  representation used on the server. Given a set of data, the client should have
  enough information to edit and delete the respective information of the
  server. 
\end{itemize}

As a part of developing this new system, another meeting was held in order to
further determine the database resource requirements for the applications.
Considering the incomplete state of the previous system, this information was
needed in order to model the server and client libraries such that they would
be able to transfer all necessary data to the applications.\nl

\section{Requirement Analysis}
As our group (SW609) is responsible for the \texttt{Launcher} application and
the \\\texttt{pictosearch-lib} library, we were tasked with analyzing the
relevant code before the scheduled meeting, in order to determine what database
resources would be necessary for these applications/libraries. These tasks are
presented in \autoref{TaS1} as T636, T646, and T647.\nl

In our case, we mostly determined the requirements testing the \textc{Launcher}
and\\ \textc{pictosearch-lib} through general use. Based on this testing, and
analysis of the code, we determined the following requirements:\nl

In order to use the \textc{Launcher}, a user needs to login to an
\textc{account}, which is either a \textc{Guardian} or a \textc{Citizen}. Each
account has a \textc{name}, a \textc{picture}, an a \textc{unique identifier}.
In order to use \textc{pictograms} in applications, the system needs to access
pictogram objects by ID. As such, users require the following fields on the
server: \textbf{Name}, \textbf{UserID}, \textbf{Icon}, \textbf{Department},
\textbf{Username}, \textbf{Password}, \textbf{Screen Name}, and \textbf{Saved
Settings}.
Pictograms require the following fields on the database: \textbf{ID},
\textbf{Title}, and a \textbf{Bitmap (Picture)}.\nl

Based on a discussion with the other groups, we created a UML class-diagram,
which has been subject to redefinitions throughout the project. The final
version of this diagram can be seen in \autoref{Giraf_RestModel}.
\fix{}{Forklar hvad de forskellige pile betyder}

\figx[0.65]{Giraf_RestModel}{UML class-diagram depicting the objects requested
from the database}

In the UML diagram shown in \autoref{Giraf_RestModel}, each class has a
specific use and responsibility. This is described below in
\autoref{restModelTableEh}.

\begin{table}[H] 
\begin{centering}
\begin{tabular}{|l|p{9cm}|l|}
\hline 
Class 					&	Description	\\ \hline
GirafObject				&	Used for inheritance for most other objects. Implements the
Serializable interface in order to allow it to be passed between applications. \\\hline 
User					&	Contains data about a single user and its planned
schedule and activities.\\\hline 
Department  			&	Used to differentiate users of
different institutions, which allows us to implement institution-specific information.\\\hline 
Settings    			&	Contains information about a users local settings. This allows
the same settings to be mirrored across multiple devices. \\\hline 
Permission  			&	Used to specify the permissions of the user e.g. Admin,
Guardian, and Citizen. Allows and restricts access to certain features and data.\\\hline 
Frame              		&   Acts as a container class for pictograms. Contains all
relevant contextual data, and a Pictogram object.\\\hline 
Choice              	&	Acts as a container class for pictograms like a Frame,
but contains multiple pictograms, in the case that the citizen has a choice between
multiple activities in their schedules.\\\hline 
Pictogram              	&	Contains a Bitmap image of the Pictogram. \\\hline
Sequence              	&	List of Frames or Choices which together represent a
users daily schedule. \\\hline 
Weekday              	&	A Weekday is used to contain Frames for the week
schedule.                  \\ \hline 
WeekdayActivity			& A relation between Weekdays and Frames such that additional
properties can be added (e.g Index and Progress).\\\hline 
WeekSchedule         	&	A set of weekdays, which together represents a week's
schedule.\\\hline 
WeekdayActivityProgress & 	Contains information about a users progress on a
given schedule.\\\hline 
Application    			&	Information about applications a user is allowed to access.\\\hline     
\end{tabular}
\caption{Classes used in the REST model, and their descriptions}\label{restModelTableEh}
\end{centering}
\end{table}

\section{Client/Database Interface Design}
Following the initial requirement analysis and the respective meetings, the
database groups (SW613 and SW615) began designing the client/database interface.
Towards the end of the first sprint, group SW612 held a meeting with the
system's users, who defined a number of new requirements for the GIRAF
applications. Based on the user's feedback, the database requirements that are
relevant for our groups were further defined.

\begin{itemize}
  \item We want to be able to change users' pictures by uploading new ones.
  \item Pictograms should be retrieved in batches to decrease wait time.
  \item Pictograms should be searchable by category.
  \item In the Launcher, we only need direct access to guardians, as citizens
  are aggregated from them.
  \item We need a server timestamp in order to limit login sessions.
  \item When accessing users, we need a sub-method to only retrieve ID, name and
  citizens, as we don't need all information.
\end{itemize}

Based on these requirements, the database groups were able to further define
the design of the client/database interface.\nl

Based on the meetings and discussion between our groups, the database groups
(SW613 and SW615) formulated the following design for the interface, which
should be used by the \textc{rest-models} library in order to retrieve and
manipulate data from the database. This design can be seen in
\autoref{interface-uml}. In the diagram the hardline arrows means inheritance
and the dashed arrows means interface implementation.

\figx{interface-uml}{UML diagram detailing the client/database interface.}

In order to describe the UML diagram, group SW613 supplied us with the
following listing, which contains explanations for each of the diagram's
elements. 

\fix{}{Er det funktioner, der er jo forskel på interfaces, metoder og
funktioner}
\begin{itemize}
    \item \textbf{createSchedule}\\
    The \texttt{createSchedule} function is called when a new schedule has been created on a device and the schedule should be synchronised across other devices connected to the same department.
    \item \textbf{createImage}\\
    This function is called when a user takes a photograph and wishes to share it across devices.
    \item \textbf{readGuardianByDepartment}\\
    The launcher application needs to read all guardian-objects from the current department. This function returns all guardians associated with the relevant department.
    \item \textbf{readScheduleByCitizen}\\
    This function returns all schedules associated with a given citizen.
    \item \textbf{readPictogramByName}\\
    When searching for pictograms, this function is used for getting pictograms by their name.
    \item \textbf{readPictogramByQuery}\\
    This function allows query passing to the database, allowing the application to search for pictograms with certain tags.
    \item \textbf{updateGuardianImage}\\
    This function and the following overloads modify the attributes of the given guardian object.
    \begin{itemize}
        \item \textbf{updateGuardianName}
        \item \textbf{updateGuardianPhone}
        \item \textbf{updateGuardianEmail}
    \end{itemize}
    \item \textbf{updateCitizenName}\\
    This function and the following overloads modify the attributes of the given citizen object.
    \begin{itemize}
        \item \textbf{updateCitizenImage}
        \item \textbf{updateCitizenPhone}
        \item \textbf{updateCitizenEmail}
    \end{itemize}
    \item \textbf{updateScheduleAttribute}\\
    This function and the following functions modify the given schedule object and sub-objects.
    \begin{itemize}
        \item \textbf{updateScheduleDay}
        \item \textbf{updateScheduleFrame}
        \item \textbf{updateScheduleChoice}
    \end{itemize}
    \item \textbf{deleteSchedule}\\
    This function deletes a schedule object.
    \item \textbf{deleteSchedules} (batch)\\
    This function deletes a number of schedule object.
\end{itemize}

Until the REST framework is finished, the overall design of this interface
will be used to identify what opportunities we have in the applications, when it
comes to manipulating data on the database.

\section{Using the Rest Libraries}\label{UsingRest}
Following the final implementation of the REST model, the database groups SW613
and SW615 have supplied our group with a new approach to requesting and updating
information on the database. This approach consists of four different
elements: \textc{login requests}, \textc{get requests}, \textc{put requests}
and the \textc{queue}. An example of a login request can be seen throughout the
code examples in \autoref{CSLoginRequest1}, \autoref{CSLoginRequest2}, and
\autoref{CSLoginRequest3}, where we have two nested requests, a
\textc{LoginRequest} and a \textc{GetUserRequest}.\nl

\begin{minipage}[H]{\linewidth}
\begin{lstlisting}[caption = Initial request the respective server response object, label = CSLoginRequest1]
queue = RequestQueueHandler.getInstance(gui.getApplicationContext()).getRequestQueue();

LoginRequest loginRequest = new LoginRequest(username, password, new Response.Listener<Integer>() 
{
	@Override
    public void onResponse(Integer statusCode) 
    {
    	//Creates a GetRequest for the user, which is then added to the queue within the loginRequest 
    	GetRequest<User> userGetRequest = new GetRequest<User>(username, User.class, new Response.Listener<User>() 
    	{ 
    		//Passes the userinfo to homeIntent 
    		@Override public void onResponse(User response) 
    		{
	            Intent homeIntent = new Intent(gui, HomeActivity.class);
	            
                homeIntent.putExtra(Constants.CURRENT_USER,response);
                homeIntent.setFlags(Intent.FLAG_ACTIVITY_NEW_TASK);
                gui.startActivity(homeIntent);
            }
        },
        ...
\end{lstlisting}
\end{minipage}

In the code example in \autoref{CSLoginRequest1} we start by creating a new
\textc{LoginRequest} on \textbf{line 3}, where we pass the users'
\textc{username}, \textc{password}, and the \textc{ResponseListener}, which is
an object used for server communication. Everything after \textbf{line 4} is
a part of the \textc{LoginRequests} nested inner class. On \textbf{line 6} we
have an \textc{onResponse} method, which is called whenever we recieve a
response from the server. Given a user from the server, on \textbf{lines 14-18},
we create a new intent and start a HomeActivity based on the user.\nl

\begin{minipage}[H]{\linewidth}
\begin{lstlisting}[caption = Using an errorListener to detect mishaps in the GetUserRequest, label = CSLoginRequest2]
        new Response.ErrorListener() 
        {
        	@Override
            public void onErrorResponse(VolleyError error) 
            {
            	if(error!=null && error.networkResponse !=null) 
            	{
            		//The user is for some reason unavailable
            	     gui.showDialogWithMessage(gui.getString(R.string.error_try_agian) + "" + error.networkResponse.statusCode);
                } else if(error!=null) {
                	Log.e("error",error.getMessage());
                }
            }
        });
        
     queue.add(userGetRequest);
     }
},
\end{lstlisting}
\end{minipage}

The code example in \autoref{CSLoginRequest2} follows directly after
\autoref{CSLoginRequest1}, and is used to check for errors in the
\textc{GetUserRequest}. On \textbf{line 1} we create an \textc{ErrorListener}
which will listen for error messages retrieved from the server request. The
following lines are a nested class. On \textbf{line 4} we have the
\textc{onErrorResponse} method which is called with a \textc{VolleyError}. On
lines {7-13} we choose an action based on the error-code contained in the error
object. On \textbf{line 16} we add the GetUserRequest (which we defined in
\autoref{CSLoginRequest1}) to the request queue.\nl

\begin{minipage}[H]{\linewidth}
\begin{lstlisting}[caption = Using an errorListener to detect mishaps in the LoginRequest, label = CSLoginRequest3] 
new Response.ErrorListener() 
{
	@Override
    public void onErrorResponse(VolleyError error) {
        int errorCode = error.networkResponse.statusCode;
        if (errorCode == 401) 
        {
        	//The username and/or password are incorrect
            gui.showDialogWithMessage(gui.getString(R.string.error_username_password));
        } else {
            //The server is for some reason unavailable
            gui.showDialogWithMessage(gui.getString(R.string.error_try_agian));
        }
    }
});

//The login request is added to the queue
queue.add(loginRequest);
\end{lstlisting}
\end{minipage}

The code example in \autoref{CSLoginRequest3} follows directly after
\autoref{CSLoginRequest2} and uses the same approach. and is used to detect
errors in the \textc{LoginRequest}. On \textbf{lines 6-13} we choose which error
message to display in the console, based on what error-code we retrieved from
the server.

\begin{minipage}[H]{\linewidth}
\begin{lstlisting}[caption = Using a RequestQueue to commicate with the server,
label = RequestQueue]
public void login(final String username, String password) {
    RequestQueue queue = RequestQueueHandler.getInstance(gui.getApplicationContext()).getRequestQueue();

    //Creates a login request which is then added to the queue later
    LoginRequest loginRequest = new LoginRequest(username, password,
        new Response.Listener<Integer>() {
            @Override
            public void onResponse(Integer statusCode) {
                //Creates a GetRequest for the user, which is then added to the queue within the loginRequest
                GetRequest<User> userGetRequest =
                    new GetRequest<User>(username, User.class, new Response.Listener<User>() {
                       	...
                    }
                    ...
                );
                queue.add(userGetRequest);
            }
        },
		...
    );
    //The login request is added to the queue
    queue.add(loginRequest);
}
\end{lstlisting}
\end{minipage}

The code example of \autoref{RequestQueue} is used to demonstrate how the
applications communicate with the server. The communication is done through a
\textc{RequestQueue} which we instantiate on \textbf{line 2}. The
\textc{RequestQueue} communicates asynchronously with the server, which means
that in order for our requests to be handled in the correct order we need to
nest our requests inside each as other, as seen on \textbf{line 16} and
\textbf{22}, where the \textc{userGetRequest} is nested inside the
\textc{loginRequest}'s \textc{onResponse}. This is because to get a user we need
a token / cookie and the way we get this, is using a \textc{LoginRequest} then
we can use the \textc{GetRquest} to get the user for that username.

\subsection{REST Framework Discussion}
\fix{}{DonF, har jeg ret?}\nl
\fix{}{Discuss the fact that we have to make requests at the start of a call-chain}\nl
During development of the client-to-server communications framework, the server
groups SW613 and SW615 thought that it would be possible to wrap the server
communication requests inside a helper class, which would allow us to simply
call a number of methods whenever we wanted to manipulate/request data. This
however, has appeared to be impossible to implement while still adhering to the
principle of REST. This is because REST requires server communication to be
asynchronous, such that UI threads are not put on hold while waiting for a
response. The problem occours, as methods themselves are synchronous by
definition, as control is only passed back to the calling class whenever the
method returns. The necessity for asynchronous communication has limited us to
using the approach shown in \autoref{CSLoginRequest1},
\autoref{CSLoginRequest2}, and \autoref{CSLoginRequest3} where we can use 50 or
more lines of code for a single request. The only solution we have found to this
issue would be to use lambda-expressions, as they allow for us to pass methods
as method parameters.
This would allow us to pass two methods to the helper class, which would be used
respectively on success and failure of the request. However, lambda-expressions
are not currently available for Android development, as the current version of
the Android SDK uses Java7, while lambda-expressions are only implemented from
Java8 and onwards.














\subsection{Design}

This subsection will be used to describe the design process used to create the
new login system. The design is based on a design manual created a previous
year. The purpose of the design manual is to create some design guidelines, such
that every application is recognizable as a GIRAF application. Therefore, the
design manual contains a set of rules of how for example the colour scheme which
is supposed to be used. Another example of the design guidelines is how the
structure of the application is to be made, for example the back button which
should always be placed as the left-most button at the top. Furthermore, the
back button should always be the same see \autoref{icon_back}.

\figx[0.3]{icon_back}{Back button}

Based on the design guidelines and the requirements found in
\autoref{LoginRequirements} we created a mock up for the new login system, see
\autoref{LoginScreen}.

\figx{LoginScreen}{Mock up for the login screen}

We then use the informal requirements to create a diagram of the login procedure
see \autoref{LoginDiagram}. We use this diagram to show how the
different users interact with the login system, for example if a guardian is inactive for
more than 20 minutes they will automatically be logged out and direacted to the
login screen.\\
It also shows how the interaction is between the guardian and the citizen, thus a
guardian can switch user to a citizen but not the other way round. Another thing
is that a citizen cannot go from inactive to logged out, which means that a citizens session will never expire.

\figx{LoginDiagram}{Diagram for the Login procedure}
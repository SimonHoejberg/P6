\chapter*{Summary\markboth{Summary}{Summary}}
In this report we present our work on the GIRAF multi-project which was started
in 2011. It is a collaborative software development project shared between
the Software bachelor students at Aalborg University. The GIRAF multi-project is focused on
developing a suite of applications for Android tablets, which are aimed at
supporting autistic children in planning and structuring their daily lives.\\
Due to the scale of the project, the development has spanned across multiple
years, with numerous project groups. As such, the project is focused on code
maintenance and developing upon a larger code base.\nl

The project development is shared between 7 groups of 4 persons each.
These groups are organized using the Scrum of Scrums organizational approach, in
which each group has a Scrum representative, which engages with other
representatives in order to organize the project direction at weekly Scrum
meetings. In addition to these weekly Scrum meetings, the project tasks are
planned at sprint meetings at the start of each sprint, in which all members of
the project participates.\nl

The project is split into four individual sprints, which each has a specified
set of tasks, and a defined sprint goal. The tasks range from implementing
specific features, researching possible solutions, planning meetings, writing
development guides, and refactoring old code. The tasks themselves are based on
user feedback, internal discussion, and the backlog from last year's
students.\nl

The main focus for this semester is to replace the server-to-client
communications framework, as it is of insufficient quality to be used for any
further development. This new framework should make use of the constraints
defined in the \textc{REST} (Representational State Transfer) model.\\
In the greater GIRAF project, our group's responsibility is to improve and
repair the GIRAF Launcher application, and the libraries it is
dependent upon. As such, we have to familiarize ourselves with the
\textc{Giraf-Component-Lib} library and the Launcher application to the
point where we are capable of replacing the current backend with the \textc{REST}
library developed by other project groups. Based on this responsibility, this
report is focused on the implementation of the new \textc{REST} library, and the
preperatory work needed in order to complete this process.\nl

In general, most of the semester was spent preparing for the finalization of the
\textc{REST} framework, such that it could be implemented in the \textc{Launcher} and
\textc{Component-Lib} library.\\
The first sprint is used to familiarize ourselves with the GIRAF code base by
refactoring the \textc{Launcher}, \textc{Component-Lib} and
\textc{PictoSearch-Lib}. It allowed us to increase the code quality, and to
make sure that the code used a consistent style. We also began fixing a number
of bugs, which were passed on to us through the backlog of the previous years
students.\nl



In the second sprint we worked on fixed some security issues, discussed design
decisions with other groups, and fixed more bugs. During the second sprint, our
build tools broke down, which prevented us from building our project. Due to
this problem, the second sprint was spent developing potential solutions to
problems defined by the GIRAF users. Additionally, we designed a new login
screen based on user feedback.\nl

For the third sprint, we were tasked with helping to restore our build tools,
and to implement a greyscale option for the GIRAF GUI. During the third sprint our build
tools became operational once again, and we could begin to implement the new
login screen and the greyscale option. Additionally, we began to remove the old
server-to-client communication in order to prepare for the new \textc{REST} framwork.\nl

The final sprint was focused on implementing a beta version of the \textc{REST}
framework into the \textc{Launcher} application and the component-lib library.
As this framework did not contain all necessary functionality, parts of the sprint was
also spent holding meetings with the other groups in order to determine the
final design of the \textc{REST} framework. This sprint ended with the Launcher in a
mostly finished state, and the component-lib libary in a finished state. The
functionality missing from the launcher is due to the partly unfinished state of
the \textc{REST} framework.\nl

In conclusion, we have manged to familiarize ourselves with a larger exisiting
code base. We have managed to implement a new login system, where the
design was based on user feedback. We have also implemented a global greyscale
functionality in the GIRAF GUI. We have also familarized ourselves with
our build tools (Artifactory, Gradle and Jenkins), and how to manage
these tools in a project setting. Finally, we have implemented the \textc{REST}
framework into the \textc{Launcher} and the {Component-Lib} library.

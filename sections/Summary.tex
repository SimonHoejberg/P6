\chapter*{Summary\markboth{Summary}{Summary}}
In this report, we present our work on the GIRAF multi-project, which is a
collaborative software development project, which is shared between Software
bachelor students at Aalborg University. The GIRAF multi-project is focused on
developing a suite of applications for Android tablets, which are aimed at
supporting autistic children in planning and structuring their daily lives.\\
Due to the scale of the project, the development has spanned across multiple
years, and numerous project groups. As such, the project is focused on code
maintenance and developing upon a larger code base.\nl

The project development is shared between 7 different groups of 4 persons each.
These groups are organized using the Scrum of Scrums organizational approach, in
which each group has a Scrum representative, which engages with other
representatives in order to organize the project direction at weekly Scrum
meetings. In addition to these weekly Scrum meetings, the project tasks are
planned at sprint meetings at the start of each sprint, in which all members of
the project participates.\nl

The project is split into four individual sprints, which each has a specified
set of tasks, and a defined sprint goal. The tasks range from implementing
specific features, researching possible solutions, planning meetings, writing
development guides, and refactoring old code. The tasks themselves are based on
user feedback, internal discussion, and the backlog from last year's
students.\nl

The main goal for this semester is to replace the server-to-client
communications framework, as it is of insufficient quality to be used for any
further development. This new framework should make use of the constraints
defined in the REST (Representational State Transfer) model. Due to this
requirement, we have chosen to simply call the new framework REST or REST
Model.\\
In the greater GIRAF project, our group's responsibility is to improve and
repair the GIRAF Launcher application, and the libraries it is dependent upon.
As such, we have to familiarize ourselves with the \textc{giraf-component-lib}
library and the Launcher application to the point where we are capable of
replacing the current backend with the REST library developed by other
project groups. Based on this responsibility, this report is focused on the
implementation of the new REST library, and the preperatory work needed in order
to complete this process.\nl

In general, most of the semester was spent preparing for the finalization of the
REST framework, such that we could begin to implement it into the Launcher and
component-lib library.\\
The first sprint is used to familiarize ourselves with the GIRAF code base. This
was done through fixing a number of bugs, which were passed on to us through the
backlog of the previous years students.\nl

The focus of the second sprint was to refactor the Launcher and the various
libraries. This was done in order to increase the code quality, and to make sure
that the code used a consistent style. Additionally, we fixed some security
issues, discussed design decisions with other groups, and fixed more bugs.
During the second sprint, our build tools broke down, which prevented us from
building our project. Due to this problem, the second sprint was spent
developing potential solutions to problems defined by the GIRAF users.
Additionally, we designed a new login screen based on user feedback.\nl

For the third sprint, we were tasked with helping to restore our build tools,
and to implement a greyscale option for the GIRAF GUI. During the third sprint our build
tools became operational once again, and we could begin to implement the new
login screen and the greyscale option. Additionally, we began to remove the old
server-to-client communication in order to prepare for the new REST framwork.\nl

The final sprint was focused on implementing a beta version of the REST
framework into the Launcher application and the component-lib library. As this
framework did not contain all necessary functionality, parts of the sprint was
also spent holding meetings with the other groups in order to determine the
final design of the REST framework. This sprint ended with the Launcher in a
mostly finished state, and the component-lib libary in a finished state. The
functionality missing from the launcher is due to the partly unfinished state of
the REST framework.\nl

In conclusion, during the project, we have manged to familiarize ourselves with
a larger exisiting code base, and managing to further implement based on it.
Additionally, we have managed to implement a new login system, which design was
based on user feedback. We have also implemented a global greyscale
functionality in the GIRAF GUI. We have also familarized ourselves with
our build tools (Artifactory, Gradle and Jenkins), and how to manage
these tools in a project setting. Finally, we have implemented the REST
framework into the Launcher and the component-lib library.




% The full project was split into 5 sprints: an initial mini–sprint to get familiar with the source code
% and the assigned responsibility areas, and 4 regular sprints. The spare time between the sprints
% were used for writing the report as well as preparing for the next sprint. For our responsibility
% area, testing, we have written a technical chapter, in which we describe three types of tests: unit,
% integration, and smoke tests. Additionally we have also written technical chapters on REST,
% and code reviewing, as we became the authority for these areas in the project. Code review was
% used because we pushed for this development style. We became the authority of REST since we
% were both the ones suggesting and developing the architecture.
% In the rest of the report we describe our work and thoughts in the different sprints, especially
% REST development and monkey testing appear often throughout the sprints.
% In the first sprint we spread our efforts on a series of tasks. Most of these were concerning bug
% removal. We also began the work on the REST API together with the new login system, which
% we continued to work on throughout all the sprints.
% From the second sprint major organizational changes took place, as to how tasks were distributed
% and which tasks were deemed most important. This was spurred on by us, since we suggested
% that an overall goal for the semester and the project should be defined, which we could structure
% our work around. From here on the distributed tasks where confined to a small collection of
% application, which were classified as the minimum viable product. We continued to work on the
% new login system and the REST API, as these were considered to be an important part of the
% minimum viable product.
% During the third sprint we mainly focused on the developing the REST API for the login system,
% and getting monkey testing to work. We only took in a single new task, as the API and the
% monkey testing were large tasks which consumed most of our time.
% At the beginning of the fourth sprint we became aware that we were behind on our report, as
% we had spent time between the sprints on the REST API and monkey testing. We therefore
% halted the development early on, such that we could catch up on our report as well as write
% documentation on our work for the next developers of Giraf project. A lot of our documentation
% can be seen in our appendix, and can also be found on the Phabricator webpage for the project.
% In the end of the report we evaluate our efforts in the areas which we were responsible for. Code
% review we believe was worth the initial difficulties of changing the developers normal workflow.
% Once the developers became comfortable with this way of developing we believe that the process
% became streamlined and bugs and errors was removed from the produced code. However we did
% not come up with a plan of evaluating the efficiency of code reviewing when we introduced it to
% the project, which meant that we could not conclude to which degree code review had helped.
% Testing turned out completely different than we thought, since we found that it was difficult
% to separate business logic from UI code in the applications. The initial plan was to produce
% unit tests for most of the project, but based on these difficulties this was later reduced to only
% producing unit tests for new code.
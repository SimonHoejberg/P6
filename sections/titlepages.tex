{\selectlanguage{english} \pdfbookmark[0]{Title page}{label:titlepage_en}
\aautitlepage{%
  \englishprojectinfo{
   \name%title
  }{%
    Multi Project
	%theme
  }{%
    Spring 2017%project period
  }{%
    SW609F17 % project group
  }{%
    % list of group members 
    Jonas Ibrahim\\
    Jonathan Magnussen\\
    Christoffer Mouritzen
    
  }{%
    % list of supervisors
	Ying Wang
  }{%
    \today % date of completion
  }%
}{%department and address
  \textbf{Computer Science}\\
  Aalborg Universitet\\
  \url{http://www.aau.dk}
}{% the abstract
This project is a collaborative development effort exercised by 7 groups of
bachelor students. Together we aim to continue development of the GIRAF suite
of applications for Android tablets, which aim to support autistic children. 
This is done by reworking the server-to-client communications framework (which
is based on the REST principles), and implementing features requested by GIRAF's
users. Our group is focused on fixing the GIRAF Launcher application,
implementing a new login system, implementing greyscale GUI functionality, and
refactoring the old code base. Following the project GIRAF is still unfinished.
The REST framework is almost entirely finsihed, but still lacks some necessary
requests. The Launcher is finished, but lacks some data-objects which are not
yet correctly represented on the server. The greyscale functionality is
finished, and the new login system is finished.


 }
 }

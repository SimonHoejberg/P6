{\selectlanguage{english} \pdfbookmark[0]{Title page}{label:titlepage_en}
\aautitlepage{%
  \englishprojectinfo{
   \name \\
   Collaboratively Developing and Maintaining Complex Systems for the
    Android Platform
  }{%
    Multi Project
	%theme
  }{%
    Spring 2017%project period
  }{%
    SW609F17 % project group
  }{%
    % list of group members 
    Jonas Alberg Ibrahim\\
    Jonathan Nygaard Magnussen\\
    Christoffer Donskov Mouritzen
    
  }{%
    % list of supervisors
	Ying Wang \& Mathias Ruggaard Pedersen 
  }{%
    \today % date of completion
  }%
}{%department and address
  \textbf{Computer Science}\\
  Aalborg Universitet\\
  \url{http://www.aau.dk}
}{% the abstract
This project is a collaborative development effort by 7 groups of bachelor
students. The projects' aim is to continue development of the GIRAF suite of
applications for Android tablets, which aim to support autistic citizens.
This is done by reworking the server-to-client communications framework (which
is based on the REST principles), and implementing features requested by GIRAF's
users. We focus on fixing the GIRAF Launcher application, implementing a new
login system, implementing grayscale GUI functionality, and refactoring the old
code base. The REST framework is almost finished, but still lacks some
necessary requests. The Launcher is finished, but lacks some data-objects which
are not yet correctly represented on the server. The grayscale functionality and
the new login system is finished.


 }
 }

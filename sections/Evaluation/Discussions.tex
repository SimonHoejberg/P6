\chapter{Discussion}
The purpose of this chapter is to discuss the GIRAF project as a whole, in order
to determine what went well, and what problems occurred during the project.

\section{REST}
When we originally began working on GIRAF, the server
architecture group (SW615) updated the firmware on the server and noticed that
the server would no longer operate at all. By individual research, and by
reading the previous groups documentation, they determined that the previous
groups had used a ``hack'' or loophole in order to get the server to run. A
hack, which was fixed in later versions of the server's firmware. Based on this
problem, the project groups determined to replace the old
server and data models. This choice to replace the entirety of
the server/database backend has led to the largest problem for developing GIRAF
in this semester.\nl

Since the server/database backend is essential to all parts of developing the
applications, we have been unable to commence the application development in
full before the beta version of the \rlib\ was finished. As the server groups
vastly underestimated the complexity and time required to research and
implement an entirely new server/database backend, this beta version of the
\rlib\ was first ready for limited use at the end of the third sprint.\nl

Additionally, many essential parts of the framework were changed multiple times
or was simply not in an acceptable state. This meant that most of the
development was done in the last two weeks of sprint 4. Most of our work
during the first three sprints were rendered invalid, as the REST framework
fundamentally changed how the respective classes functioned.

\section{Code Review}
After the conclusion of sprint 1 it was decided that the project groups should
use code reviews. It was given a channel on Discord which was dedicated to code
reviews. Guidelines for the process were designed by SW611, and can be seen
in \autoref{CRP}.\nl

The process was not used more than a couple of time throughout the semester,
mainly because it was a voluntary effort. It required people to choose
which parts of the code needed a review and then it required another
person to voluntarily review it. Besides this, the code review practice was not
advertised or mentioned much. We did not make use of code reviews, because
when we finally had code that could be reviewed, we were at the end of sprint 4.
Aditionally, while we were too busy during the fourth sprint to submit our code
for review, it would have been a good idea to use the process during the first
and second sprint, where our build tools were not operational.

\section{Sprint Meetings}
The sprint meetings took place at the end of each sprint. The goal of the
meetings was to give a brief presentation of what each group has worked on,
discuss the previous sprint and then create tasks.\nl

\subsection{Presentation}
The presentation phase remained unchanged throughout the semester and has in
hindsight not provided much use. The brief presentation from each group has not
had much meaning to the other groups, as there has been little understanding of
what it means. While there was important information, much of it was not in a
context that made it easily accessible for other groups. A better way could have
been to provide a presentation on the status of certain important parts and
explain slightly more detailed what is missing or why it is delayed.\nl

\subsection{Sprint Discussion}
The discussion of the previous sprint phase remained unchanged as well,
though this has not been a problem. This phase has provided a good way to
visualize the general progress of the project, and as the sprint is task based
has allowed us to find problems with some tasks.\nl

\subsection{Task Generation}
During the project, tasks were created by setting a specific goal for each
sprint, which often came in for form of user stories. At the meeting for the
first sprint, the project groups were divided into two main groups, which
focused on the server and applications respectively. This process was changed in
sprint 2, where we instead chose to create a number of groups consisting of a
representative from each group. The idea was to better generate tasks and see
how they might depend on tasks from other groups. This did not work as there was
not a lot of tasks whic involved both the server and applications.
This resulted in some group members not being able to participate in a
meaningful way. Due to these problems, it was changed back to the original
format in sprint 3. As a final note it could be considered beneficial for each
project group to prepare a list of tasks they want to work on, as this would
save a lot of time and would allow the individual tasks to be discussed more.

\section{General Communication}
For general communication during the sprints, we have this year used both weekly
SCRUM meetings, a Discord server, and talking face to face.

\subsection{SCRUM Meetings}
The SCRUM meetings have not changed during the project. This is fine as they
have worked well in providing a small overview of what other groups are working
on. This has been especially helpful to give a distinct impression of what each
of the server groups is working on. It has helped with sharing information in
regards to how the REST framework was going to influence the applications.
However, in hindsight, we have not been good at communicating when we had a lack
of work, as we could possibly have helped the server groups to finish the REST
framework in a more timely manner.

\subsection{Discord}
The Discord server has been used daily by nearly all groups. It has worked
great as long as at least one person from every group has been online, as it
has allowed us to communicate with the other groups. Discord was effective
because most of the groups have had members online after normal work hours, as
such the communication has not only been while the groups are in the group rooms.
One of the issues with Discord was that there were too many channels at the
beginning of the semester, which were removed later. Another issue is that the
some of the labeled channels are not used for their intended purpose. It is
also problematic when some conversations could fit into multiple channels. The
General, REST, and Server channels have been affected by this, and it is
sometimes difficult to find the correct conversation. One of the good things
about our usage of Discord was that SCRUM had their own channel, which they
could use to deliver important messages.

\subsection{Face to Face}
This form of communication has become increasingly used as the project has
neared the end. It has allowed us to find solutions to problems that were caused by
other libraries far faster than otherwise. A prime example of this was when the
\textc{REST-API-Client-Lib} library was created when we tried to include it as
a dependency it created a lot of problems in the intermediate files. It was
concluded that this problem was due to \textc{REST-API-Client-Lib} being an
application and not a library.

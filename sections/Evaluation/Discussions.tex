\chapter{Discussion}
The purpose of this chapter is to discuss the GIRAF project as a whole, in order
to determine what problems occoured during the project, and what could have been
done differently in order to avoid them. As such, the following sections will
each discuss one such problem, and present our opinion on what went wrong.

\section{Late Implementation}
When we originally began working on GIRAF, the server
architecture group (SW615) updated the firmware on the server, and noticed that
the server would no longer operate at all. By individual research, and by
reading the previous groups documentation, they determined that the previous
groups had used a ``hack'' or loophole in order to get the server to run. A
hack, which was fixed in later versions of the servers firmware. Based on this
problem, all of the project groups unilitarally determined to replace the old
server and the data models upon it. This choice to replace the entirety of
the server/database backend has led to the largest problem for developing GIRAF
in this semester.\nl

Since the server/Database backend is essential to all parts of developing the
applications, we have been unable to commence the application development in
full before the beta version of the REST framework was finished. Based on the
fact that the server groups vastly underestimated the complexity and time
requirement of researching and implementing an entirely new server/database
backend, this beta version of the REST framework was first ready for limited use
at the end of the third sprint.\nl





Additionally, many essential parts of the
framework was changed multiple times or was simply not in an acceptable state. This meant that
most of the development was done in the last two weeks of sprint four. Most of
our work during the first three sprints were rendered invalid, as the REST
framework fundamentally changed how the respective classes functioned.

\section{REST}
Late implementation\nl
Incompatible\nl

\section{Sprint Meetings}

\section{General Communication}
For genral communication during the sprints we have this year used both weekly
SCRUM meetings, a Discord channel / server and also walking from room to room.

\subsection{SCRUM meetings}

\subsection{Discord}
The Discord server have been used daily by nearly all groups. It has worked
great as long as at least one from every group is online such that it is
possible to communicate with every group. Discord was also effective because m

\subsection{Room to room}


\chapter{Discussion}
The purpose of this chapter is to discuss the GIRAF project as a whole, in order
to determine what problems occoured during the project, and what could have been
done differently in order to avoid them. As such, the following sections will
each discuss one such problem, and present our opinion on what went wrong.

\section{Late Implementation}
When we originally began working on GIRAF, the server
architecture group (SW615) updated the firmware on the server, and noticed that
the server would no longer operate at all. By individual research, and by
reading the previous groups documentation, they determined that the previous
groups had used a ``hack'' or loophole in order to get the server to run. A
hack, which was fixed in later versions of the servers firmware. Based on this
problem, all of the project groups unilitarally determined to replace the old
server and the data models upon it. This choice to replace the entirety of
the server/database backend has led to the largest problem for developing GIRAF
in this semester.\nl

Since the server/Database backend is essential to all parts of developing the
applications, we have been unable to commence the application development in
full before the beta version of the REST framework was finished. Based on the
fact that the server groups vastly underestimated the complexity and time
requirement of researching and implementing an entirely new server/database
backend, this beta version of the REST framework was first ready for limited use
at the end of the third sprint.\nl





Additionally, many essential parts of the
framework was changed multiple times or was simply not in an acceptable state. This meant that
most of the development was done in the last two weeks of sprint four. Most of
our work during the first three sprints were rendered invalid, as the REST
framework fundamentally changed how the respective classes functioned.

\section{REST}
Late implementation\nl
Incompatible\nl

\section{Sprint Meetings}
The sprint meetings took place at the end of each sprint. The goal of the
meetings were to give a brief presentation of what each group has worked on,
discuss the previous sprint and then create tasks.\nl

\subsection{Presentation}
The presentation phase remained unchanged throughout the semester, and has in
hindsight not provided much use. The brief presentation from each group has not
had much meaning to the other groups, as there has been little understanding of
what it means. While there were important information, much of it was not in a
context that made it easily accessable for other groups. A better way could have
been to provide a presentation on the status of certain important parts, and
explain slightly more detailed what is missing or why it is delayed.\nl

\subsection{Sprint Discussion}
The discussion of the previous sprint phase remained unchanged as well,
though this has not been a problem. This phase has provided a good way to
visualize the general progress of the project, and as the sprint is task based
has allowed us to find problems with some tasks.\nl

\subsection{Task Generation}
The tasks created by focusing on a specific goal for the sprints, such as a
series of user stories. This process were changed after the first and second
sprint. The groups at the first sprint were divided into two main groups of
server and application. This was changed in sprint 2, such that groups now
consisted of members from every group. The idea was to better generate tasks and
see how they might depend on tasks from other groups. This did not work as there
was not a lot tasks to generate that involved both server and application, which
resulted in a lot of silence from some group members. It was changed back to the
original format in sprint 3. As a final note it could be considered beneficial
for each project group to prepare a list of tasks they want to work on, this
would save a lot of time and would allow the individual task to be discussed
more.

\section{General Communication}
For genral communication during the sprints we have this year used both weekly
SCRUM meetings, a Discord server and also walking from room to room.

\subsection{SCRUM meetings}
The SCRUM meetings have not changed during the project. This is fine as they
have worked well in providing a small overview of what other groups are working
on. This has been especially helpful to give a distinct impression of what each
of the server groups are working on. It has helped with sharing information in
regards to how \textc{REST} was going to influence the application.

\subsection{Discord}
The Discord server have been used daily by nearly all groups. It has worked
great as long as at least one person from every group has been online, as it
has allowed us to communicate with the other groups. Discord was effective
because most of the groups have had members online after normal work hours, as
such the communication has not only been while the groups is in the grouprooms.
One of the issues with Discord was that there was too many channels at the
beginning of the semester, which were removed later. Another issue is that the
some of the labeled channels are not used for their intended purpose. It is
also problematic when some conversations could fit into multiple channels. The
General, REST and Server channels have been affected by this, and it is
sometimes difficult to find the correct conversation. One of the good things
about our usage of Discord was that SCRUM had their own channel, which they
could use to deliver important messages.

\subsection{Room to room}
This form of communication has become increasingly used as the project has
neared the end. It has allowed us to find solutions to problems that were caused by
other libraries far faster than otherwise. A prime example of this was when the
\textc{REST-API-Client-Lib} library was created, when we tried to include it as
a dependency it created alot of problems in the intermediate files. It was
concluded that this problem was due to \textc{REST-API-Client-Lib} being an
application and not a library.

\chapter{Conclusion}
Throughout the project, we have been working in collaboration with the 6 other
bachelor groups, in order to further develop the GIRAF suite of Android
applications. As such, this chapter will be used to conclude on the final state
of the project, and the success of the project goals.\nl

To make an initial conclusion on the success of the project; while we have been
unable to finalize GIRAF to such a state that it can be distributed to
customers, we have managed to complete most of the goals for our specific group.
Our group has been able to finish the \lapp, with the exception of some
functionality, which can't be implemented yet, as it lacks database support,
and a final fixed version of the \rlib.\nl

The following sections will be used to conclude on the individual system parts,
in relation to their current state, and what improvements and changes we have
made to them during the span of the project.

\section{Launcher}
At the beginning of the project in \autoref{GirafProblems}, we analyzed the
\lapp\ application, and determined a number of problems which needed to
be solved. The problems and the current state of their solution is presented below.

\subsection*{The Launcher is very unresponsive when starting up, and takes up to
a minute before we can access the home screen}
This problem has been solved by switching to the new REST framework, as the main
problem which causes the slow loading was the necessity to retrieve a large
amount data from the server and store them in files and \ttt{SharedPreferences}.
Currently, the \lapp\ is capable of presenting the login screen in about a
second, and logging in is just as quick.
While we still retrieve data from the server it is only data relating to the
given user, the new system is optimized for this purpose and much quicker than
the old system.

\subsection*{The Launcher experience frequent crashes, especially when switching
between activities} In the process of creating a new login system, and changing
the system to use the new \rlib, we have rewritten large parts of the
classes in which these crashes occurred. As such, these crashes no longer
happen. There were some crashes caused to exceptions which we were unable to
identify the source of, but due to the implementation of a \ttt{try-catch}
structure, we have been able to remove the effect of these issues.

\subsection*{The Launcher only loads user-data on startup. This can lead to
outdated data being used} By switching to the new REST framework, we have
reworked the \lapp\ to request an up-to-date \ttt{User} object when we need to
use the relevant data. As such, the system should make more requests to the server,
but we are guaranteed that the data we use are up-to-date.

\subsection*{The Launcher uses the DB-lib library for server communication and
needs to implement the new REST framework, which should be developed during the
project} Throughout the project, we have worked with the other groups in order
to specify design requirements for the REST framework. Based on this effort, we
have been able to fully replace all functionality from the \ttt{DB-Lib} library
in the \lapp. While we have finished the implementation on our end, \rlib\
is still missing some features which we need in order to finish the \lapp.

\section{PictoSearch-Lib}
At the beginning of the project in \autoref{GirafProblems}, we analyzed the
\plib\ library and determined a number of problems which needed to be solved.
The problems and the current state of their solutions are presented below. It
should be noted, that while this library was originally our responsibility, it
was handed over to group SW612 at the start of the fourth sprint, as they lacked
work, and we were busy working on the \lapp.

\subsection*{Searching for pictograms can take a very long time, and sometimes
it will get stuck and never finish} We determined that this situation was caused
by the client sending a request each time a character was entered into the
search field. This caused long search strings to send an unreasonable amount of
requests to the server. A potential solution to this problem was to require the
user to press a search button, or press enters on their keyboards. We have been
unable to test this solution, as the responsibility for the library was passed
on to group SW612.
  
\subsection*{Some search-inputs crashes the application, namely special
characters and commas} This problem was caused by the server returning null
whenever we searched for pictograms where none contained the specific character.
This problem was fixed by introducing a null-check before displaying the
pictograms on the screen.
  
  
\subsection*{The library uses the DB-lib library for server communication and
needs to implement the new REST framework, which should be developed during the
project} Due to our focus on the \lapp\ application in the third and fourth
sprint, we handed the \plib\ library over to group SW612. As such, they
have attempted to implement the new \rlib\ but has ultimately been
unable to succeed. Currently, the server should contain all information required
by the library, but the library is incapable of retrieving pictograms. Based on
group SW612 reports, they were unable to implement the asynchronous nature of
the REST framework into the library's search for pictograms such they can be
returned.

\section{Giraf-Component-Lib}
At the beginning of the project in \autoref{GirafProblems}, we analyzed the
\ttt{Giraf-Component-Lib} library, and determined a number of problems which
needed to be solved. The problems and the current state of their solution is presented
below.

\subsection*{The library contain large amounts of deprecated code and classes,
which makes the library incomprehensible} While we determined this to be a
substantial problem, we have chosen not to remove any of the code which we
assume to be deprecated. This is because GIRAF consists of a large number of
applications, which we have not worked on in this semester. As any of these
applications could potentially make use of the different classes in the library,
we have deemed it safer to leave them for now.

\subsection*{The library uses the DB-lib library for server communication, and
needs to implement the new REST framework, which should be developed during the
project.} The implementation of \rlib\ into the \clib\ library
has been a success, in that we have managed to implement it in all of the
classes used by the \lapp.

\section{Final Conclusion}
In the end, while we have managed to complete the goals for our group, we have
not been able to complete the overall project goal, which was:\nl

\say{Create a finalized version of the Launcher and Weekday Planner, such that
the GIRAF system is ready for use, and can be released.}\nl

The inability to reach this goal has been caused by the incomplete state of the
REST framework, which has prevented us from implementing all of the required
client-to-server communication. While this is the case, the REST framework is
well documented, and we hope that next year's students will be able to finish
its development and make a new release version of GIRAF. 



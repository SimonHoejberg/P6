\chapter{Conclusion}
This chapter will be used to conclude on the final state of the project, and the
success of the project and sprint goals. \\
Throughout the project, we have been working in collaboration with the 6 other
bachelor groups, in order to further develop the GIRAF suite of Android
applications.\nl

To make an initial conclusion on the success of the project; while we have been
unable to finalize GIRAF to such a state that it can be distributed to users, we
have managed to complete a large part of the project and sprint goals.
Additionally, our group has been able to finish the Launcher, with the exception
of some functionality, which can't be implemented yet, as it lacks the correct
data-objects on the database.\nl

The following sections will conclude to the individual system parts, in relation
to their current state, and what improvements and changes we have made to them
during the span of the project.\nl

In the final state of GIRAF, we have mostly finished our goals, but based on
problems with the finalization of the REST framework, we have been unable to
implement all of the features which are necessary in order to make a realase
version of the GIRAF system.

\section{Launcher}
At the beginning of the project in \autoref{GirafProblems}, we analyzed the
launcher application, and determined a number of problems which needed to be
solved. The problems and the current state of their solution is presented
below.

\subsection*{The launcher is very unresponsive when starting up, and takes up to
a minute before we can access the main screen}
 This problem has been solved by
switching to the new REST framework, as the main problem which cause the slow
loading, was the necessity to retrieve a large amount data from both the server
and from local sharedPreferences. Currently, the launcher is capable of
presenting the login screen in about a second, and loggin in is just as quick.
While we still retrieve all data relating to the given user, the new system is
optimized for this purpose, and much quicker than the old system.

\subsection*{The launcher experience frequent crashes, especially when switching between
activities}
In the process of creating a new login system, and changing the
system to use the new REST framework, we have rewritte large parts of the
classes in which these crashes occoured. As such, these crashes no longer
happen. There were some crashes caused to exceptions which we were unable to
identify the source of, but due to the implementation of a try-catch structure,
we have been able to prevent these issues.

\subsection*{The launcher only loads user-data on startup. This can lead to outdated data
being used}
By switching to the new REST framework, we have reworked the
Launcher to request an up-to-date User object when we need to use the relevant
data. As such, the system should make more requests to the server, but we are
guaranteed that the data we use are up-to-date.

\subsection*{The launcher uses the db-lib library for server communication, and needs to
implement the new REST framework, which should be developed during the project}
Throughout the project, we have worked with the other groups in order to
specify design requirements for the REST framework. Based on this effort, we
have been able to fully replace all functionality from the db-lib library in the
launcher. While we have finished the implementation on our end, the server is
sill missing some end-points, which results in us being unable to retrieve the
last bits of data which we need in order to finish the launcher.

% 
% \begin{table}[H]
% \centering
% \begin{tabular}{|p{5cm}|p{9cm}|}
% \hline
% The launcher is very unresponsive when starting up, and takes up to a minute
% before we can access the main screen. &  This problem has been solved by
% switching to the new REST framework, as the main problem which cause the slow
% loading, was the necessity to retrieve a large amount data from both the server
% and from local sharedPreferences. Currently, the launcher is capable of
% presenting the login screen in about a second, and loggin in is just as quick.
% While we still retrieve all data relating to the given user, the new system is
% optimized for this purpose, and much quicker than the old system. \\\hline
% The launcher experience frequent crashes, especially when switching between
% activities. & In the process of creating a new login system, and changing the
% system to use the new REST framework, we have rewritte large parts of the
% classes in which these crashes occoured. As such, these crashes no longer
% happen. There were some crashes caused to exceptions which we were unable to
% identify the source of, but due to the implementation of a try-catch structure,
% we have been able to prevent these issues. \\\hline 
% The launcher only loads user-data on startup. This can lead to outdated data
% being used. & By switching to the new REST framework, we have reworked the
% Launcher to request an up-to-date User object when we need to use the relevant
% data. As such, the system should make more requests to the server, but we are
% guaranteed that the data we use are up-to-date. \\\hline
% The launcher uses the db-lib library for server communication, and needs to
% implement the new REST framework, which should be developed during the project.
% & Throughout the project, we have worked with the other groups in order to
% specify design requirements for the REST framework. Based on this effort, we
% have been able to fully replace all functionality from the db-lib library in the
% launcher. While we have finished the implementation on our end, the server is
% sill missing some end-points, which results in us being unable to retrieve the
% last bits of data which we need in order to finish the launcher. \\\hline
% \end{tabular}
% \caption{Conclusion for the Launcher application} 
% \label{LConcT}    
% \end{table} 

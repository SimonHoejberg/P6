\chapter{Conclusion}
Throughout the project, we have been working in collaboration with the 6 other
bachelor groups, in order to further develop the GIRAF suite of Android
applications. As such, this chapter will be used to conclude on the final state
of the project, and the success of the project goals.\nl

To make an initial conclusion on the success of the project; while we have been
unable to finalize GIRAF to such a state that it can be distributed to users, we
have managed to complete most of the goals for our specific group. Our group has
been able to finish the Launcher, with the exception of some functionality,
which can't be implemented yet, as it lacks the correct data-objects on the
database.\nl

The following sections will conclude to the individual system parts, in relation
to their current state, and what improvements and changes we have made to them
during the span of the project.

\section{Launcher}
At the beginning of the project in \autoref{GirafProblems}, we analyzed the
launcher application, and determined a number of problems which needed to be
solved. The problems and the current state of their solution is presented
below.

\subsection*{The launcher is very unresponsive when starting up, and takes up to
a minute before we can access the main screen}
 This problem has been solved by
switching to the new REST framework, as the main problem which cause the slow
loading, was the necessity to retrieve a large amount data from both the server
and from local sharedPreferences. Currently, the launcher is capable of
presenting the login screen in about a second, and loggin in is just as quick.
While we still retrieve all data relating to the given user, the new system is
optimized for this purpose, and much quicker than the old system.

\subsection*{The launcher experience frequent crashes, especially when switching between
activities}
In the process of creating a new login system, and changing the
system to use the new REST framework, we have rewritte large parts of the
classes in which these crashes occoured. As such, these crashes no longer
happen. There were some crashes caused to exceptions which we were unable to
identify the source of, but due to the implementation of a try-catch structure,
we have been able to prevent these issues.

\subsection*{The launcher only loads user-data on startup. This can lead to outdated data
being used}
By switching to the new REST framework, we have reworked the
Launcher to request an up-to-date User object when we need to use the relevant
data. As such, the system should make more requests to the server, but we are
guaranteed that the data we use are up-to-date.

\subsection*{The launcher uses the db-lib library for server communication, and needs to
implement the new REST framework, which should be developed during the project}
Throughout the project, we have worked with the other groups in order to
specify design requirements for the REST framework. Based on this effort, we
have been able to fully replace all functionality from the db-lib library in the
launcher. While we have finished the implementation on our end, the server is
sill missing some end-points, which results in us being unable to retrieve the
last bits of data which we need in order to finish the launcher.

\section{Pictosearch-lib}
At the beginning of the project in \autoref{GirafProblems}, we analyzed the
pictosearch-lib library, and determined a number of problems which needed to be
solved. The problems and the current state of their solution is presented
below. It should be noted, that while this library was originally our
responsibility, it was handed over to group SW612 during the thrid sprint, as
they lacked work, and we were busy working on the launcher. 

\subsection*{Searching for pictograms can take a very long time, and sometimes
it will get stuck and never finish}
We determined that this situation was caused by the client sending a request
each time a character was entered into the search field. This caused long search
strings to send an unreasonable amount of requests to the server. A
potential solution to this problem was to require the user to press a search
button, or press enter on their keyboards. We have been unable to test this
solution, as the responsibility for the library was passed on to group SW612.
  
\subsection*{Searching happens synchronously on the GUI thread, which prevents
user input while it is searching}
This problem should be solved by implementing the new REST framework, as it
requires server requests to be  asynchronous. This means that the requests are
not handled on the GUI thread, and that the GUI will not have to wait for the
request to finish.
  
\subsection*{Some search-inputs crashes the application, namely special
characters and commas}
This problem was caused by the server returing null whenever we searched for
pictograms where none contained the specific character. This problem was fixed
by introducing a null-check before displaying the pictograms on the screen.
  
  
\subsection*{The library uses the db-lib library for server communication, and
needs to implement the new REST framework, which should be developed during the
project}
Due to our focus on the launcher application in the third and fourth sprint, we
handed the pictoserach-lib library over to group SW612. As such, they have
attempted to implement the new REST framework, but has ultimately been unable to
succeed. Currently, the server should contain all information required by the
library, but the library is incapable of retrieving pictograms. Based on group
SW612 reports, they were unable to implement the asynchronous nature of the REST
framwork into the previously synchronous nature of the library's search.

\section{Giraf-component-lib}
At the beginning of the project in \autoref{GirafProblems}, we analyzed the
giraf-component-lib library, and determined a number of problems which needed to
be solved. The problems and the current state of their solution is presented
below.

\subsection*{The library contain large amounts of deprecated code and classes,
which makes the library incomprehensible}
While we determined this to be a substantial problem, we have chosen not to
remove any of the code which we assume to be deprecated. This is beacuse GIRAF
consists of a large number of applications, which we have not worked on in this
semester. As any of these applications could potentially make use of the
different classes in the library, we have deemed it safer to leave them for now.

\subsection*{The library uses the db-lib library for server communication, and
needs to implement the new REST framework, which should be developed during the
project.}
The implementation of the REST framework into the component library has been a
success, in that we have managed to implement it in all of the classes used by
the Launcher.

\section{Final Conclusion}
In the end, while we have manged to complete the goals for our group, we have
not been able to complete the overall project goal, which was:\nl

\say{Create a finalized version of the Launcher and Weekday Planner, such that
the GIRAF system is ready for use, and can be released.}\nl

The inability to reach this goal has been caused by the incomplete state of the
REST framework, which has prevented us from implement all of the required
client-to-server communication. While this is the case, the REST framework is
well documented, and we hope that next years student will be able to finish
its development, and make a new release version of GIRAF. 



\chapter{Future Work}\label{fwork}
In this chapter, we will present our ideas for future delopment of the
\lapp\ and \clib. All the features and bugs we mention in this chapter have been
presented to SW611 aswell, for the creation of tasks for next year's students on
the GIRAF project.

\section{Redesigns}
During the implementation of \rlib\ in \clib, see
\autoref{T836_T868_T870}, we discovered that many classes in the library were
either deprecated, unusable, or implemented in a bad way. The number of classes
was too large to effectively tackle in sprint 4, and as such \clib\ should be
redesigned with regards to the GUI classes. That is, their design should be
changed to be in line with \ttt{GirafPopupDialog} which uses listeners instead
of the current implementation which uses interfaces and methods calls.
The classes should also be analyzed to figure out which ones are actually
needed, as we have worked with the library the entire semester and we do not
know which of the classes are actually used, and for what.

\section{Addtional Features}
During sprint 4 it was determined that there should be another approach to
logging in, where the user could input a department's login information, which
should give access to all guardians in that institution. As seen in
\textbf{T889} in \autoref{S4Invalid}, this task was unfinished, and this feature
should be finalized when departments are implemented on REST framework level.
The idea is that the departments should be able to log in, and then a
\ttt{GirafProfileSelector} should appear, where the user can choose between
every guardian from that department. When a guardian is chosen, the \lapp\
should log in as that guardian.\nl

It could also be a good idea to reimplement the previous setting, which made it
such that the spinning giraffe head on the login screen should only spin if the
user has it enabled. Like greyscale, this should be implemented as a switch as
seen in \autoref{grayScale}. The server should already support this
because of our specification of the settings page, which we made before
implementing the REST framework in the \textc{Launcher}. As we first retrieve a
\ttt{User}'s \ttt{Settings} object after the login, we can use
\ttt{SharedPreferences} in order to store the settings of the last \ttt{User}.

\section{Bugs and Missing Features}
During sprint 4 we discovered a few bugs and missing features, which are not
critical in size or effect, but smaller things which can improve the
\textc{Launcher} and \clib.

\subsection{Launcher}
The bugs and missing features we found in the \lapp\ are presented below:

\subsubsection{Dialogs should be in grayscale}
When testing grayscale in the settings page of the \textc{Launcher}, is was
found that it did not effect dialog boxes. This should be fixed such that the
dialogs respect the grayscale setting. For the \textc{Launcher} this means
that the boxes should have a way of knowing the setting.

\subsubsection{Implement grayscale in login screen}
As with the \textc{Launcher} the login screen should also be in
grayscale, but the problem here is that the user object does not exist before
they are logged in, and when they are logged in, the app changes to the home screen. So
to implement this feature an idea could be to use a \ttt{SharedPreferences} to
store the boolean for grayscale for the last logged in \ttt{User}. Doing this
should allow the login screen to be in grayscale as long as it is the same user that logs
in and does not change settings.

\subsubsection{Stop the changing of users through the settings page}
Currently, on the settings page it is possible to change which \ttt{User} you
set \ttt{Settings} for, but this has the side effect of changing which user is
logged in. This happens because when we change the user from the settings
page, a new \textc{LoginRequest} is made with the new user, and as such we get
a token/cookie for this new user. Then on the home page of the
\textc{Launcher}, we use an empty user \textc{GetRequest} which retrieves the
user for the cookie we have. To fix this issue, the changed to user should be
logged out and the old user logged in again.

\subsubsection{Make it possible to switch back to the guardian used in the
settings page without logging out} Currently, in the settings panel, it is
possible to change to a user for which the logged in guardian is a guardian of.
However, it is not possible to return to the settings page of the guardian after
a change to a citizen. This should be fixed, by possibly implementing a button
that logs the citizen out, and the guardian in again.
  
\subsubsection{Fix such that settings are saved onto the server and are able to
be downloaded again} Currently, it is not possible to save settings on the
server. This bug exist because \rlib\ does not save the settings from the
\textc{ResourceRequest}, and therefore the settings downloaded at the start of
an \ttt{Activity} does not have the right settings and no matter how many times
the are sent, they still do not persist. The bug can be fixed when
the settings part of the REST framework works.

\subsubsection{Fix such that the apps a user can access are sent to the server}
This bug is a lot like the other settings bug, but this time it is not because
\\\rlib\ does not save them. The bug occurs when the \textc{ResourceRequest} is
given a \ttt{Settings} object which contains a list of apps a \ttt{User} can
access, it causes a \ttt{StackOverflowException}. The code which should create
this request is disabled and needs to be enabled when the \rlib\ is fixed.

\subsubsection{Create a better icon for missing apps}
Currently, the icon for missing apps looks like a ``synchronize'' icon which
does not fit the GIRAF design guidelines. This icon should be changed to an
icon that signals that the app will be downloaded from the Google Play Store.

\subsubsection{Find a better way to collect icons for apps}
Currently, the \textc{Launcher} loads the icons for the applications, by
iterating over the entire list of installed apps, and comparing their names with
the respective application. This means that loading icons could take a long
time. There should exist a better way to get the icon of the apps. This could
improve both the load times of the home screen and the settings page.

\subsubsection{Use XML onClick in more activities}
Currently, method calls originating from the login screen's buttons are defined
from the login screen's \ttt{XML} file. As such, they do not need to be defined
listeners in the code. This approach should also be used in the other
\ttt{Activities}, as we believe this makes the code easier to read, as the
listener is replaced with a single method. This can only be done for
\ttt{Activities} which have the buttons in XML, not when the button is added
though code.

\subsection{Giraf Component Library}
The only real bug we found in the \clib\ is that the remaining dialogs in
the \textc{Launcher} should support grayscale. Given that the only
necessary information is the user's boolean for grayscale, this feature should
exist in all new and old dialogs.


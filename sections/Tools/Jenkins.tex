\subsection{Jenkins and Artifactory}
In order to automate and simplify building the GIRAF software releases, the
GIRAF project is organized using a combination of the tools Jenkins and
Artifactory.\nl

Jenkins is a build tool which allows us to automate the build
process \citep{Jenkins}. We use it to take the GIRAF code base from the
relevant Git repositories, and compile them into the respective binary files.
After compilation, Jenkins uploads the compiled applications and libraries to
Artifactory's repository. This repository is used to contain the successful
builds of the applications and libraries. The benefit of using Artifactory is,
that it is not necessary for each member of the GIRAF project to have a local
copy of a specific version of the application's /librarie's dependencies
repository. Instead, during a local build a request can be sent to
Artifactory, which returns the dependencies for the apps and libaries such that they can use
the newest stable version.

% Jenkins is used to build the different tools, apps and libaries from the git
% repositories and then publish them to articatory. Artifactory is used to get the
% dependencies for the apps and iibaries such that they can use the newest stable
% version.
\subsection{Git}\label{GitTool}
The entirety of the GIRAF code base is divided into a number of repositories,
where each one contains either an application or a library. Following the work
done by the last year's students, the GIRAF repositories  were all contained in
a single organization called \textc{Giraf}, which we did not have permission to
edit. As such, we have chosen to fork, reorganize, and rename the GIRAF
repositories, and split them into a number of new organizations, such that the
Git-management group (SW610) can manage access to the different repositories
for each GIRAF group and member.


%  The last years students left the Giraf project in many different
% repositories on the Giraf git server. This year the different repositories have been forked to
% new repositories, because we did not have acces to last years repositories. The
% new forked repositories and renamed such that they are easiere to both find and
% manage, they have also been diveded in new git organisations such that one group 
% can manage access to the different repositories for each Giraf group and member.\\

The new Git organizations can be seen below:
\begin{itemize}
	\item Giraf17-Rest
	\begin{itemize}
  		\item Contains repositories relating to the REST framework, which is an
  		intermediate layer used for database communication.
	\end{itemize}
  	\item Giraf17-Tools
  	\begin{itemize}
  		\item Contains tools for the server
	\end{itemize}
  	\item Giraf17-AndroidLibs
  	\begin{itemize}
  		\item Contains libaries for the Android applications
	\end{itemize}
  	\item Giraf17-AndroidApps
  	\begin{itemize}
  		\item Contains the Giraf Android applications 
	\end{itemize}
\end{itemize}
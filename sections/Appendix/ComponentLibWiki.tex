\chapter{Component Lib}\label{WikiCompLib}
\textbf{IMPORTANT - Most classes should be redesigned. The ones which should not
be are, GrayscaleHelper, GirafPopupDialog, GirafUserItemView and
GirafPictogramItemView. The Library is cluttered with deprecated classes and
other classes that may or may not be used. It would
be beneficial to be remake the library from scratch.}\nl

This library contains many GIRAF GUI elements and classes for the graphics of
the applications. Most of the code is the same from 2016, the classes we have
worked on are:

\begin{itemize}
  \item GirafActivity which should restart the launcher on a crash but this
  does not currently work. Next year's students should fix this class.
  \item GrayscaleHelper which has the abillity to set the actvity's rootView to
  use grayscale, this works on an activity, but not on dialogs,  which next
  year's students should fix.
  \item GirafPopupDialog which is a new class, it has two GirafButtons. It is
  possible to have the GirafPopupDialog with either one or both buttons. It is
  also possible to give each of the buttons a title and a drawable (icon). It
  uses listeners instead of the method-call implemented using interfaces,
  (listerners should be the way to go for the rest of the component lib).
  \item GirafProfileSelectorDialog where we just made a quick fix so it can
  be used in the launcher, it should be redesigned.
  \item GirafUserItemView and GirafPictogramItemView which is used to show the
  user image and the pictogram,  we made the UserItemView class, because both
  pictogram and user used to behave in the same way, when it came to their
  image.
  
\end{itemize}

We have worked on switching from the old libraries to the REST libraries and as
such every class that used Profile was changed to User and old Pictogram was
changed to the new Pictogram class.

\subsubsection{Repository}
http://git.giraf.cs.aau.dk/Giraf17-AndroidLibs/giraf-component-lib | Giraf Component Lib

\subsubsection{Dependencies}
\begin{itemize}
  \item Android support v4 (com.android.support:support-v4:23.2.1)
  \item rest-api-client-lib
  \item rest-models
  \item volley (com.android.volley:volley:1.0.0)
\end{itemize}

